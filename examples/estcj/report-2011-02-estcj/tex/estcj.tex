% estcj.py

\documentclass[10pt,a4paper]{article}
\usepackage[body={16cm,25cm}]{geometry}
\usepackage{hyperref}
\hypersetup{colorlinks=true,linkcolor=blue}
\usepackage{graphicx}
\usepackage{listings}
\lstset{basicstyle=\ttfamily\scriptsize,identifierstyle=,keywordstyle=}
\usepackage{lscape}

%------------------------------------------------------------------
% Extra formatting for codes (from report style)
%------------------------------------------------------------------
% a couple horizontal bars to delimit embedded code
% the width suits that set above and
% the mathmode eliminates spaces between the three elements
\newcommand{\topbar}{\ensuremath{
    \rule{0.1mm}{2.0mm} \rule[2.0mm]{159.5mm}{0.1mm} \rule{0.1mm}{2.0mm}
}}
\newcommand{\bottombar}{\ensuremath{
    \rule{0.1mm}{2.0mm} \rule{159.5mm}{0.1mm} \rule{0.1mm}{2.0mm}
}}

%------------------------------------------------------------------
% Title page information
%------------------------------------------------------------------

\title{
  Estimation of high-enthalpy flow conditions
  for simple shock and expansion processes.
}
\author{
  Mechanical Engineering Report 2011/02 \\
  P.~A.~Jacobs, R.~J.~Gollan, D.~F.~Potter, \\
  F.~Zander, P.~Blyton and W.~K.~Y.~Chan \\
  Centre for Hypersonics\\
  The University of Queensland
}

\begin{document}
\maketitle

\baselineskip = 1.2pc

\begin{abstract}
This report presents the software tools that we have built to do simple
flow process calculations for ideal gases and gases in chemical equilibrium.
The software comes in the form of a library for the most fundamental processes and
a couple of application programs for convenient calculation of 
the combined flow processes relevant to shock- and expansion-tube operation.
\end{abstract}

\bigskip
\tableofcontents

%------------------------------------------------------------------

\newpage
\section{Introduction}
%
ESTCj\,\footnote{Equilibrium Shock Tube Conditions, junior} 
began as a reimplementation of the ideas in the ESTC code\,\cite{mcintosh_70}
written by Malcom McIntosh in the late 1960s and 
the shock-tube-plus-nozzle (STN) code\,\cite{krek_jacobs_93} written in the early 1990s.
The new program\,\cite{jacobs_gardner_2003a} was started 
while PJ was on study leave at DLR Goettingen,
with a decision to delegate the equilibrium thermochemistry issues to the 
Gordon and McBride's Chemical Equilibrium Analysis (CEA) 
code\,\cite{gordon_mcbride_1994,mcbride_gordon_1996}.
With the thermochemistry provided by CEA, the ESTCj program had to be concerned
only with the smaller task of computing the flow changes across shocks and through
the steady nozzle expansion.

\medskip
Implementation was done in the Python programming language\,\footnote{http://www.python.org}
which was easy for end users to customize so the program tended to grow in an ad hoc fashion.
This report describes the current generation of the program, which has been refactored into
three layers:
\begin{enumerate}
 \item Thermochemical gas models for perfect gases and gases in thermochemical equilibrium.
 \item A library of functions for simple flow processes such as normal shocks, oblique shocks,
  steady and unsteady and expansions.
 \item A top-level code (actually called estcj.py) that coordinates the calling 
  of the flow-process functions using information provided by the user on the command line.
\end{enumerate}
One of the advantages of moving the flow-process calculations to a library is that 
they can be conveniently reused, 
as has been done for the NENZFr code\,\cite{doherty_etal_2012a}, for example.

\medskip
The following sections provide an overview of the new functions and their capabilities.
The detail is in the source code which we've tried to make modular and very readable.
Despite the code being central to this report, we put it in the Appendix so that there is
a reasonable chance that the reader might at least get the overview 
before being overwhelmed by detail and giving up.


\bigskip
\section{Thermochemical Models}
%
Perfect gas and a gas mixture in thermochemical equilibrium...

\bigskip
\section{Flow Process Calculations}
%
Cover all of McIntosh's cases and then some (conical flow, etc).
Need figures...

\bigskip
\section{ESTCj program}
%
The application-level code is essentially a command-line interpreter
that writes the results of the requested calculation to the standard-output stream
by default.
It's easiest to get a reminder of the available settings by asking for ``help''
on the command line.

\medskip
\noindent\topbar
\lstinputlisting[language={},breaklines=true]{../notes/estcj-help.txt}
\bottombar

\bigskip
\subsection{Example calculations}

\bigskip
\subsubsection*{Example of use for T4 condition}
%
A typical flow condition for the T4 shock tunnel
with the Mach 4 nozzle may be computed using:
\begin{verbatim}
$ estcj.py --task=stn --gas=air --T1=300 --p1=125.0e3 --Vs=2414 --pe=34.37e6 --ar=27.0
\end{verbatim}
The full output is included below, where you should see that
this condition has an enthalpy of 5.43 MJ/kg and the nozzle-exit condition
has a pressure of 93.6 kPa and a static temperature of 1284 degrees K,
with a flow speed of 2.95 km/s.
Note that we have selected to stop the expansion at a particular nozzle area ratio.
Alternatively, we may stop the expansion at a particular Pitot pressure by specifying
a suitable ratio for the option \verb?--pp_on_pe?.

\medskip
\noindent\topbar
%\begin{landscape}
\lstinputlisting[language={},breaklines=true]{../notes/t4-condition-transcript.txt}
%\end{landscape}
\bottombar

\medskip
Subset calculations can be done by selecting a different task.
The following sections show the subproblems that can be exercised from the command line.
These calculations can also be done inside other programs by calling 
the relevant \verb?gas_flow.py? functions.

\bigskip
\subsubsection*{Flow conditions behind an incident shock}
%

\bigskip
\subsubsection*{Reflected-shock condition}
%

\bigskip
\subsubsection*{Pitot pressure calculation}
%


\bigskip
\subsubsection*{Cone surface pressure calculation}
%


\bigskip
\section{Custom application programs}
%
\subsection{Classic shock tube}
%
As an example of building a custom application, consider the idealized shock tube 
with equal area sections separated by a diaphragm.
See for example, Section 7.8 (Shock tube relations) in Anderson's text\,\cite{anderson_82}
for a discussion based on perfect gas behaviour.

\medskip
\noindent\topbar
\lstinputlisting[language={}]{../../../eilmer3/2D/classic-shock-tube/classic_shock_tube.py}
\bottombar

\bigskip
\subsection{Idealized expansion tube}
%
As a second example of building a custom application, consider the idealized expansion
of the test gas in an expansion tube\,\cite{trimpi_62}.

\medskip
\noindent\topbar
\lstinputlisting[language={}]{../../../../lib/cfpylib/gasdyn/classic_expansion_tube.py}
\bottombar

%------------------------------------------------------------------

\newpage
\bibliographystyle{unsrt}
\bibliography{../bibtex/pj,../bibtex/shocktube,../bibtex/gas,../bibtex/gas_dynamic.bib}

%--------------------------------------------------------------------
% Appendices
%--------------------------------------------------------------------

\newpage
\appendix
\section{Source code for gas models}
%
\subsection{ideal\_gas.py}
\label{ideal-gas-py}
%
Thermodynamic functions for an ideal gas.

\noindent\topbar
\lstinputlisting[language={}]{../../../../lib/cfpylib/gasdyn/ideal_gas.py}
\bottombar

\newpage
\subsection{libgas\_gas.py}
\label{libgas-gas-py}
%
Thermodynamic functions for the gas model used by Eilmer3.

\noindent\topbar
\lstinputlisting[language={}]{../../../../lib/cfpylib/gasdyn/libgas_gas.py}
\bottombar

\newpage
\subsection{cea2\_gas.py}
\label{cea2-gas-py}
%
Thermodynamic functions for the equilibrium gas model backed by CEA2.

\noindent\topbar
\lstinputlisting[language={}]{../../../../lib/cfpylib/gasdyn/cea2_gas.py}
\bottombar

\newpage
\section{Source code for flow process calculations}
%
\subsection{ideal\_gas\_flow.py}
\label{ideal-gas-flow-py}
%
Basic flow relations for an ideal gas.

\noindent\topbar
\lstinputlisting[language={}]{../../../../lib/cfpylib/gasdyn/ideal_gas_flow.py}
\bottombar

\newpage
\subsection{gas\_flow.py}
\label{gas-flow-py}
%
Basic flow relations for a more general gas.

\noindent\topbar
\lstinputlisting[language={}]{../../../../lib/cfpylib/gasdyn/gas_flow.py}
\bottombar

\newpage
\section{Source code for ESTCj application}
\label{estcj-py}
%
Top-level application code.

\noindent\topbar
\lstinputlisting[language={}]{../../../../app/nenzfr/estcj.py}
\bottombar


\end{document}

