% just-enough-python.tex
\section{Just enough Python to be dangerous}
\label{python-notes-sec}
When \texttt{e3prep.py} is run, the first thing that happens is that a number of Eilmer3-specific
modules are loaded and a number of classes are defined to assist with the definition of flow and geometry.
The user's input file is then read in and executed by the Python interpreter
in the context of these predefined classes and functions.
Since the input file has to be valid Python code, it's worth knowing a little about the language itself.
We will discuss the features of Python using examples from the periodic shear layer input file on
page \pageref{psl-py-file}.

\medskip
Python is a statement-based language where indentation is used to define the block structure
of compound statements.
One of the implications of this significant whitespace is that the first statement in the user's input
file must start right at the beginning of the line.
That is, it must not be indented.
The first couple of lines in the periodic shear layer input file are:\\
\texttt{\# psl.py\\
gdata.title = "Periodic shear layer"\\
}
The comment line, starting with the sharp (or hash) character is ignored and the first statement assigns
a string literal to the title.

\medskip
Single statements, such as assignment statements, may extend over several lines if they are continued
by one of:
\begin{itemize}
 \item a backslash ($\backslash$) at the end of each incomplete line;
 \item an open left parenthesis, bracket or brace without the corresponding closing parenthesis, bracket or brace; or
 \item an open triple quote that has indicated the start of a multiline string.
\end{itemize}
The second of these is quite commonly seen in the example files because many of the function calls and object
constructors have a lot of arguments, some of which may be quite complex in themselves.
The following assignment statement, from near the end of the periodic shear layer input file, 
calls the \texttt{SuperBlock2D} object
constructor and then binds the resultant object to the name \texttt{superblk}.
\texttt{
\begin{tabbing}
superblk = SuperBlock2D( \= psurf=domain, nni=nnx, nnj=nny,\\
                         \> bc\_list=[SlipWallBC(),]*4,\\
                         \> fill\_condition=initial\_gas,\\
                         \> nbi=nbi, nbj=nbj, label="blk")
\end{tabbing}
}

\medskip
On the selection of names to bind to returned data objects, the usual rules apply.
Start the name with a letter from the alphabet and follow it with any number of letters, digits or 
underscores.
Don't use any of the following reserved words for your own names:
\begin{center}
\texttt{
\begin{tabular}{lllll}
and      & del      & from     & None    & try \\
as       & elif     & global   & not     & while \\
assert   & else     & if       & or      & with \\
break    & except   & import   & pass    & yield \\
class    & exec     & in       & print   & \\
continue & finally  & is       & raise   & \\
def      & for      & lambda   & return  & 
\end{tabular}
}
\end{center}
And, if you want to see the list of names that are predefined for the environment in
which your input file is interpreted, start \texttt{e3prep.py} with the \texttt{--show-names} option.
There are conventions that leading and trailing underscores are reserved for system names and that
names starting with an uppercase letter are class names.

\medskip
Control flow statements such are implemented as compound statements.
These start with an opening clause at the current indentation level.
This clause will start with a keyword, such as \texttt{if}, \texttt{while}, or \texttt{for},
and end with a colon.
The body of the compound statement will typically start on the following line, indented one level.
All statements at that level of indentation or more form part of the body of that compound statement.
There may be nested compounded statements and each level of indentation will be 4 spaces, by convention.

\medskip
The definition of a function is itself a compound statement and an example can be seen
in the periodic shear layer example (page \pageref{psl-py-file}) where the initial state of the gas 
is defined in the function \texttt{initial\_gas(x, y, z))} in the user's input file. 
Collections of functions are typically available as modules in Python.
These modules, or items from the modules may be ``imported''. 
By default, Python does not load a lot of modules so you will typically have to import math functions, for example.

\medskip
As well as the simple numerical data types of integers and floats, there are strings and more complex, 
structured data types built into the language.
These include tuples, lists and dictionaries.
\texttt{e3prep.py} also makes use of \texttt{numpy} arrays.

\medskip
You will make use of lists when defining collections of mass fractions and boundary conditions, for example.
A list literal is denoted by square brackets, with items separated by commas.
Lists are ordered collections of items that may be indexed, starting from zero.
Negative index values count backward from the end of the list.
The \texttt{for} loop is a convenient way of working through all the elements of a list.
In the periodic shear layer example above, the boundary conditions are specified as a list of 4
\texttt{SlipWallBC} objects.
The following code works through the list of blocks 
that had been returned by the SuperBlock2D constructor and makes the appropriate
connections for a periodic domain.
\texttt{
\begin{tabbing}
for \= j in range(nbj):\\
    \> connect\_blocks\_2D(superblk.blks[-1][j], EAST, superblk.blks[0][j], WEST)
\end{tabbing}
}
\noindent
Here, the call to the function \texttt{range} returns a list of integer values starting with 0,
going up to but not including the value bound to \texttt{nbj}.

\medskip
Dictionaries are collections of named objects.
They are a convenient way of setting species mass fractions, especially for a gas model
that has many species.
You may typically only have only one or a couple of species present in and particular inflow or 
initial gas condition as, for example, the literal dictionary \{\texttt{'He':0.1, 'air':0.9}\}
is used to set the mass fraction of helium to 0.1 and the mass fraction of air to 0.9.
Use of the dictionary has the benefit of making the input script somewhat self-documenting and you
don't have to remember the order in which the gas species were defined in the call 
to the \texttt{select\_gas\_model} function, earlier in the input file.

\medskip
More specialized data objects can be defined via classes, and \texttt{e3prep.py} does exactly that.
The name \texttt{gdata} is bound to an instance of the \texttt{GlobalData} class and contains
many attributes that set the configuration of the flow solver.
The user input file will use the already defined gdata object but will typically create new instances
of objects such as \texttt{Node} and \texttt{SuperBlock2D}.
It is often convenient to bind the reference returned to the newly created object to a name in the input script
so that it can be conveniently referenced in later statements.
In the periodic shear layer case, the \texttt{Node} objects are bound to names 
that are then used to construct the \texttt{Line} objects that are, in turn, 
used to define the rectangular flow domain.

\medskip
When working in Python it is possible to see what options are available to you with a particular
function or object using the \texttt{dir} command. This enables you to get a print out of the properties
and functions associated with the object. For example create a node and see what the \texttt{dir} output is.\\
\topbar\\
\texttt{a = Node(0.0,0.1,label='a')}\\
\texttt{print dir(a)}\\
\begin{scriptsize}
\texttt{['\_\_add\_\_', '\_\_class\_\_', '\_\_del\_\_', '\_\_delattr\_\_', '\_\_dict\_\_', '\_\_div\_\_', '\_\_doc\_\_',
 '\_\_format\_\_', '\_\_getattr\_\_',\\ '\_\_getattribute\_\_', '\_\_hash\_\_', '\_\_iadd\_\_', '\_\_idiv\_\_', 
'\_\_imul\_\_', '\_\_init\_\_', '\_\_isub\_\_', '\_\_module\_\_', '\_\_mul\_\_',\\ '\_\_neg\_\_', '\_\_new\_\_', 
'\_\_pos\_\_', '\_\_reduce\_\_', '\_\_reduce\_ex\_\_', '\_\_repr\_\_', '\_\_rmul\_\_', '\_\_setattr\_\_', 
'\_\_sizeof\_\_',\\ '\_\_str\_\_', '\_\_sub\_\_', '\_\_subclasshook\_\_', '\_\_swig\_destroy\_\_', '\_\_swig\_getmethods\_\_', 
'\_\_swig\_setmethods\_\_',\\ '\_\_weakref\_\_', 'clone', 'copy', 'label', 'mirror\_image', 'nodeList', 'norm', 
'rotate\_about\_zaxis', 'str',\\ 'this', 'transform\_to\_global', 'transform\_to\_local', 'translate', 'vrml\_str', 
'vtk\_str', 'x', 'y', 'z']}\end{scriptsize}\\
\bottombar\\

The last items in this list are the different options available to a Node object that can be used within the
prep file.
