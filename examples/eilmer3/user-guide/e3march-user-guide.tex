% e3march-user-guide.tex
% PJ, June 2014
%
\documentclass[12pt,a4paper,twoside]{article}
\usepackage[body={16cm,24.5cm}]{geometry}
\usepackage{hyperref}
\hypersetup{colorlinks=true,linkcolor=blue}
\usepackage{graphicx}
\usepackage{makeidx}
% \usepackage{showidx}
\usepackage{listings}
\usepackage{lscape}
\usepackage{longtable}
\usepackage{amsmath}
\usepackage{amssymb}
\usepackage{subfig}
\lstset{basicstyle=\ttfamily\scriptsize, showstringspaces=false, identifierstyle=, keywordstyle=}
\lstset{numbers=left, numberstyle=\tiny, stepnumber=1, numbersep=5pt}
\newcommand{\code}[2]{
 \hrulefill
 \scriptsize
 \lstinputlisting{#2}
 \hrulefill
 \vspace{2em}
 \normalsize
}
%\usepackage{booktabs}
%\usepackage{pgfplots} \pgfplotsset{compat=newest} % Doesn't work on LinuxMint 13; need something more recent.

%------------------------------------------------------------------
% a couple horizontal bars to delimit embedded code
% the width suits the page size set above and
% the mathmode eliminates spaces between the three elements
\newcommand{\topbar}{\ensuremath{
    \rule{0.1mm}{2.0mm} \rule[2.0mm]{159.5mm}{0.1mm} \rule{0.1mm}{2.0mm}
}}
\newcommand{\bottombar}{\ensuremath{
    \rule{0.1mm}{2.0mm} \rule{159.5mm}{0.1mm} \rule{0.1mm}{2.0mm}
}}
\newcommand{\topbarshort}{\ensuremath{
    \rule{0.1mm}{2.0mm} \rule[2.0mm]{149.5mm}{0.1mm} \rule{0.1mm}{2.0mm}
}}
\newcommand{\bottombarshort}{\ensuremath{
    \rule{0.1mm}{2.0mm} \rule{149.5mm}{0.1mm} \rule{0.1mm}{2.0mm}
}}
%------------------------------------------------------------------

\title{
    A block-marching {CFD} solver for compressible flow.
}
\author{
    Mechanical Engineering Report 2014/06\\
    Wilson Y. K. Chan ~ Peter A. Jacobs  ~ Fabian Zander\thanks{Institut fuer Raumfahrtsysteme
      Universitaet Stuttgart, Pfaffenwaldring 29, 70569 Stuttgart.}\\
    ~ Rowan J. Gollan and Luke Doherty\\
    Centre for Hypersonics, The University of Queensland, Brisbane, Australia.
}
\makeindex

\begin{document}
\maketitle

\begin{abstract}

\end{abstract}

\cleardoublepage
\tableofcontents

%------------------------------------------------------------------
\cleardoublepage
\baselineskip = 1.5pc

\section{Introduction}
%
Eilmer3 is a transient compressible flow solver that uses block-structured 
grids\,\cite{gollan_jacobs_2013a,jacobs_etal_2014a,jacobs_etal_2014b}.  
Since it is a simulation code that uses explicit updates for the solution 
of the full Navier-Stokes equations and all parts of the flow field are included in all time steps, 
the calculation can become very computationally expensive, especially when using fine grids.  
Further, if the primary interest is in the long-term or steady-state solution over the domain, 
much of the intermediate (transient flow) data is irrelevant and is then discarded.

\medskip
A large class of flow problems of interest to our research group may be described 
as mostly supersonic flow with a dominant direction and having duct-like boundary surfaces.  
Specific examples include supersonic wind-tunnel nozzles and scramjet inlets, combustors and thrust nozzles.  
For these flows, the upstream parts of the flow field settle early and remain unchanged 
while the downstream parts of the flow solution evolve to steady state.  
Of course, specialized CFD codes based on parabolized Navier-Stokes equations are available 
for producing fast solutions for this class of flow, however, 
they are limited in the presence of strong shock-wave boundary-layer interactions.

\medskip
The “block-marching” approach that we have developed for Eilmer3 has some advantages 
in that it retains the use of the full transient solver for 
the complete Navier-Stokes equations but accelerates the overall computation 
by concentrating on subsets of blocks, until they reach steady-state, 
marching in the dominant flow direction and building up the steady solution for the full domain. 
Our first attempt at making the flow simulation code into a form of space-marching code
was to sequence the processing of blocks within the main C++ sumulation code\,\cite{zander_etal_2011b},
however, that approach could not easily make use of many processors on a cluster computer. 
Instead, a higher-level \textit{coordinating} approach was coded 
as a set of shell scripts which coordinated separate computation jobs on subsets of blocks
and could make use of several processors by running these jobs on the local cluster computer\,\cite{chan_2012a}.  

\medskip
The present implementation has been rewritten in a modern scripting language (Python) 
and is essentially a coordinating program that delegates the actual work 
of solving the flow equations on subsets of the original grid to an underlying CFD solver.  
Behind the scenes, it generates a potentially large simulation, 
with many block-structured grids covering the full domain and organised in a regular array.  
It then works through the large array of blocks, one subset at a time, 
setting up a small transient calculation with adjusted boundary conditions and 
submitting that to the original transient solver for integration.  
On the completion of each sub-problem, the coordinating program marshalls the blocks 
of steady state data such that the full set of blocks is finally presented 
as a single solution field at the end of the process.  
A computational advantage of this approach is that we may continue to make use 
of a cluster computer to accelerate the individual subset calculations.  
As well, it requires no changes to the underlying flow solver, 
so it may be applied to any flow solver for which we can automate the writing of job files. 

\medskip
In the following sections, we will more completely describe the approach 
and provide verification and validation examples.  
% We will also provide an example of its application to nozzle flow calculation 
% and nozzle surface design, an application that benefits greatly 
% from having fast flow calculations available.

\section{Simulation Process -- Block Marching}

\clearpage
\section{Examples}

\subsection{Channel with circular-arc bump}
%
Classic case for testing inviscid solvers.  
Sudden changes at start and end of bump set off oblique shocks that propagate across blocks..


\subsection{Reacting flow in a supersonic streamtube}
%
From Fabs' report.  
We have to get our chemistry right...


\subsection{Laminar boundary layer on a flat plate}
%
Comparison with Shetz boundary-layer code.


\subsection{Turbulent boundary layer on a flat plate}
%
Coles flat plate, maybe.


\subsection{Mohammadian convex plate}
%
Laminar boundary layer with pressure gradient.


\subsection{Glancing shock interaction with a boundary layer}
%
3D case


\clearpage
\bibliographystyle{unsrt}
\bibliography{bibtex/pj,bibtex/computing,bibtex/gas_dynamic,bibtex/adm,bibtex/gas,bibtex/dan,bibtex/upwind,bibtex/wilson}

\clearpage
\appendix
\section{Source code}

\noindent
\code{}{../../../app/eilmer3/source/e3march.py}

\clearpage
\phantomsection % re-set the hyperref anchor so that TOC page number link is correct.
\addcontentsline{toc}{part}{\indexname}
\printindex
\end{document}
