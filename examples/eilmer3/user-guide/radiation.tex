% radiation.tex
\section{Radiation transport models}
\label{app:rt-models}
\index{radiation transport model!available models}

A variety of radiation transport models are implemented in Eilmer3:

\begin{itemize}
 \item Optically thin model
 \item Tangent slab model 
 \item Modified discrete transfer model
 \item Photon Monte-Carlo model
\end{itemize}

The radiation transport model is defined the \texttt{transport\_data} table of the radiation Lua input file.

\subsection{Optically thin model}

The optical thin radiation transport model is selected by setting the \texttt{transport\_model} field in the \texttt{transport\_data} field to \texttt{"optically thin"}.
The following code snippet gives an example of selecting and defining the parameters for the optically thin model:

\noindent \topbar
\begin{lstlisting}[basicstyle=\ttfamily\normalsize]
transport_data = {
   transport_model = "optically thin",
   spectrally_resolved = true
}
\end{lstlisting}
\bottombar

A description of the Lua input fields for the optically thin radiation transport model is given in Table~\ref {tab:OT-input}.

\begin{table}[h]
 \begin{center}
  \caption{Description of Lua input fields for the optically thin radiation transport model}
  \label{tab:OT-input}
\begin{tabular}{llp{8cm}}
 \hline \hline
 Field                                                   & Type                 &  Description \\ \hline
\texttt{spectrally\_resolved}            & \textit{bool}       &  Flag to request a spectrally resolved or unresolved determination of the radiative power density \\
\hline
\end{tabular}
\end{center}
\end{table}

\subsection{Tangent slab model}

The tangent slab radiation transport model is selected by setting the \texttt{transport\_model} field in the \texttt{transport\_data} field to \texttt{"tangent slab"}.
No other input parameters need to be set.
The following code snippet gives an example of selecting and defining the parameters for the tangent slab model:

\noindent \topbar
\begin{lstlisting}[basicstyle=\ttfamily\normalsize]
transport_data = {
   transport_model = "tangent slab"
}
\end{lstlisting}
\bottombar

\subsection{Modified discrete transfer model}

An implementation of the modified discrete transfer radiation transport model is selected by setting the \texttt{transport\_model} field in the \texttt{transport\_data} field to \texttt{"discrete transfer"}.
The following code snippet gives an example of selecting and defining the parameters for the discrete transfer model:

\noindent \topbar
\begin{lstlisting}[basicstyle=\ttfamily\normalsize]
transport_data = {
   transport_model = "discrete transfer",
   nrays = 32,
   clustering = "by volume",
   binning = "opacity",
   N_bins = 10
}
\end{lstlisting}
\bottombar

A description of the Lua input fields for the modified discrete transfer radiation transport model is given in Table~\ref {tab:DTM-input}.

\begin{table}[h]
 \begin{center}
  \caption{Description of Lua input fields for the modified discrete transfer radiation transport model}
  \label{tab:DTM-input}
\begin{tabular}{llp{8cm}}
 \hline \hline
 Field                        & Type              &  Description \\ \hline
\texttt{nrays}            & \textit{int}       &  Number of rays emitted per cell and per frequency interval \\
\texttt{clustering}    & \textit{string}  &  Ray clustering: \texttt{by volume}, \texttt{by area} or \texttt{none} \\
\texttt{binning}        & \textit{string}  &  Binning model: \texttt{opacity}, \texttt{frequency} or \texttt{none} \\
\texttt{N\_bins}         & \textit{int}       &  Number of bins (does not need to be set if \texttt{ binning = "none" }) \\
\hline
\end{tabular}
\end{center}
\end{table}

\subsection{Photon Monte-Carlo model}

The photon Monte-Carlo radiation transport model is selected by setting the \texttt{transport\_model} field in the \texttt{transport\_data} field to \texttt{"monte carlo"}.
The following code snippet gives an example of selecting and defining the parameters for the photon Monte-Carlo model:

\noindent \topbar
\begin{lstlisting}[basicstyle=\ttfamily\normalsize]
transport_data = {
   transport_model = "monte carlo",
   nrays = 512,
   clustering = "by area",
   absorption = "partitioned energy"
}
\end{lstlisting}
\bottombar

A description of the Lua input fields for the photon Monte-Carlo radiation transport model is given in Table~\ref {tab:PMC-input}.
Note that here \texttt{nrays} is the total number of rays emitted per cell, whereas for the discrete transfer model \texttt{nrays} is the number of rays emitted per cell per frequency interval.

\begin{table}[h]
 \begin{center}
  \caption{Description of Lua input fields for the photon Monte-Carlo radiation transport model}
  \label{tab:PMC-input}
\begin{tabular}{llp{8cm}}
 \hline \hline
 Field                        & Type              &  Description \\ \hline
\texttt{nrays}            & \textit{int}       &  Number of rays emitted per cell \\
\texttt{clustering}    & \textit{string}  &  Ray clustering: \texttt{by volume}, \texttt{by area} or \texttt{none} \\
\texttt{absorption}  & \textit{string}  &   Absorption model: \texttt{standard}, or \texttt{partitioned energy} \\
\hline
\end{tabular}
\end{center}
\end{table}