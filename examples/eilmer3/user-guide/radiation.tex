% radiation.tex
\section{Radiation transport models}
\label{app:rt-models}

A variety of radiation transport models are implemented in Eilmer3:

\begin{itemize}
 \item Optically thin
 \item Discrete transfer
 \item Monte Carlo
 \item Tangent slab
\end{itemize}

The radiation transport model is defined the \texttt{transport\_data} field of the radiation Lua input file.

\subsection{Discrete transfer model}

The discrete transfer model can be selected via:

\noindent \topbar
\begin{lstlisting}[basicstyle=\ttfamily\normalsize]
{
transport\_data = {
   transport\_model = 'discrete transfer',
   nrays = 8,
   clustering = "none",
   binning = "none"
}
\end{lstlisting}[basicstyle=\ttfamily\normalsize]
\bottombar

A description of the Lua input fields for the discrete transfer radiation transport model is given in Table~\ref {tab:DTM-input}.

\begin{table}
 \begin{center}
  \caption{Description of Lua input fields for the discrete transfer radiation transport models}
  \label{tab:DTM-input}
\begin{tabular}{llp{6cm}}
 \hline \hline
 Field                        & Type              &  Description \\ \hline
\texttt{nrays}            & \textit{int}       &  Number of rays emitted per cell and per frequency \\
\texttt{clustering}    & \textit{string}  &  Ray clustering: \texttt{by volume}, \texttt{by area} or \texttt{none} \\
\texttt{binning}        & \textit{string}  &  Binning model: \texttt{opacity}, \texttt{frequency} or \texttt{none} \\
\texttt{N\_bins}         & \textit{int}       &  Number of bins \\
\hline
\end{tabular}
\end{center}
\end{table}