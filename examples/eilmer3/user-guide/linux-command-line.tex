% linux-command-line.tex
\section{Surviving the Linux Command Line}
\label{linux-command-notes-sec}
For running jobs on a Linux machine, it is worth knowing how to get around and do things in the \textit{shell},
which is a command interpreter and programming language.
Sobell's text \cite{sobell_2005a} is a good source of information but here are a few notes to get you started.

\medskip
A basic command is composed of a sequence of words, separated by spaces and has the usual form\\
\texttt{cmd [options] arguments}\\
where 
\begin{itemize}
 \item\texttt{cmd} is the name of the command or utility program that will do the work.
   Command names on Linux are often terse, 2 or 3 character names.
 \item\texttt{options} are words that are optionally included and are typically preceded by one or two dashes.
   These modify the behaviour of the command, if the default behaviour is not quite what you want.
 \item \texttt{arguments} are the things to work on.
   If these are file names, you can often use patterns with \textit{wildcard} characters that may match
   more then one file at a time.
\end{itemize}
Commands often put their \textit{standard output} to the console.
If the amount of text output is overwhelming, it can be \textit{redirected} to a file 
or \textit{piped} through a paging filter.
This latter option is an example of putting multiple command together so that the output from
one command becomes the input for another.
Once you understand the system, customised commands can be build rather simply in this way.

The following tables summarize a number of commands that you are likely to find useful while
using \texttt{Eilmer3}.\\

\subsection*{Logging in and getting out}
\begin{tabular}{l|l}
 \texttt{ssh} \textit{user}\texttt{@}\textit{host} & Connect to computer named \textit{host} as \textit{user}. \\
 \textit{Ctrl+d} & Quit current session. \\
 \texttt{exit} & Quit current session. \\
\end{tabular}

\subsection*{Getting help}
\begin{tabular}{l|l}
 \texttt{man} \textit{cmd-name} & Display the manual page for the named command. \\
 \texttt{man} \textit{cmd-name} \texttt{| less} & Display the manual page through the paging filter. \\
 \texttt{ls --help | less} & Look at the online help provided by the \texttt{ls} command. \\
 \texttt{man -k} \textit{keyword} & List \texttt{man} pages that contain \textit{keyword}. \\
 \texttt{apropos} \textit{subject} & List \texttt{man} pages on \texttt{subject}. \\
\end{tabular}
 
\subsection*{Moving about and looking in your folders}
\begin{tabular}{l|l}
 \texttt{cd} \textit{dir} & Change to directory \textit{dir}. \\
 \texttt{cd} & Change to home directory. \\
 \texttt{cd ..} & Change to parent of current directory. \\
 \texttt{pwd} & Print current (working) directory. \\
 \texttt{pushd} \textit{dir} & \parbox{0.8\textwidth}{Change to new directory \textit{dir}, 
   putting the current directory onto a stack.} \\
 \texttt{popd} & Go back to the directory at the top of that stack. \\
 \texttt{ls -l} & List the files in the current directory, long format. \\
 \texttt{ls -a ..} & List the files in the directory above, including all hidden files. \\
 \texttt{du -h} \textit{dir} & Report the size of the directory and its subdirectories. \\
 \texttt{df -h} & Report the capacities of the file systems and how much is used for each. \\
 \texttt{mkdir} \textit{dir} & Make new directory. \\
 \texttt{rmdir} \textit{dir} & Remove an empty directory. \\
\end{tabular}

\subsection*{Handling files}
\begin{tabular}{l|l}
 \texttt{cat} \textit{file} & Displays the content of a text file. \\
 \texttt{head -n 20} \textit{file-to-show} & Display the first 20 lines of a text file. \\
 \texttt{tail -f} \textit{file-to-show} & \parbox{0.6\textwidth}{Show the last few lines of a file and continue 
   to show lines as that file changes.} \\
 \texttt{grep 'ideal gas' *.py} & \parbox{0.6\textwidth}{Find the string \texttt{ideal gas} in all of 
   the Python files in the current directory.} \\
 \texttt{mv} \textit{src-file} \textit{dest-file} & Renames the source file to the destination name. \\
 \texttt{cp} \textit{src-file} \textit{dest-file} & Copy the content from the source file to the destination file. \\
 \texttt{scp} \textit{src-file} \textit{user}\texttt{@}\textit{host}\texttt{:} & \parbox{0.6\textwidth}{Copy the file
   from the local computer to the home directory of \textit{user} on the remote computer \textit{host}.} \\
 \texttt{rm -r} \textit{dir} & Remove a directory and all of its contents (recursively). \\
 \texttt{gzip} \textit{src-file} & Compresses the file, adding the extension \texttt{.gz} to its name. \\
 \texttt{tar -zcf} \textit{tarfile} \textit{dir} & Pack all of the contents of \textit{dir} into the \textit{tarfile}. \\
 \texttt{tar -zxf} \textit{tarfile} & Unpack the contents of \textit{tarfile} into the current directory. \\
\end{tabular}

\subsection*{Managing processes}
\begin{tabular}{l|l}
 \texttt{top} & \parbox{0.9\textwidth}{Display information about all running processes. This is very handy
   for finding out which jobs are taking all of your workstation's CPU cycles and memory.} \\
 \textit{Ctrl+z} & Stops the current command. \\
 \texttt{bg} & Resumes a stopped job in the background. \\
 \texttt{fg} & Brings most recent job to the foreground. \\
 \textit{Ctrl+c} & Halts current command. \\
\end{tabular}

\subsection*{Command-line editing}
On most Linux systems, it seems that you can use the cursor keys to move about within the command line.
Delete and backspace also seem to have suitable effect.

\smallskip \noindent
\begin{tabular}{l|l}
 \textit{Ctrl+u} & Erases whole command line. \\
 \texttt{!!} & Repeats last command. \\
 \texttt{history} & Shows command history. \\
 \texttt{!}\textit{n} & Repeats command \textit{n}. \\
\end{tabular}
