As explained in Section~\ref{thermo-flow-sec}, most users
can select a gas model by a call to \texttt{select\_gas\_model}
using the \texttt{model} and \texttt{species} keyword arguments.
For more advanced uses of the gas models, a configuration file
created directly by the user is required.
This section discusses the creation of that file.

The configuration file has a Lua-style format, meaning
that the statements and expressions in the file must conform
to proper Lua syntax.
Do not be concerned with the need to learn Lua in order to
build a configuration file: nearly all of the statements
you will require can be taken from the following examples.
Besides, Lua has been designed from the outset as an embedded
language for configuration purposes, and, with that aim in mind,
it has a simple syntax suitable for non-programmers~\cite{ierusalimschy_etal_1996}.
The following subsections explain the requirements of configuration
files for specific gas models.

\subsection{User-defined gas model}\index{gas model!user-defined}
The user may define her own gas model by providing
callable functions that implement the desired behaviour.
There is a minimal (read mandatory) set of functions
that the user must provide in the configuration file.
There are also some optional functions.
When the optional functions are not provided, the
underlying C++ code will provide a default implementation.
For example, if the user does not provide a function \texttt{dT\_dp\_const\_rho}
then the default implementation will use a numerical differentiation
technique to compute this value when required.
In addition to providing some mandatory functions, the user's
configuration file needs to set three global variables:
\begin{itemize}
 \item \texttt{model}: set as \texttt{'user-defined'}
 \item \texttt{nsp}: the number of species in the gas model
 \item \texttt{nmodes}: the number of thermal modes in the gas model
\end{itemize}

Each of the functions which the user specifies has certain rules
that they must conform to: they must accept a distinct set of arguments
in the correct order; and they must return the expected number of results
of the correct type and in correct order.\footnote{This statement about received function
arguments is not strictly true.  If the user is familiar with how Lua treats missing
and or extra arguments, then (s)he will be aware that the implementation may still
function even if not all arguments are present.  In practice and for ease of understanding
the code, it is best to stick to the documented function signatures.}
The job of the function will be to compute the required results
based on the input arguments.
The set of functions recognised by a `user-defined' gas model, 
along with their arguments lists and return value lists, are given
in Table~\ref{tab:user-defined-gas-model}.
The mandatory functions are listed first, followed by the optional functions.
A majority of the functions accept a \texttt{gas\_data} table as an argument
and also return a \texttt{gas\_data} table.
As a pre-condition to the function calls, certain data members in the
\texttt{gas\_data} table may be assumed to be present and correct.
As a post-condition to the function calls, certain data members in the
\texttt{gas\_data} table should be correctly set upon return.
These pre- and post-conditions for the \texttt{gas\_data} table
are also shown in Table~\ref{tab:user-defined-gas-model}.
Finally, Table~\ref{tab:gas-data} describes the entry
in a \texttt{gas\_data} table.

\begin{landscape}
\begin{center}
\setlength{\LTcapwidth}{15cm}
\begin{longtable}{p{5.5cm}p{4.5cm}p{4.5cm}p{6.5cm}}
\caption{User-defined functions for specification of gas model behaviour \label{tab:user-defined-gas-model}}\\
\hline \hline Function & Arguments & Return values & Description \\ \hline
\endfirsthead
\hline \hline Function & Arguments & Return values & Description \\ \hline
\endhead
\hline
\endfoot
\multicolumn{4}{c}{\emph{Mandatory functions}} \\
\texttt{eval\_thermo\_state\_rhoe} & \raggedright{\textbf{Q}: \texttt{gas\_data} \\
                                                  \textit{pre-conditions}: \texttt{rho}, \texttt{e} and \texttt{massf} are set}
                                   & \raggedright{\textbf{Q}: \texttt{gas\_data} \\
                                                  \textit{post-conditions}: set all other thermodynamic variables} 
                                   &   Given density, the internal energy (vector) and mass fractions,
                                       compute the rest of the thermodynamic state. \\
\texttt{eval\_transport\_properties} & \raggedright{\textbf{Q}: \texttt{gas\_data} \\
                                                   \textit{pre-conditions}: thermodynamic state is up-to-date} 
                                    & \raggedright{\textbf{Q}: \texttt{gas\_data} \\
                                                   \textit{post-conditions}: transport data set, i.e., \texttt{mu} and
                                                   \texttt{k}}
                                    & Given an up-to-date thermodynamic state, compute the transport properties:
                                      viscosity and thermal conductivity (in all thermal modes as appropriate). \\
\texttt{eval\_diffusion\_coefficients} & \raggedright{\textbf{Q}: \texttt{gas\_data} \\
                                                      \textit{pre-conditions}: thermodynamic state is up-to-date} 
                                       & \raggedright{\textbf{Q}: \texttt{gas\_data} \\
                                                      \textit{post-conditions}: all binary diffusion coefficients set, \texttt{D\_AB}}
                                       & Given an up-to-date thermodynamic state, compute the binary diffusion coefficients for
                                         all species interaction pairs in the mixture. \\
\texttt{molecular\_weight}             & \raggedright{\textbf{isp}: species index, \texttt{integer}} 
                                       & \raggedright{\textbf{MW}: molecular weight of species \textit{isp}, \texttt{float}}
                                       & Returns the molecular weight, g/mol, for species \textit{isp}. \\ \hline
\multicolumn{4}{c}{\emph{Optional functions}} \\
\texttt{eval\_thermo\_state\_pT} & \raggedright{\textbf{Q}: \texttt{gas\_data} \\
                                                \textit{pre-conditions}: \texttt{p}, \texttt{T} and \texttt{massf} are set}
                                 & \raggedright{\textbf{Q}: \texttt{gas\_data} \\
                                                  \textit{post-conditions}: set all other thermodynamic variables} 
                                   &   Given pressure, temperature(s), and mass fractions,
                                       compute the rest of the thermodynamic state. \\
\texttt{eval\_thermo\_state\_rhoT} & \raggedright{\textbf{Q}: \texttt{gas\_data} \\
                                                \textit{pre-conditions}: \texttt{rho}, \texttt{T} and \texttt{massf} are set}
                                 & \raggedright{\textbf{Q}: \texttt{gas\_data} \\
                                                  \textit{post-conditions}: set all other thermodynamic variables} 
                                   &   Given density, temperature(s), and mass fractions,
                                       compute the rest of the thermodynamic state. \\
\texttt{eval\_thermo\_state\_rhop} & \raggedright{\textbf{Q}: \texttt{gas\_data} \\
                                                \textit{pre-conditions}: \texttt{rho}, \texttt{p} and \texttt{massf} are set}
                                 & \raggedright{\textbf{Q}: \texttt{gas\_data} \\
                                                  \textit{post-conditions}: set all other thermodynamic variables} 
                                   &   Given density, pressure, and mass fractions,
                                       compute the rest of the thermodynamic state. \\
\texttt{encode\_conserved\_energy} & \raggedright{\textbf{Q}: \texttt{gas\_data} \\
                                                  \textit{pre-conditions}: \texttt{rho}, \texttt{e} and \texttt{massf} are set} 
                                   & \raggedright{\textbf{rhoe}: table of \texttt{float} values \\
                                                  \textit{post-conditions}: table has dimension \texttt{nmodes} and stores
                                                  the conserved energy quantities} 
                                   & Given the primary variables, encode the conserved energy quantities. \\
\texttt{decode\_conserved\_energy} & \raggedright{\textbf{Q}: \texttt{gas\_data} \\
                                                  \textbf{rhoe}: table of \texttt{float} values
                                                  \textit{pre-conditions}: \texttt{rho} and \texttt{massf} are set} 
                                   & \raggedright{\textbf{Q}: \texttt{gas\_data} \\
                                                  \textit{post-conditions}: the vector member \texttt{e} is up-to-date (decoded)} 
                                   & Given density and the vector of conserved energies, set the specific energies. \\
\texttt{update\_massf\_mode}       & \raggedright{\textbf{Q}: \texttt{gas\_data} \\
                                                  \textit{pre-condition}: \texttt{massf} are set}
                                   & \raggedright{\textbf{Q}: \texttt{gas\_data} \\
                                                  \textit{post-condition}: \texttt{massf\_mode} are set}
                                   & Given the mass fractions, update the vector \texttt{massf\_mode} which
                                     stores the mass fraction associated with each thermal mode. \\
\texttt{dTdp\_const\_rho}          & \raggedright{\textbf{Q}: \texttt{gas\_data} \\
                                                  \textit{pre-condition}: thermodynamic state variables are up-to-date}
                                   & \raggedright{\textbf{dTdp}: \texttt{float}}
                                   & Return $\left(\frac{\partial T}{\partial p}\right)_{\rho}$. \\
\texttt{dTdrho\_const\_p}          & \raggedright{\textbf{Q}: \texttt{gas\_data} \\
                                                  \textit{pre-condition}: thermodynamic state variables are up-to-date}
                                   & \raggedright{\textbf{dTdrho}: \texttt{float}}
                                   & Return $\left(\frac{\partial T}{\partial \rho}\right)_{p}$. \\
\texttt{dpdrho\_const\_T}          & \raggedright{\textbf{Q}: \texttt{gas\_data} \\
                                                  \textit{pre-condition}: thermodynamic state variables are up-to-date}
                                   & \raggedright{\textbf{dpdrho}: \texttt{float}}
                                   & Return $\left(\frac{\partial p}{\partial \rho}\right)_{T}$. \\
\texttt{dedT\_const\_v}            & \raggedright{\textbf{Q}: \texttt{gas\_data} \\
                                                  \textit{pre-condition}: thermodynamic state variables are up-to-date}
                                   & \raggedright{\textbf{dedT}: \texttt{float}}
                                   & Return $\left(\frac{\partial e}{\partial T}\right)_{v}$: $C_v$, the specific
                                     heat capacity at constant volume. \\
\texttt{dhdT\_const\_p}            & \raggedright{\textbf{Q}: \texttt{gas\_data} \\
                                                  \textit{pre-condition}: thermodynamic state variables are up-to-date}
                                   & \raggedright{\textbf{dhdT}: \texttt{float}}
                                   & Return $\left(\frac{\partial h}{\partial T}\right)_{p}$: $C_p$, the specific
                                     heat capacity at constant volume. \\
\texttt{gas\_constant}             & \raggedright{\textbf{Q}: \texttt{gas\_data} \\
                                                  \textit{pre-condition}: thermodynamic state variables are up-to-date}
                                   & \raggedright{\textbf{R}: \texttt{float}}
                                   & Return $R$, the specific gas constant. \\
\end{longtable}
\end{center}
\end{landscape}

\begin{table}
 \begin{center}
  \caption{Description of fields in \texttt{gas\_data} table}
  \label{tab:gas-data}
\begin{tabular}{llp{6cm}}
 \hline \hline
 Field        & Type            &  Description \\ \hline
\multicolumn{3}{c}{\emph{Thermodynamic properties}} \\
\texttt{rho}  & \textit{float}  &  density, kg/m$^3$ \\
\texttt{p}    & \textit{float}  &  pressure, Pa \\
\texttt{a}    & \textit{float}  &  sound speed, m/s \\
\texttt{e}    & vector of \textit{floats} & specific internal energies, J/kg \\
\texttt{T}    & vector of \textit{floats} & temperatures, K \\
              &                           &                 \\
\multicolumn{3}{c}{\emph{Transport properties}} \\
\texttt{mu}   & \textit{float}  &  kinematic viscosity, Pa.s \\
\texttt{k}    & vector of \textit{floats} & thermal conductivities (for each mode), W/(m.K) \\
\texttt{D\_AB} & matrix of \textit{floats} & binary diffusion coefficients, m$^2$/s \\
              &                 &             \\
\multicolumn{3}{c}{\emph{Composition}} \\
\texttt{massf} & vector of \textit{floats} & species mass fractions \\
\texttt{massf\_mode} & vector of \textit{floats} & mass fraction associated with specific thermal modes \\ \hline
\end{tabular}
\end{center}
\end{table}

\subsubsection{An example minimal user-defined gas model}\index{gas model!user-defined!minimal example}
The following code listing shows the minimum requirements to specify
a user-defined gas model.
This is a concrete example to complement the abstract discussion presented
previously.
This particular example implements ideal air.
This is a trivial example for the sake of demonstration, and one would not use the slow user-defined gas model
for ideal air when an internally implemented model already exists.
The intended use for the user-defined gas models is for more exotic gases or rapid prototyping
of a new gas model.

\noindent
\topbar
\lstinputlisting[language={}]{../../../lib/gas/test/ideal-air-minimal-udm.lua}
\bottombar

\subsection{Equilibrium gas based on a look-up table}
\index{gas model!look-up table}
\index{gas model!equilibrium chemistry|see{look-up table}}
The properties of a gas mixture in thermochemical equilibrium can be computed using the 
CEA program~\cite{gordon_mcbride_1994, mcbride_gordon_1996}.
By pre-computing the properties for a range of densities and internal energies,
a look-up table can be created.
The use of a look-up table is much more efficient to use than calling out to the CEA program during 
simulation execution; there is some small sacrifice in accuracy using the look-up table.

\subsubsection{Selecting a look-up table for the gas model}
A number of pre-built look-up tables are provided as par the code collection.
After installing Elmer3, the pre-built look-up tables are provided in \texttt{\$HOME/e3bin/cea-cases/}.
A list of these tables and a description of what gas mixture they model is given in 
Table~\ref{tab:lut}.

\begin{table}[ht]
\begin{center}
\caption{Description of pre-built look-up tables distributed with Elmer3}
\label{tab:lut}
\begin{tabular}{lp{8cm}}
 \hline \hline
Pre-built table & Description \\
\hline
\texttt{cea-lut-air.lua.gz} & Equilibrium air with ionisation. Useful for Earth reentry problems. \\
\texttt{cea-lua-CO2.lua.gz} & Pure carbon dioxide in equilibrium with ionisation. \\
\texttt{cea-lua-Jupiter-like.lua.gz} & A H$_2$-Ne mixture used to simulate the Jovian H$_2$-He atmosphere
                                       in expansion tube work. \\
\texttt{cea-lut-Kr.lua.gz} & Pure Kryton with ionisation allowed. \\
\texttt{cea-lut-Mars-basic.lua.gz} & A basic Martian atmosphere, without trace species.  The included species
                                     are CO$_2$ (97\% by weight) and N$_2$ (3\% by weight).  No ionisation is considered. \\
\texttt{cea-lut-Mars-trace.lua.gz} & A Martian atmosphere which includes the trace species O$_2$ and Ar. No ionisation is considered. \\
\texttt{cea-lut-Mars-trace-with-ions.lua.gz} & A Martian atmosphere incluing trace species and ionisation. \\
\texttt{cea-lut-N2.lua.gz} & Pure nitrogen in equilibrium with ionisation. \\
\texttt{cea-lut-Titan-like-no-ions.lua.gz} & A Titan-like atmosphere (N$_2$ and CH$_4$, no trace species). No ionisation is considered. \\
\texttt{cea-lut-Titan-like-with-ions.lua.gz} & A Titan-like atmosphere which includes ionisation. \\
\hline
\end{tabular}
\end{center}
\end{table}

The steps to using a look-up table in your simulation are:
\begin{enumerate}
 \item Copy a pre-built table to your working directory.
 \item Specify the name of thie pre-built table in your call to \texttt{select\_gas\_model} in your
       simulation setup script.
\end{enumerate}
As an example, suppose that we wish to run a simulation with CO$_2$ in equilibrium.
Then as per above the sequence of steps is (assuming your are in your working directory):
\begin{enumerate}
 \item \texttt{cp \$HOME/e3bin/cea-cases/cea-lut-CO2.lua.gz .}
 \item Add the following function call to your script:\\
       \texttt{select\_gas\_model(fname='cea-lua-CO2.lua.gz')}
\end{enumerate}

\subsubsection{Building your own look-up table}
Of course, you might have a gas mixture you wish to simulate that is not listed in Table~\ref{tab:lut}.
The tool \texttt{build-cea-lut} is provided as part of the code collection to aid in building a 
look-up table of the appropriate format.
You will need access to the \texttt{cea2} program~\cite{gordon_mcbride_1994, mcbride_gordon_1996}, and have that setup in your working area
\footnote{By setup, I mean
that the \texttt{thermo.inp} and \texttt{trans.inp} files have been processed and the corresponding \texttt{.lib} files
are available in the working directory.
Also, the program \texttt{cea2} needs to be available as an executable in your \texttt{\$PATH}.}.

The first step is to build a ``template'' file.
This file specifies the composition of the gas model and some other options appropriate to the modelling.
Here is an example template file to build a look-up table for a Mars-like atmosphere with no ionisation but
includes trace species. 

\noindent
\topbar
\lstinputlisting[language={}]{../../../lib/gas/cea-cases/Mars-trace.lua}
\bottombar

\noindent
The list of expected parameters are given in Table~\ref{tab:lut-template}
\begin{table}[ht]
\begin{center}
\caption{Description of parameters required to build a template file for \texttt{build-cea-lut}}
\label{tab:lut-template}
\begin{tabular}{llp{8cm}}
\hline \hline
Parameter & Type & Description \\ \hline
\texttt{fname} & \textit{string} & A name for the look-up table.  Usually just specifed as
                                   \texttt{some\_name.lua}.  The program will add a trailing
                                   \texttt{.gz} when it zips up the look-up table. \\
\texttt{units} & \textit{string} & This denotes the units for specifying the relative amount of reactants.
                                   The choices are:
                                   \begin{itemize}
                                      \item \texttt{'moles'} \textit{(default if not specified)} to specify by molar composition
                                      \item \texttt{'massf'} to specify by mass (weight) composition 
                                   \end{itemize} \\
\texttt{reactants} & \textit{table} & This table lists the reactants in the mixture.  The keys are the species names as
                                      recognised by CEA and the values are the composition. \\
\texttt{with\_ions} & \textit{boolean} & If true, allow ionisation (by specifying \texttt{'ions'} in the CEA input files.
                                         If false, ionisation is not allowed \textit{(default)}. \\
\texttt{only\_list} & \textit{string} & This string restricts the allowable species in gas mixture computation.  The format for
                                        the string is a space separated list of species.  This item is optional: if omitted, 
                                        the CEA program is free to choose which species are considered according to its
                                        own internal algorithm. \\
\hline
\end{tabular}
\end{center}
\end{table}

For more examples, the template files for all of the pre-built tables listed in Table~\ref{tab:lut} are available in
\texttt{\$HOME/e3bin/cea-cases/}.

Once you have created a template file, the \texttt{build-cea-lut} tool can be called.
Supposing your template file is called \texttt{Venus-atmosphere.lua}, then issue
the following command in your working directory:\\
\topbar\\
\texttt{> build-cea-lut --user-defined=Venus-atmosphere.lua} \\
\bottombar\\
You may use the \texttt{--extrapolate} flag to fill in the edges
of the look-up table by linear interpolation where no data is available, however, this option
is not presently recommended.
If this flag is omitted, a smaller table is built that does not extrapolate at the edges.
Note that this flag only controls extrapolation during the \textit{construction} of the table; in 
the actual use of the table during \textit{simulation}, the values are always extrapolated
if they exceed the limits of the table.
Upon succesful execution of the \texttt{build-cea-lut}, you will have a compressed (gzipped)
file in your working directory.
This file (not the template file) can be used to select an equilibrium gas in the
same manner as using a pre-built table, as was discussed in the preceding section. 

