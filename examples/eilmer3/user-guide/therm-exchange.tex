% therm_exchange.tex
\section{Thermal energy exchange mechanisms: specification by configuration file}
\label{app:therm-exchange}
\index{thermal nonequilibrium!energy exchange scheme file}
For thermal nonequilibrium flow simulations, the user may wish to model a set of 
energy exchange mechanisms operating between the thermal modes.
In a similar fashion as for chemical reactions (see Appendix~\ref{app:chem}), 
thermal energy exchange mechanisms are described in a Lua input file prepared by
the user.

As a first example, let's look at an input file for a two-temperature
simulation of nitrogen flow (although this is not explicit in the
the energy exchange file: this information appears in the accompanying
gas model file).
The two temperatures are a transrotational temperature (translational and
rotational energy modes are assumed to be equilibriated at a common temperature)
and a vibroelectronic temperature (the vibrational and electronic energy modes
are assumed to equilibriated a common temperature, different from the transrotational
temperature).
The user-created energy exchange input file describes the mechanism and rate
by which these two temperature modes relax (or equilibriate) with one another.
In this example, we just consider a V-T exchange: a mechanism for the vibrational
energy mode to exchange energy with the translational energy mode.
The input file is listed here.\\
\topbar\\
\lstinputlisting[language={}]{../2D/giordano/N2-TV.lua}
\bottombar\\
There is only one mechanism listed here.
What this says is that in the collision between a nitrogen molecule
and another nitrogen molecule, the vibrational energy may be altered and
the change in energy is soaked up in the translational mode.
This energy exchange occurs at a particular rate which is controlled
by a relaxation time which is dependent on the local thermodynamic state.
In this case, the relaxation time is modelled as a curve fit to a
Landau-Teller type relaxation.
The parameters $A$, $B$ and $C$ control the shape of the curve fit and
have been determined to a give a best fit to experimental measurement
of the relaxation time.
In this example, there is only one energy exchange mechanism.
For certain gas mixtures, there may be several mechanisms of energy exchange
amongst the various energy modes.
Each of these mechanisms is listed in separate \texttt{mechanism} tables.


\subsection{Overview of the input file format}
The Lua programming language is used for the input data description.
Any legal Lua code may appear in the energy exchange file.
However, the user should not rename the following special
pre-defined functions:
\begin{enumerate}
\item \texttt{mechanism}
\item \texttt{scheme}
\item \texttt{ode\_solver}
\end{enumerate}

There are supplied default values for the selection of a \texttt{scheme}
(how the energy exchange relaxation is computed) and the \texttt{ode\_solver}
(if used).
These defaults should be adequate for the vast majority of cases.
The bulk of the work for the user is usually specifying a set of
appropriate \texttt{mechanism} entries.
The format for a \texttt{mechanism} entry is discussed next.

\subsection{Details of the \texttt{mechanism} table}
The \texttt{mechanism} table consists of two mandatory fields, and an optional
\texttt{list} field used in certain circumstances.
The first mandatory field is unnamed and always appears first.
It is a string describing the particular energy exchange mechanism.
The second mandatory entry is a named field \texttt{rt} which stands for
`relaxation time'.
This field is used to select the model for how the relaxation time
of the particular energy exchange mechanism is computed.
Thus, the minimal format of the mechanism table is:
\begin{verbatim}
mechanism{
  'mechanism string',
  rt={...}
}
\end{verbatim}
If the \texttt{'mechanism string'} makes use of the symbol \texttt{(*list)},
then a \texttt{list} entry should also appear in the \texttt{mechanism} table.

The \texttt{'mechanism string'} is used to list which species and which
energy modes are involved in a particular energy exchange mechanism.
This string must conform to a strict syntax in order to be a valid
description of an energy exchange system.\footnote{For those interested
in computer programming, the syntactical parsing of the 
mechanism string is an example of an embedded domain specific language.}
The general form of the mechanism string is:
\begin{verbatim}
'A ~~ <colliders> : A_mode-C_mode'
\end{verbatim}
The first part of the string (before the colon), declares which main species
is having one of its energy modes changed due to collisions with certain
other species.
So in this declaration, our attention is focussed on how a particular
energy mode of species A is altered due to collisions with other
particles.
The second part of the string (after the colon) tells us which energy modes are
affected.
There should always be two modes affected: the first corresponds to a mode
of species A and the second to a mode of the colliding species.
The details and allowable values for the generic fields in the mechanism string are:
\begin{description}
\item[\texttt{A}] is the name of a single species. This is the main species of interest.
                  We are going to consider how collisions of other particles with this
                  species affect the energy in one of its energy modes.
\item[\texttt{<colliders>}] is the list of colliding species which will affect the
                  energy content of the main species \texttt{A}. There are four possible
                  values allowed.
  \begin{enumerate}
  \item a single species name, e.g. \texttt{'O2'}
  \item a bracketed list of species, e.g. \texttt{('O2', 'N2')}
  \item the special keyword `all' to denote collisions with all species in the mix, e.g. \texttt{(*all)}
  \item the special keyword `list' to denote collisions with a specific list of species, e.g. 
        \texttt{(*list)}. If this value is used, a \texttt{list} field should appear in
        the mechanism table. Basically, this is used to instruct the parser to look in the
        mechanism table for a list of colliding species.
  \end{enumerate}
\item[\texttt{A\_mode}] kklk
\item[\texttt{C\_mode}] kl
\end{description}




Creating the Lua input file for thermal energy exchange, however, is presently less user-friendly
than for chemical reactions.
Users are advised to use the input file \verb TV-TE_exchange.lua  from the thermal nonequilibrium
3D cylinder simulation described in Section~\ref{finite-cyl-therm-noneq} as a template when
creating their own energy exchange schemes.
The input file must have the following general structure:

\begin{verbatim}
scheme_t = {...}

ode_t = {...}

rates = {...}

equilibriation_mechanisms = {...}
\end{verbatim}

The \texttt{scheme\_t} table defines the scheme that will be used to model the energy exchange
update during a timestep.
The table should have the following format:

\begin{verbatim}
scheme_t = {
    update ='energy exchange ODE',
    temperature_limits = {
        lower = 20.0,
        upper = 100000.0
    },
    error_tolerance = 0.000001
}
\end{verbatim}

\begin{description}
 \item[\texttt{update}] \hspace{1cm} \\
 A string defining the update method.  Presently the only available option is
 \texttt{energy exchange ODE}, where the energy exchange update is modelled via solving a
 system of ordinary differential equations.
 \item[\texttt{temperature\_limits}] \hspace{1cm} \\
 Specifies the range of \textit{translational} temperatures where thermal energy exchange is 
 permitted to occur.  The fields \texttt{lower} and \texttt{upper} expect floating point values.
 \item[\texttt{error\_tolerance}] \hspace{1cm} \\
 Although not currently used in the code, a floating point value is expected in this field.
\end{description}

\par

The \texttt{ode\_t} table defines parameters for controlling the ODE solver used during the
energy exchange update.
Note this has the same format as the \texttt{ode\_solver} table in the chemistry input
file described in Appendix~\ref{app:chem}.
The table should have the following format:

\begin{verbatim}
ode_t = {
    step_routine = 'rkf',
    max_step_attempts = 4,
    max_increase_factor = 1.15,
    max_decrease_factor = 0.01,
    decrease_factor = 0.333
}
\end{verbatim}

\begin{description}
 \item[\texttt{step\_routine}] \hspace{1cm} \\
 A string specifying the desired ODE stepping method.
 The available methods are:
   \begin{tabular}{lll}
    \texttt{'qss'} & : & Mott's $\alpha$-QSS method~\cite{mott_99a} \\
    \texttt{'rkf'} & : & Runge-Kutta-Fehlberg method~\cite{fehlberg_69a} \\
    \texttt{'euler'} & : & Euler stepping \\
   \end{tabular}
\item[\texttt{max\_step\_attempts}] \hspace{1cm} \\
    This integer value sets the maximum number of retry attempts the stepping
    routine will attempt on a single step if the ODE system indicates failure.
\item[\texttt{max\_increase\_factor}] \hspace{1cm} \\
    This value is used to control the maximum factor the thermal timestep
    will increase when the step is successful.  The \texttt{'qss'} and \texttt{'rkf'}
    methods can suggest their own timestep increase. However, the increase will
    be calculated as \texttt{MIN(suggestion, max\_increase\_factor)}.
\item[\texttt{max\_decrease\_factor}] \hspace{1cm} \\
    This value is used to control the maximum amount of decrease or reduction
    in timestep that occurs.  It is computed as \texttt{MAX(suggestion, max\_decrease\_factor)}.
\item[\texttt{decrease\_factor}] \hspace{1cm} \\
    Occasionally, the step fails and yet the step routines suggests using a \emph{larger}
    timestep for the retry.  In this case, the \texttt{decrease\_factor} is used to reduce
    the timestep size for the retry attempt.
\end{description}

The \texttt{rates} table lists the thermal energy exchange mechanisms to be considered for 
each thermal mode \textit{except the primary mode}\footnote{The energy of the primary thermal
mode is solved for by enforcing the conservation of total energy during the thermal time-step.}.
Therefore one entry is expected for a two temperature model, two entries for a three 
temperature model, etc.
For a three temperature model, for example, where the list of thermal modes in the
\texttt{gas-model.lua} file reads:

\begin{verbatim}
thermal_modes = { 'transrotational', 'vibrational', 'electronic' }
\end{verbatim}

\noindent the table should have the following format:

\begin{verbatim}
rates = {
    {
        -- vibrational mode
        mechanisms = {...}
    },
    {
        -- electronic mode
        mechanisms = {...}
    }
}
\end{verbatim}

\noindent where the first table entry is for the vibrational thermal mode, whilst the second 
table entry is for the electronic thermal mode.
The \texttt{mechanisms} tables list the thermal energy exchange mechanisms to be applied to
the respective thermal modes.
The mandatory items for a \texttt{mechanisms} table entry are:

\begin{description}
 \item[\texttt{type}] \hspace{1cm} \\
 A string specifying the type of energy exchange mechanism.
 The available types are: \\
   \begin{tabular}{lll}
    \texttt{'VT\_exchange'} & : & Vibration-translation exchange \\
    \texttt{'ET\_exchange'} & : & Electron-translation exchange \\
    % \texttt{'VE\_exchange'} & : & Vibration-electron exchange \\
    % \texttt{'EV\_exchange'} & : & Electron-vibration exchange \\
   \end{tabular}
\item[\texttt{relaxation\_time}] \hspace{1cm} \\
    A table listing the parameters for the relaxation time model.
\end{description}

\par

When specifying a \texttt{'VT\_exchange'} mechanism, an additional field \texttt{'p\_name'}
that indicates the name of the vibrating species is required.
A detailed description of the \texttt{relaxation\_time} table will be available in a 
future version of this user guide.
For the moment, please refer to the following example as a basic guide.

\par

Below is the thermal energy exchange Lua input file for dissociating and ionising nitrogen
described by the two temperature model (see Section~\ref{finite-cyl-therm-noneq} for
an example simulation using this model).
The gas consists of five species, namely N$_2$, N$_2^+$, N, N$^+$ and e$^+$, and two thermal
modes, translation-rotation and vibration-electron-electronic.
Two thermal energy exchange mechanisms are specified: vibration-translation exchange due to
inelastic collisions with the N$_2$ molecule, and electron-translation exchange due to elastic 
collisions between free-electrons and heavy particles.\\
\topbar\\
\lstinputlisting[language={},firstline=21,lastline=54]{../3D/finite-cylinder/thermal-noneq/TV-TE_exchange.lua}
\bottombar\\
