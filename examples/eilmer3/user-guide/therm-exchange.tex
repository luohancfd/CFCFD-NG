% therm_exchange.tex
\index{thermal nonequilibrium!energy exchange scheme file}
For thermal nonequilibrium flow simulations, the user may wish to model a set of 
energy exchange mechanisms operating between the thermal modes.
In a similar fashion as for chemical reactions (see Appendix~\ref{app:chem}), 
thermal energy exchange mechanisms are described in a Lua input file prepared by
the user.

\par

Creating the Lua input file for thermal energy exchange, however, is presently less user-friendly
than for chemical reactions.
Users are advised to use the input file \verb TV-TE_exchange.lua  from the thermal nonequilibrium
3D cylinder simulation described in Section~\ref{finite-cyl-therm-noneq} as a template when
creating their own energy exchange schemes.
The input file must have the following general structure:

\begin{verbatim}
scheme_t = {...}

ode_t = {...}

rates = {...}

equilibriation_mechanisms = {...}
\end{verbatim}

The \texttt{scheme\_t} table defines the scheme that will be used to model the energy exchange
update during a timestep.
The table should have the following format:

\begin{verbatim}
scheme_t = {
    update ='energy exchange ODE',
    temperature_limits = {
        lower = 20.0,
        upper = 100000.0
    },
    error_tolerance = 0.000001
}
\end{verbatim}

\begin{description}
 \item[\texttt{update}] \hspace{1cm} \\
 A string defining the update method.  Presently the only available option is
 \texttt{energy exchange ODE}, where the energy exchange update is modelled via solving a
 system of ordinary differential equations.
 \item[\texttt{temperature\_limits}] \hspace{1cm} \\
 Specifies the range of \textit{translational} temperatures where thermal energy exchange is 
 permitted to occur.  The fields \texttt{lower} and \texttt{upper} expect floating point values.
 \item[\texttt{error\_tolerance}] \hspace{1cm} \\
 Although not currently used in the code, a floating point value is expected in this field.
\end{description}

\par

The \texttt{ode\_t} table defines parameters for controlling the ODE solver used during the
energy exchange update.
Note this has the same format as the \texttt{ode\_solver} table in the chemistry input
file described in Appendix~\ref{app:chem}.
The table should have the following format:

\begin{verbatim}
ode_t = {
    step_routine = 'rkf',
    max_step_attempts = 4,
    max_increase_factor = 1.15,
    max_decrease_factor = 0.01,
    decrease_factor = 0.333
}
\end{verbatim}

\begin{description}
 \item[\texttt{step\_routine}] \hspace{1cm} \\
 A string specifying the desired ODE stepping method.
 The available methods are:
   \begin{tabular}{lll}
    \texttt{'qss'} & : & Mott's $\alpha$-QSS method~\cite{mott_99a} \\
    \texttt{'rkf'} & : & Runge-Kutta-Fehlberg method~\cite{fehlberg_69a} \\
    \texttt{'euler'} & : & Euler stepping \\
   \end{tabular}
\item[\texttt{max\_step\_attempts}] \hspace{1cm} \\
    This integer value sets the maximum number of retry attempts the stepping
    routine will attempt on a single step if the ODE system indicates failure.
\item[\texttt{max\_increase\_factor}] \hspace{1cm} \\
    This value is used to control the maximum factor the thermal timestep
    will increase when the step is successful.  The \texttt{'qss'} and \texttt{'rkf'}
    methods can suggest their own timestep increase. However, the increase will
    be calculated as \texttt{MIN(suggestion, max\_increase\_factor)}.
\item[\texttt{max\_decrease\_factor}] \hspace{1cm} \\
    This value is used to control the maximum amount of decrease or reduction
    in timestep that occurs.  It is computed as \texttt{MAX(suggestion, max\_decrease\_factor)}.
\item[\texttt{decrease\_factor}] \hspace{1cm} \\
    Occasionally, the step fails and yet the step routines suggests using a \emph{larger}
    timestep for the retry.  In this case, the \texttt{decrease\_factor} is used to reduce
    the timestep size for the retry attempt.
\end{description}

The \texttt{rates} table lists the thermal energy exchange mechanisms to be considered for 
each thermal mode \textit{except the primary mode}\footnote{The energy of the primary thermal
mode is solved for by enforcing the conservation of total energy during the thermal time-step.}.
Therefore one entry is expected for a two temperature model, two entries for a three 
temperature model, etc.
For a three temperature model, for example, where the list of thermal modes in the
\texttt{gas-model.lua} file reads:

\begin{verbatim}
thermal_modes = { 'transrotational', 'vibrational', 'electronic' }
\end{verbatim}

\noindent the table should have the following format:

\begin{verbatim}
rates = {
    {
        -- vibrational mode
        mechanisms = {...}
    },
    {
        -- electronic mode
        mechanisms = {...}
    }
}
\end{verbatim}

\noindent where the first table entry is for the vibrational thermal mode, whilst the second 
table entry is for the electronic thermal mode.
The \texttt{mechanisms} tables list the thermal energy exchange mechanisms to be applied to
the respective thermal modes.
The mandatory items for a \texttt{mechanisms} table entry are:

\begin{description}
 \item[\texttt{type}] \hspace{1cm} \\
 A string specifying the type of energy exchange mechanism.
 The available types are: \\
   \begin{tabular}{lll}
    \texttt{'VT\_exchange'} & : & Vibration-translation exchange \\
    \texttt{'ET\_exchange'} & : & Electron-translation exchange \\
    % \texttt{'VE\_exchange'} & : & Vibration-electron exchange \\
    % \texttt{'EV\_exchange'} & : & Electron-vibration exchange \\
   \end{tabular}
\item[\texttt{relaxation\_time}] \hspace{1cm} \\
    A table listing the parameters for the relaxation time model.
\end{description}

\par

When specifying a \texttt{'VT\_exchange'} mechanism, an additional field \texttt{'p\_name'}
that indicates the name of the vibrating species is required.
A detailed description of the \texttt{relaxation\_time} table will be available in a 
future version of this user guide.
For the moment, please refer to the following example as a basic guide.

\par

Below is the thermal energy exchange Lua input file for dissociating and ionising nitrogen
described by the two temperature model (see Section~\ref{finite-cyl-therm-noneq} for
an example simulation using this model).
The gas consists of five species, namely N$_2$, N$_2^+$, N, N$^+$ and e$^+$, and two thermal
modes, translation-rotation and vibration-electron-electronic.
Two thermal energy exchange mechanisms are specified: vibration-translation exchange due to
inelastic collisions with the N$_2$ molecule, and electron-translation exchange due to elastic 
collisions between free-electrons and heavy particles.\\
\topbar\\
\lstinputlisting[language={},firstline=21,lastline=54]{../3D/finite-cylinder/thermal-noneq/TV-TE_exchange.lua}
\bottombar\\