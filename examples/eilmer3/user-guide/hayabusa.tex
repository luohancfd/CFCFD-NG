% hayabusa.tex

\section{Hayabusa re-entry capsule forebody with radiation coupling}
\label{sec:hayabusa}
\index{radiation coupling!example of use}
\index{finite-rate chemistry|see {chemical reaction}}
\index{gas model!two temperature!example of use}
%
The Hayabusa spacecraft was developed by the Japan Aerospace Exploration Agency to return a sample of a near-Earth asteroid to Earth for scientific analysis.
The re-entry of the spacecraft into the Earths atmosphere in June 2010 occurred at an estimated velocity of 12.2\,km/s and was observed by a number of teams via ground and airborne instruments~\cite{butts_2010}.
The high entry velocity for Hayabusa makes it an interesting case for radiation--flowfield coupling.
The peak heating condition was predicted to occur at an altitude of X\,km and a velocity of 11.6\,km/s~\cite{SFA2003}.

\begin{table}[h]
 \small
 \centering
 \caption{Peak heating freestream conditions for the Hayabusa capsule. }
 \label{tab:hayabusa_effective_flight}
 \begin{tabular*}{0.5\textwidth}{ccc}
  \hline Altitude (km)                                                            & X\\
             Density,  $\rho_\infty$ (kg/m$^3$)                      & $1.645 \times 10^{-4}$   \\
             Temperature, $T_\infty$ (K)                                 & 233.25         \\
             Velocity, $u_\infty$ (m/s)                                      & 11600  \\
             N$_2$ mass-fraction, $f_\text{N$_2$}$            & 0.767 \\
             O$_2$ mass-fraction, $f_\text{O$_2$}$            & 0.233 \\
  \hline
 \end{tabular*}
\end{table}

\newpage

\subsection{Input script (.py)}
\topbar
\lstinputlisting[language={}]{../2D/hayabusa/coupled/hayabusa.py}
\bottombar

\subsection{Shell scripts}
\topbar
\lstinputlisting[language={}]{../2D/hayabusa/coupled/prep.sh}
\bottombar

\subsection{Notes}
