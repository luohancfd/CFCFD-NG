% chem.tex
\index{chemical reaction!reaction scheme file}
The chemical reactions which may take place in a reacting flow
simulation are described in a Lua input file.
This input file, prepared by the user, is read directly by the
main simulation code at run time.
There is no pre-processing step for this input file.
As the input file is Lua-based, the user has access to the 
full extent of the Lua scripting language when
preparing her files.
Do not be concerned if you do not know the Lua syntax; the instructions
and examples given here should be ample to get you started building
reaction schemes.\footnote{If you are worried about needing to ``learn Lua'' just
to get started, then don't be.  First, you may just look at this as an input
format for the chemistry, and forget that it has anything to do with Lua altogether.
Second, Lua was designed with non-programmers in mind and so it uses a simple syntax, 
specifically so that those non-programmers could quickly use Lua
as a configuration language.}

Let's proceed by looking at an example input file and discussing
the keywords and syntax.
Listed here is an input file which describes the simple thermal
dissociation of nitrogen.
There are only two participating species, N$_2$ and N, and
only two reactions.\\
\topbar\\
\lstinputlisting[language={},firstline=13]{../2D/n90/nitrogen-2sp-2r.lua}
\bottombar\\
The first reaction is the dissociation of N$_2$ by collision with other
N$_2$ molecules.
The forward reaction rate coefficient is computed with a generalised
Arrhenius model, and the parameters for that model are specified.
Similarly the backward reaction rate coefficient is computed using
the Arrhenius expression.

More generally, each reaction is specified within a \texttt{reaction} table.
The table is delimited by the opening and closing braces (\{ \}).
The first entry in the table is always a string.
That string is the chemical equation for the reaction.
The remaining items in the table are denoted by key-value pairs (of the form \texttt{key=val}),  and may appear
in any order.
Each item in the table is separated by a comma.\footnote{Lua also permits the use
of semi-colons instead of commas to delimit table entries.}
This example file contained two \texttt{reaction} tables, hence two reactions
are treated in the reaction scheme.

Some final notes before discussing the input file in further depth.
There is no explicit mention of the participating species in the reaction file.
The participating species are taken from the species that are present in
the gas model file for the same flow simulation.
In other words, if you list species in the reaction scheme that are not
present in the gas model, then you will get an error message.

\subsection{Overview of input file format}
By leveraging Lua as the input data description language,
the input file is almost self-describing, in my opinion.
This provides an excellent record of what modelling was used when
you performed a simulation.
A valid reaction input file will conform to the following rules.
\begin{enumerate}
 \item Any legal Lua code is acceptable, but you must not
       rename the following the pre-defined functions:
       \begin{itemize}
        \item \texttt{reaction}
        \item \texttt{remove\_reactions\_by\_label}
        \item \texttt{remove\_reactions\_by\_number}
        \item \texttt{select\_reactions\_by\_label}
        \item \texttt{select\_reactions\_by\_number}
       \end{itemize}
 \item Reactions are declared in \texttt{reaction} tables.
 \item Comments in the file begin when two dashes (\texttt{--}) are encountered
       and proceed to the end of the line. (This is a repetition of Item 1 in that
       comments are legal Lua code.)
\end{enumerate}

As the the reactions are listed, they are numbered internally beginning from 1.
In some cases, it is convenient to list all reactions in a scheme but then 
only use some of the reactions.
This is quite common if you wish to use a reduced mechanism or if you believe
that one of the species is inert at your flow conditions of interest, and
so would want to remove all reactions involving the transformation of that species.
Two convenience functions are provided so that you do not have to 
hack into your input file to remove the unwanted reactions:
\begin{itemize}
 \item \texttt{remove\_reactions\_by\_label}
 \item \texttt{remove\_reactions\_by\_number}
\end{itemize}
Both functions will take a single item or an array of items.
An array is a special form of Lua table which is bracketed
with braces (\texttt{\{ \}}).
The first function accepts strings which correspond to the labels
of reactions.
The labelling of reactions is explained in the next section.
The second function accepts integers which correspond to the
internal numbering.
The convenience functions must be called \emph{after} the declaration
of the associated reactions.
Typically, the user would place the calls to these functions at the end of his input
file.
Two examples follow.\\
%
\topbar\\
\texttt{remove\_reactions\_by\_label(\{'r3', 'r5'\})}\\
\bottombar\\
This call would remove the reactions labelled \texttt{'r3'} and \texttt{'r5'}\\
from the list of participating reactions.\\
\topbar\\
\texttt{remove\_reactions\_by\_number(13)}\\
\bottombar\\
In this call, the 13th listed reaction is removed from the list (because
we all know that 13 is unlucky, right?)\footnote{Actually, unlike the Americans and their buildings, you don't get rid of 13 that easily.
If you have more than 13 reactions, the higher numbered reactions will shuffle up one spot
so that the numbering remains continuous from 1.  This all happens internally.}

Similarly, there are two complementary convenience functions that allow for the selection
of only certain reactions from the full set:
\begin{itemize}
 \item \texttt{select\_reactions\_by\_label}
 \item \texttt{select\_reactions\_by\_number}
\end{itemize}
They work in reverse to the \texttt{remove} functions: these functions
will only select those reactions listed in their arguments for inclusion in the chemistry scheme.

Note, it is not advisable to mix and match the use of the \texttt{remove} and \texttt{select} functions in the one reaction script.
The behaviour is untested.
Now on to the details of the \texttt{reaction} table.

\subsection{Details of the \texttt{reaction} table}
The \texttt{reaction} table accepts a number of items; some are mandatory, most are not.
The full list of items is shown here, and each item is described below.
\begin{verbatim}
reaction{
   'equation string',
   fr={...},
   br={...},
   ec='model name',
   efficiencies={...},
   label='r1'
}
\end{verbatim}

\begin{description}
  \item[\texttt{'equation string'}] \textit{(mandatory)} \\
    As mentioned earlier, this string must appear first in the table and has no key associated with it.
    This string represents the reaction equation.
    As an example, dissociation of nitrogen may be written as
\begin{verbatim}
'N2 + N2 <=> N + N + N2'
\end{verbatim}
    If the reaction involves a collision with a general third body,
    then this is strictly denoted as species \texttt{'M'}.
    For example, the formation of hydroperoxyl from oxygen and
    monatomic hydrogen requires the presence of a third body.
    This reaction is written as
\begin{verbatim}
'H + O2 + M <=> HO2 + M'.
\end{verbatim}
   The reactants and products are delimited by direction arrows.
   The use of \texttt{<=>} indicates that the reaction proceeds
   in both directions, while \texttt{=>} will mean that the reaction
   proceeds in the forward direction only (no backward
   rate of conversion will be computed).

\item[\texttt{fr}] \textit{(optional, if \texttt{br} supplied)} \\
    The \texttt{fr} key is used to specify the forward
    reaction rate coefficient and expects a table value.
    The format of the table is a string naming the model followed
    by key-value pairs giving the parameters for the model.
    The currently implemented reaction rate coefficient models
    are listed in Table~\ref{tab:rr_coeffs}, along with their
    input format.
\item[\texttt{br}] \textit{(optional, if \texttt{fr} supplied)} \\
   The \texttt{br} key is used to specify the backward reaction
   rate coefficient.  It is used in the same manner as the forward
   rate key (\texttt{fr}).
\item[\texttt{ec}] \textit{(optional)} \\
   The \texttt{ec} key is used to specify the model for computing
   the equilibrium constant.  It accepts a string naming the model.
   Currently, there is only one model implemented, \texttt{'from thermo'},
   which calculates the equilibrium constant based on thermodynamic principles.
   For reversible reactions, if only one of \texttt{fr} or \texttt{br} is specified,
   then the use of the equilibrium constant is assumed and does not need
   to be declared. 
\item[\texttt{efficiencies}] \textit{(optional)} \\
   If declaring a third body reaction, all species in the mixture are assumed
   to react with an efficiency of 1.0.
   The \texttt{efficiencies} key accepts a list of \emph{exceptions} to
   that assumption of a value of 1.0.
   The list contains the key-value pairs of the type \texttt{species=efficiency\_value}.
   For example, to denote that N$_2$ has a 6-fold efficiency value and O$_2$ a value
   of 3.5, the list would be:
\begin{verbatim}
efficiencies={N2=6.0, O2=3.5}
\end{verbatim}
  Remember that all species are assumed to have a value of 1.0 unless otherwise
  noted in the list.
  If you have a species that does \emph{not} participate as a third body, then
  be sure to set its efficiency value to 0.0 (e.g. \texttt{H=0.0}).
\item[\texttt{label}] \textit{(optional)} \\
  The \texttt{label} accepts a string allowing the user to give the reaction
  a name.
  This is useful if one wishes to later remove certain reactions based on
  their labels using the \texttt{remove\_reactions\_by\_label} convenience
  function.
\end{description}

Note that if you specify all three of \texttt{fr}, \texttt{br} and
\texttt{ec}, you have overspecified the modelling of
reaction rate coefficients.
In this case, no error is given.
Instead, the \texttt{ec} model is ignored.

\begin{table}
\begin{center}
 \caption{Reaction rate coefficient models input format}
 \label{tab:rr_coeffs}
\begin{tabular}{|p{5cm}|p{9cm}|}
\hline \hline 
Model & Format \\ \hline
generalised Arrhenius
\[ k = A T^n \exp(-T_a/T) \]
 & \texttt{\{'Arrhenius', A=..., n=..., T\_a=...\}}
                         \begin{itemize}
                          \item \texttt{'Arrhenius'} appears first to name the model
                          \item \texttt{A=...} is the pre-exponential coefficient given 
                                               in `cgs' units (because they are most common
                                               in the chemistry reaction rate literature).
                          \item \texttt{n=...} is the non-dimensional power for T
                          \item \texttt{T\_a=...} is the activation temperature in Kelvin.
                         \end{itemize}
                         Do not get confused by the appearance of a negative sign in the 
                         formula; you are required to input the activation barrier temperature
                         which in the majority of cases is a \emph{positive} value.
                         On occasion, the activation temperature is negative.
                         This will be given in the reaction scheme you are following.
      \\ \hline
\end{tabular}
\end{center}
\end{table}

\subsection{Extra control of the chemistry scheme}
There are a number of details to do with solving the finite-rate chemistry
problem that are set by default for the user.
However, all of these parameters may be controlled by the user by setting values in the input file.

Let's first describe the \texttt{scheme} table.
The user may set values in this table that pertain to the chemistry scheme as a whole.
In the example input snippet below, the lower temperature limit is set to 300\,K
and the upper limit is set to 50\,000\,K.
This value is used to control the temperature limits at which reactions are
``turned off'', that is, a chemistry increment is not computed in cells where the
temperature is lower or higher than these limits, as the case may be.
Note that these values are set as part of a subtable, \texttt{temperature\_limits}.\\
%
\topbar\\
\begin{verbatim}
scheme{
   temperature_limits = {lower = 300.0,
                         upper = 50000.0}
}
\end{verbatim}
\bottombar\\
%
The \texttt{scheme} table is currently only a container for the temperature limits.
In future implementations it is planned to contain other options.
For example, the \texttt{scheme} table will contain options for setting parameters related
to multi-temperature chemistry schemes.

The other table presently offered to the user is the \texttt{ode\_solver} table which,
unsurprisingly, contains parameters that allow the user to select details
about the ODE method used to solve the chemistry system.
Let's look at an example of its use.\\
%
\topbar\\
\begin{verbatim}
ode_solver{
   step_routine = 'qss',
   max_step_attempts = 4,
   max_increase_factor = 1.15,
   max_decrease_factor = 0.01,
   decrease_factor = 0.333
}
\end{verbatim}
\bottombar\\
%
The various parameters in the \texttt{ode\_solver} table are described
in the list below.
As an aside, the values shown in the example above are actually the default
values used when the user does not specify any \texttt{ode\_solver} table.
\begin{description}
\item[\texttt{step\_routine}] \hspace{1cm} \\
   This string sets the ODE stepping method. The available stepping methods are: \\
   \begin{tabular}{lll}
    \texttt{'qss'} & : & Mott's $\alpha$-QSS method~\cite{mott_99a} \\
    \texttt{'rkf'} & : & Runge-Kutta-Fehlberg method~\cite{fehlberg_69a} \\
    \texttt{'euler'} & : & Euler stepping \\
   \end{tabular}
\item[\texttt{max\_step\_attempts}] \hspace{1cm} \\
    This integer value sets the maximum number of retry attempts the stepping
    routine will attempt on a single step if the ODE system indicates failure.
\item[\texttt{max\_increase\_factor}] \hspace{1cm} \\
    This value is used to control the maximum factor the chemistry timestep
    will increase when the step is successful.  The \texttt{'qss'} and \texttt{'rkf'}
    methods can suggest their own timestep increase. However, the increase will
    be calculated as \texttt{MIN(suggestion, max\_increase\_factor)}.
\item[\texttt{max\_decrease\_factor}] \hspace{1cm} \\
    This value is used to control the maximum amount of decrease or reduction
    in timestep that occurs.  It is computed as \texttt{MAX(suggestion, max\_decrease\_factor)}.
\item[\texttt{decrease\_factor}] \hspace{1cm} \\
    Occasionally, the step fails and yet the step routines suggests using a \emph{larger}
    timestep for the retry.  In this case, the \texttt{decrease\_factor} is used to reduce
    the timestep size for the retry attempt.
\end{description}


