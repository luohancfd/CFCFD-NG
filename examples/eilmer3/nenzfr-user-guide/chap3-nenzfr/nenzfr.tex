% nenzfr.tex

\newpage
\section{NENZFr}
\label{chapter-nenzfr}

NENZF-r is comprised of a set of Python scripts and template files that coordinate the running of a space-marched nozzle simulation using Eilmer3. The following sections provide an overview of the implementation of NENZF-r and its capabilites. Though based on Eilmer3, minimal details concerning Eilmer3 are provided in this report. Readers are instead referred to the reports by \cite{} for further information on Eilmer3 and on the space-marching algorithm. Table \ref{tab:code}, given below, summarises the scripts that comprise NENZF-r. A copy of each of these files in it's current form may be found in Appendices \ref{app:code} - \ref{app:reactions}. 

Obviously in order to run NENZF-r, Eilmer3 must be successfully compiled. Users should also ensure that the following files are present in the \textbf{local} working directory:
\begin{enumerate}
\item \textit{nenzfr.py}, \textit{nozzle.input.template} and \textit{nenzfr\_compute\_viscous\_data.py}
\item For equilibrium gas calculations a look-up-table should be present. These may be prepared using \textit{setup\_lut.sh}.
\item For non-equilibrium gas calculations, the appropriate reaction scheme as listed in Table \ref{tab:code} is required. These reaction schemes may be found in \textit{~/cfcfd3/lib/gas/reaction-schemes/}. Note that ions are not considered and so the reaction scheme should be checked and adjusted accordingly.
\item A contour or grid file for the nozzle.
\end{enumerate}
Finally, users should ensure that within \textit{nenzfr.py} the directory paths for the Eilmer3 executables are correct. 

\subsection{Method}
NENZF-r has been designed specifically to estimate the test flow conditions produced in reflected shock tunnels. Referring to Figure \ref{fig:x-t-diagram}, given the intial conditions of the gas in the shock tube (state 1), the shock speed and nozzle supply pressure, the global algorithm used by NENZF-r is:
\begin{enumerate}
\item compute the gas conditions behind the incident shock (state 2);
\item compute the gas conditions after shock reflection (state 5);
\item complete an isentropic relaxation to the measured nozzle supply pressure (state 5s);
\item complete an isentropic relaxation to the nozzle throat (state 6);
\item from the throat, complete a 2D axis-symmetric expansion of the gas to the nozzle exit (state 7).
\end{enumerate}
Steps 1-4 are completed by \textit{estcj.py} assuming an equilibrium gas and using 1D shock relations. Calls to NASA's CEA \cite{Gordon_1994} program are used to determine the thermodynamic properties. Step 5, the nozzle expansion, is completed using Eilmer3 assuming either an equilibrium, non-equilibrium or frozen gas. Viscosity and turbulence may be considered depending on the Eilmer3 simulation control data specified in \textit{nozzle.input.template}. 

The top-level coordinating script, \textit{nenzfr.py}, writes an Eilmer3 input file based on \textit{nozzle.input.template}, calls Eilmer3 and then completes the necessary post-processing to determine the averaged properties at the nozzle exit. The following sections discuss the inputs, output files produced and gas models/reaction schemes currently implemented. %Note that \textit{estcj.py} is called by the Eilmer3 input script and the result used to define the inflow condition. 


\begin{table}[!ht]%
\caption{NENZF-r scripts}
\begin{tabulary}{\linewidth}{|l|L|}
\hline \hline
\multicolumn{2}{|c|}{\textbf{Core Files}}\\ 
\hline
nenzfr.py & Top most coordinating script.\\
nozzle.input.template & Template Eilmer3 input script. This is the core of NENZF-r.\\
\hline \hline
\multicolumn{2}{|c|}{\textbf{Auxilary Files}}\\ 
\hline
nenzfr\_compute\_viscous\_data.py & Script used to calculate the viscous data, including $y^+$, at the wall. \\
nenzfr\_sensitivity\_and\_lut.py & Coordinating script that conducts a sensitivity analysis for NENZF-r and builds a corresponding look-up-table for the nozzle exit properties. (YET TO BE COMPLETED)\\
setup\_lut.sh & A simple shell script that builds the required equilibrium gas look-up-tables.\\
run.sh & Shell script to submit a NENZF-r job on the hpcu cluster \textit{barrine}. \\
estcj.py & Equilibrium Shock Tube Calculation, junior. This Python script completes the reflected shock calculation. Though apart of the cfcfd3/Eilmer3 suite of programs, it has been included here for completeness. \\
\hline \hline
\multicolumn{2}{|c|}{\textbf{Nozzle Profiles}} \\ \hline
contour-t4-m10.data & T4 Mach 10 nozzle profile. \\
contour-t4-m6.data & T4 Mach 6 nozzle profile.\\
contour-t4-m4.data & T4 Mach 4 nozzle profile. \\ \hline \hline
\multicolumn{2}{|c|}{\textbf{Reaction Schemes}}\\ \hline
gupta\_etal\_air\_reactions.lua & Reaction scheme for 5-species air model. \\
nitrogen-5sp-6r.lua & Reaction scheme for nitrogen.\\ \hline
\end{tabulary}
\label{tab:code}
\end{table} 


\subsection{Inputs}
NENZF-r takes a number of required and optional input arguments as detailed below.
\begin{itemize}
\item[] \texttt{--help} Displays help information including listing the available input arguments. No other inputs are relevant. \end{itemize}
\textbf{Necessary inputs:} \begin{itemize}
\item[] \texttt{--T1} Shock tube filling temperature (K) 
\item[] \texttt{--p1} Shock tube filling pressure (Pa)
\item[] \texttt{--Vs} Incident shock speed (m/s)
\item[] \texttt{--pe} (Measured) Nozzle supply pressure (Pa) \end{itemize}
\textbf{Optional inputs:}\begin{itemize}
\item[] \texttt{--job} Base name for solution files. Default: nozzle
\item[] \texttt{--gas} Gas that the shock tube has been filled with. Default: air. Options: air, air5species, n2
\item[] \texttt{--chem} Chemistry model to use in the nozzle expansion. Default: eq. Options: eq, neq, frz, frz2
\item[] \texttt{--cfile} Filename for the contour defining the shape of the nozzle. Default: contour-t4-m4.data
\item[] \texttt{--gfile} Filename for a Pointwise plot3d grid for the nozzle. This option over-rides \textit{--cfile} if both are given.
\item[] \texttt{--area} Exit area for the nozzle. Only used by \textit{estcj} which determines the nozzle exit properties (state 7) via an isentropic expansion. Default: 27.0
\item[] \texttt{--exitfile} Filename for the nozzle exit slice data. Default: nozzle-exit.data
\item[] \texttt{--just-stats} Option allowing the code to skip the detailed calculation and just retrive the exit-flow statistics
\end{itemize}
%Providing that Eilmer3 has been successfully compiled, users may invoke NENZF-r minimally using:
Several example invocations for NENZF-r are as follows.
\begin{itemize}
\item[-] Mimimal. This completes an equilibrium calculation using air for the Mach 4 nozzle. 
\newline
\texttt{   ./nenzfr.py --T1=300 --p1=160.0e3 --Vs=2707 --pe=38.0e6}
%\newline 
\item[-] Nitrogen as the test gas:
\newline
\texttt{   ./nenzfr.py --T1=300 --p1=160.0e3 --Vs=2707 --pe=38.0e6 --gas=n2}
%\newline
\item[-] A non-equilibrium solution for the Mach 10 nozzle using the 5-species air gas model:
\newline
\texttt{   ./nenzfr.py --gas=air5species --T1=300.0 --p1=160.0e3 --Vs=2300.0}
\newline
\texttt{       --pe=43.0e6 --cfile=contour-t4-m10.data --area=1581.165 --chem=neq}
\end{itemize}

\subsection{Output Files}
Providing that NENZF-r has been properly invoked and that the Eilmer3 nozzle simulation has not crashed, users can expect the files detailed in Table \ref{tab:output-files} to be produced. Note that these files are in addition to those produced by Eilmer3.

\begin{table}[!ht]
\centering
\caption{NENZF-r output files}
\begin{tabular}{p{0.22\linewidth}|p{0.25\linewidth}|p{0.43\linewidth}} 
\hline
Filename & Default Filename & Description \\ 
\hline
\textit{job}-estcj.dat & nozzle-estcj.dat & Output file detailing the results of the estcj calculation.\\
\textit{exitfile} & nozzle-exit.data & Flow properties in the exit plane.\\
\textit{job}-centerline.data & nozzle-centreline.data & Flow properties along the centerline.\\
\textit{job}-exit.stats & nozzle-exit.stats & The integrated nozzle exit flow properties.\\
\textit{job}-viscous.data & nozzle-viscous.dat & Calculated wall shear stress ($\tau_w$) and $y^+$.\\
\hline 
\end{tabular}
\label{tab:output-files}
\end{table}

Currently the integrated nozzle exit flow properties are calculated by taking an area-weighted average over the core of the nozzle using the data contained in \textit{nozzle-exit.data}. Quite arbitarily, the edge of the core has been taken to be at \textbf{2/3rds} of the nozzle exit radius.

\subsection{Gas Models and Reaction Schemes}



