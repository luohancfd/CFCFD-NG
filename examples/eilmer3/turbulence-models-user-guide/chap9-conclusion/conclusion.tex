% conclusion.tex

\newpage
\section{Conclusion}
\label{chapter-conclusion}
Wilcox's 2006 $k$-$\omega$ turbulence model has been implemented in 
Eilmer3 and validated against seven test cases (four two-dimensional and three three-dimensional) that had flowfields 
representative of those to be expected in hypersonic airbreathing 
propulsion. 

The first test case showed that Eilmer3 can be used to 
predict turbulent boundary layer profiles 
and skin friction distribution for a Mach 4 flow on a flat plate 
despite the turbulence model's sensitivity to user-defined freestream 
turbulence properties. The second test case, an axisymmetric analogy 
of the flat plate test case, showed that Eilmer3 can be used to 
predict turbulent heat flux distribution on the external surface of a cylinder 
in a Mach 9 flow. This test case also demonstrated that $y^+$ values
and maximum cell aspect ratios should be kept below 1 and 600
respectively for turbulent CFD simulations. The third test case showed
that Eilmer3 can be used to predict the boundary layer profiles of
a Mach 2 flow past a backward facing step and the fourth test case
demonstrated that Eilmer3 can be used to predict the turbulent mixing 
phenomena of two coaxial Mach 2 jets. 
%

The fifth test case was successful in closely replicating the data produced by the original two-dimensional flat plate case. Adding a third-dimension resulted in the development of $Y$ velocity components between the cells, and a lowered parallel ($X$ direction) velocity within the turbulent boundary layer for the $C_f$ calculations. The differences in skin friction coefficient $C_f$, but close correlation with other flow properties, helped to identify a potential bug in the solver caused by extending the case into 3D. 
The sixth test case underwent a variety of domain and mesh alterations, however was still unable to produce reliable results due to the presence of checker-boarding on the leading edge of the fin. An investigation into the cause of this phenomenon and altering the mesh to prevent it is a possible avenue of future work.
The seventh test case indicates the progress made towards validating the Mach 1 injection of air into a Mach 4 cross-flow. The featured solutions are close to steady-state, however an extended simulation time would allow final results to be produced. Excellent correlation with computational and experimental data were found thus far. Avenues of future work for this case include the extension of the simulation to allow a steady or quasi-steady solution to be obtained, and the increasing of the mesh resolution would allow more accurate shock development on the symmetry plane, thus producing a more accurate shock structure for validation.