% code.tex

\newpage
\section{Code}
\label{code}

The following is the space marching solver code extracted directly out of the \textit{main.cxx} file from eilmer3.

{\scriptsize
\begin{verbatim}
int integrate_blocks_in_sequence( void )
// This procedure integrates the blocks one-at-a-time in sequence.
//
// The idea is to approximate the space-marching approach of sm_3d
// and, hopefully, achieve a significant speed-up over integration
// of the full array of blocks.
//
// It is assumed that we have the restricted case 
// of all the blocks making a single line (west to east) with
// supersonic flow at the west face of block 0 and 
// extrapolate_out on the east face of the final block.
//
// The above assumptions still stand but the code has been modified
// such that the calcluation now does two active blocks at a time.
// The calculation initially does blocks 0 and 1, and then moves over
// one block for the next step calculating blocks 1 and 2 next.
// This process is then repeated for the full calculation

{
    global_data &G = *get_global_data_ptr();
    Block *bdp;
    double time_slice = G.max_time / G.nblock;
    BoundaryCondition *bcp_save;

    // Initially deactivate all blocks
    for ( int jb = 0; jb < G.nblock; ++jb ) {
	bdp = get_block_data_ptr(jb);
	bdp->active = 0;
    }

    cout << "Integrate Block 0 and Block 1" << endl;

    // Start by setting up block 0

    // Activate block 0
    bdp = get_block_data_ptr(0);
    bdp->active = 1;
    // Apply the assumed SupINBC to the west face and propogate across the block
    bdp->bcp[WEST]->apply_inviscid(0.0);
    bdp->propagate_data_west_to_east( G.dimensions );
    bdp->apply( &FV_Cell::encode_conserved, bdp->omegaz, "encode_conserved" );
    // Even though the following call appears redundant at this point,
    // fills in some gas properties such as Prandtl number that is
    // needed for both the cfd_check and the BLomax turbulence model.
    bdp->apply( &FV_Cell::decode_conserved, bdp->omegaz, "decode_conserved" );

    // Now set up block 1

    // Activate block 1
    bdp = get_block_data_ptr(1);
    bdp->active = 1;

    // Save the original east boundary condition and apply the temporary
    // ExtrapolateOutBC for the calculation
    bcp_save = bdp->bcp[EAST];
    bdp->bcp[EAST] = new ExtrapolateOutBC(*bdp, EAST, 0);

    // Read in data from block 0 and propogate across the block
    exchange_shared_boundary_data( 1, COPY_FLOW_STATE);
    bdp->propagate_data_west_to_east( G.dimensions );
    bdp->apply( &FV_Cell::encode_conserved, bdp->omegaz, "encode_conserved" );
    bdp->apply( &FV_Cell::decode_conserved, bdp->omegaz, "decode_conserved" );

    // Integrate just the first two blocks in time, hopefully to steady state.
    set_block_range(0,1);
    integrate_in_time( time_slice );


    // The rest of the blocks.
    for ( int jb = 2; jb < (G.nblock); ++jb ) {
	// jb-2, jb-1, jb
	// jb-2 is the block to be deactivated, jb-1 has been iterated and now
	// becomes the left most block and jb is the new block to be iterated
	cout << "Integrate Block " << jb << endl;
	// Make the block jb-2 inactive.
	bdp = get_block_data_ptr(jb-2);
	bdp->active = 0;

	// block jb-1 - reinstate the previous boundary condition on east face
	// but leave the block active
	bdp = get_block_data_ptr(jb-1);
	delete bdp->bcp[EAST];
	bdp->bcp[EAST] = bcp_save;

	// Set up new block jb to be integrated
	bdp = get_block_data_ptr(jb);
	bdp->active = 1;

	if ( jb < G.nblock-1 ) {
	    // Cut off the east boundary of the current block 
	    // from the downstream blocks if there are any.
	    bcp_save = bdp->bcp[EAST];
	    bdp->bcp[EAST] = new ExtrapolateOutBC(*bdp, EAST, 0);
	}
	// Now copy the starting data into the WEST ghost cells
	// and propagate it across the current block.
	exchange_shared_boundary_data( jb, COPY_FLOW_STATE );
	bdp->propagate_data_west_to_east( G.dimensions );
	bdp->apply( &FV_Cell::encode_conserved, bdp->omegaz, "encode_conserved" );
	bdp->apply( &FV_Cell::decode_conserved, bdp->omegaz, "decode_conserved" );
	// Integrate just the two currently active blocks in time,
	// hopefully to steady state.
	set_block_range(jb-1, jb);
	integrate_in_time( (jb+1)*time_slice );
    }
    // Before leaving, we want all blocks active for output.
    for ( int jb = 0; jb < G.nblock; ++jb ) {
	bdp = get_block_data_ptr(jb);
	bdp->active = 1;
    }
    set_block_range(0, G.nblock - 1);
    return SUCCESS;
} // end integrate_blocks_in_sequence()
}
\end{verbatim}
}