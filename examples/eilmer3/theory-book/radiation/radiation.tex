% thermochem.tex

\section{Radiation transport}
\label{sec:radiation_transport}

The radiation source term in the Navier-Stokes equations (see Equation~\ref{eq:Q_rad}) is the negative divergence of the local radiative heat flux vector:

\begin{equation}
 - \nabla \cdot \vec{q}_\text{rad} = - \nabla \cdot \int_0^\infty \vec{I}_\nu d \nu \label{eq:div_q_rad}
\end{equation}

\noindent For application to computational grids it is convenient to express Equation~\ref{eq:div_q_rad} as the difference between the local emission and absorption:

\begin{equation}
- \nabla \cdot \vec{q}_\text{rad} =  \int^{\infty}_{0} \int_{4\pi} \kappa_{\nu} I_{\nu} d \omega d \nu - 4 \pi \int^{\infty}_{0} j_{\nu} d \nu
 \label{eq:divq_rad}
\end{equation}

A variety of models are implemented in Eilmer3  to solve for the radiative source term:

\begin{enumerate}
 \item Optically-thin model
 \item Tangent-slab model
 \item Modified Discrete Transfer model
 \item Photon Monte-Carlo model
\end{enumerate}

All models can be used on planar, axisymmetric or three-dimensional multi-block grids\footnote{The validity of the tangent-slab model, however, may be questionable when applied to certain geometrical configurations.}, and require the spectral emission $j_\nu$ and absorption $\kappa_\nu$ coefficients at each volume element and intensity $I_\nu$ emitted from each surface element as input.
Descriptions of these of models are provided in \textsection~\ref{sec:optically_thin} to~\ref{sec:photon_monte_carlo} respectively.
Firstly, however, the flowfield coupling methodology is outlined in the following section.

\subsection{Flowfield coupling}
\label{sec:fc}

As thermal radiation is approximately proportional to $T^4$, the effect of radiation on gasdynamic processes can be important for high temperature gases. 
A useful parameter for estimating the degree of radiation-flowfield coupling for blunt body flows is the Goulard number~\cite{goulard}:

\begin{equation}
 \Gamma = \frac{2 q_\text{rad}}{ \frac{1}{2} \rho_\infty u_\infty^3 }
\end{equation}

\noindent where $q_\text{rad}$ is the radiative heat flux incident at the stagnation point, $\rho_\infty$ the freestream density and $u_\infty$ the freestream velocity.
The Goulard number is a measure of the conversion of energy flux in the freestream to radiative energy flux at the vehicle surface.
When the Goulard number becomes large ($\Gamma > 0.01$) radiation-flowfield coupling should be taken into consideration due to significant levels of radiative flux through the shock layer.

\par

As the radiation transport procedure is computationally expensive, it is desirable to update the radiation source terms at a reasonably low frequency (\textit{i.e.} a \textit{loosely} coupled approach).
When a radiation transport calculation at time-step $n$ is performed, $\rho$, $T_{e}$ and $\nabla \cdot \vec{q}_\text{rad}$ for each cell are stored\footnote{When a one temperature gas-model is used, $T_e = T$.}.
For each of the subsequent flowfield time-steps $m$ where a complete radiation transport calculation is not performed the radiative divergence is rescaled to account for variations in the gas-state:

\begin{equation}
 \left ( - \nabla \cdot \vec{q}_\text{rad} \right )_m = \left \lbrace \begin{array}{c} \frac{ \left ( \rho T_{e}^{4} \right )_{m} }{ \left ( \rho T_{e}^{4} \right )_{n} } \left ( - \nabla \cdot \vec{q}_\text{rad} \right )_n \text{ \hspace{0.6cm} for } \left ( - \nabla \cdot \vec{q}_\text{rad} \right )_n > 0 \\ \\  \frac{ \left ( \rho T_{e}^{-4} \right )_{m} }{ \left ( \rho T_{e}^{-4} \right )_{n} } \left ( - \nabla \cdot \vec{q}_\text{rad} \right)_n \text{ \hspace{0.6cm} for } \left ( - \nabla \cdot \vec{q}_\text{rad} \right )_n < 0 \end{array} \right .
 \label{eq:divq_scale}
\end{equation}

The frequency of the radiation update is dependent on the transient behavior of the flowfield;  for example, during the initial period of flow development the update frequency needs to be high to account for shifting shock positions, while as the solution approaches steady state the update frequency can be substantially reduced.

\subsection{Optically-thin model}
\label{sec:optically_thin}

The optically-thin model neglects reabsorption, reducing Eq.~\ref{eq:divq_rad} to:

\begin{equation}
 - \nabla \cdot \vec{q}_\text{rad} = - 4 \pi \int^{\infty}_{0} j_{\nu} d \nu \label{eq:divq_rad_OT}
\end{equation}

\noindent The radiative divergence therefore becomes a function of the local gas state only.
This is a good model to use when emission is much higher than absorption, as the effect of radiation-flowfield coupling can then by modelled with minimum computational effort.

\subsection{Tangent-slab model}
\label{sec:TS_model}

The tangent-slab model allows the effect of reabsorption to be modelled while avoiding a complete directional integration of the local intensity field.
A one-dimensional variation of properties is considered along each line-of-sight normal to the vehicle surface.
Computationally, a line of cells is used to represent the normal line-of-sight as demonstrated in Figure~\ref{fig:FireII-TS}.
If a single column of blocks is used to define the computational domain-between the inflow and vehicle surface boundaries, the tangent-slab model is inherently parallelisable as all the information required for the calculation is contained in the local block.

\par

\begin{figure}[h]
 \center
 \includegraphics[scale=0.85]{radiation/figures/FireII-TS.png}
 \caption{Schematic of tangent-slab calculation domain along lines of cells on a multi-block grid.}
 \label{fig:FireII-TS}
\end{figure} 

Given that the infinite-slab arrangement will result in zero net radiative flux in the transverse directions, the definition of the radiative divergence for slab $i$ reduces to:

\begin{equation}
 - \left( \nabla \cdot \vec{q}_\text{rad} \right )_{i} = - \left ( \frac{\partial q_\text{rad}}{\partial s} \right )_{i} \approx \frac{ - \left ( q_\text{rad}^{(i+1)} - q_\text{rad}^{(i)} \right ) }{ \Delta s_{i} }
 \label{eq:divq_TS}
\end{equation}

\noindent where $q_{rad}^{(i)}$ is the radiative flux at the $i^{th}$ cell interface (\textit{i.e.} preceeding the cell from right-to-left) and $\Delta s_{i}$ is the width of the cell in the (approximately) body normal direction.
The solution for the radiative flux in a gaseous medium between two parallel, infinite-slabs as a function of the spectral optical thickness $\tau_{\nu}$ is derived in the text by Modest~\cite{Mod03}.
If the computational domain is considered to be a collection of $N_\text{slabs}$ isothermal slabs with the spectral range discretised into $N_\nu$ frequency intervals the radiative flux at interface $i$ can be expressed as:

\begin{equation}
 q_\text{rad}^{(i)} = \sum_{k=1}^{N_\nu} 2 \pi I_{\nu_k, \text{wall}} E_{3} \left ( \tau_{\nu_k}^{(i)} \right ) + 2 \pi \sum_{j=1}^{N_{\text{slabs}}} S_{\nu_k}^{(j)} \left [ E_{3} \left ( \left  | \tau_{\nu_k}^{(i)} - \tau_{\nu_k}^{(j)} \right  | \right ) - E_{3} \left ( \left  | \tau_{\nu_k}^{(i)} - \tau_{\nu_k}^{(j-1)} \right  | \right ) \right ] \Delta \nu_k
 \label{eq:TS_modest}
\end{equation}

\noindent where $S_{\nu_k}^{(i)}$ is the source function for the $i^{th}$ isothermal cell at frequency $\nu_k$, $I_{\nu_k, \text{wall}}$ is the intensity emitted by the wall and the optical thickness $\tau_{\nu_k}^{(i)}$ is calculated as:

\begin{equation}
 \tau_{\nu_k}^{(i)} = \sum_{l=1}^{i} \kappa_{\nu_k}^{(l)} \Delta s_{l}
 \label{eq:tau_nu}
\end{equation}

\noindent The $E_{n}$ term is the $n^\text{th}$ order exponential integral with form:

\begin{equation}
 E_{n} ( x ) = \int_{1}^{\infty} \omega^{-n} \text{exp} \left ( - \omega x \right ) d \omega
 \label{eq:E_n}
\end{equation}

\noindent The $E_{3}$ curve fit derived  by Johnston~\cite{JohnPhd} is implemented for the Eilmer3  tangent-slab model:

\begin{equation}
 E_{3} ( x ) = 0.0929 e^{-4.08x} + 0.4071e^{-1.33x}
 \label{eq:E3}
\end{equation}

\noindent The intensity emitted by the wall with emissivity $\epsilon_{\text{wall}}$ is calculated as~\cite{VK65}:

\begin{equation}
 I_{\nu, \text{wall}} = 2 \pi \epsilon_{\text{wall}} \sigma T_{\text{wall}}^{4}
 \label{eq:qwall}
\end{equation}

\noindent where $T_{\text{wall}}$ is the blackbody wall temperature.  

\subsection{Modified discrete transfer model}
\label{sec:discrete_transfer}

A modified version of the standard discrete transfer model~\cite{shah, elbert_cinnella, karl2001} has been implemented in the \texttt{Eilmer3}  framework.
This is a ray-tracing based model, where the radiative divergence and heat-fluxes are determined by direct numerical integration of the radiant energy field over direction and space via the generation of a `radiation sub-grid' mapped over the CFD grid.
The radiation sub-grid consists of rays distributed isodirectionally from each emitting element in the flowfield, with the flow state and radiation spectra defined at distributed points along each ray.
An example of a radiation sub-grid on a simple axisymmetric CFD grid is illustrated in Figure~\ref{fig:radiation-subgrid}.
Although the modified discrete transfer model has been implemented in Eilmer3  for planar, axisymmetric and 3D geometries, only the planar and axisymmetric formulations are described herein.

\begin{figure}[h]
\includegraphics[width=\linewidth]{radiation/figures/axi-radiation-subgrid.pdf}
 \caption{Example radiation sub-grid on a simple axisymmetric grid (Reference~\cite{karl2001}).}
 \label{fig:radiation-subgrid}
\end{figure}

\par

The Discrete Transfer model proposed by Elbert and Cinella~\cite{elbert_cinnella} uses the radiation sub-grid to solve directly for the radiative divergence via Eq.~\ref{eq:divq_rad}.
The modified Discrete Transfer model method proposed by Karl~\cite{karl2001} uses the radiation sub-grid to solve for the heat flux vectors throughout the flowfield, from which the divergence at the cell centres can be calculated.
In contrast, the modified Discrete Transfer model implemented in \texttt{Eilmer3} uses the radiative sub-grid to transport packets of radiant energy through the computational domain.
This is similar to a photon Monte-Carlo method in that radiation is treated as a discrete quantity rather than a continuous field, however the ray-distribution is kept uniform and energy attenuation is not modelled in a statistical fashion.

\subsubsection{Governing equations}

The total radiative divergence for a finite-volume cell is calculated as the difference between the total emissive power $E_{\text{ems.}}$ and absorptive power $E_{\text{abs.}}$ divided by the cell volume $V$:

\begin{equation}
 - \nabla \cdot \vec{q}_\text{rad} = \frac{ - \left ( E_{\text{ems.}} - E_{\text{abs.}} \right ) }{V}
 \label{eq:my_divq}
\end{equation}

\noindent where:

\begin{eqnarray}
 E_{\text{ems.}} &=& \int_{V} \int_{0}^{{4\pi}} \int_{\nu_{\text{min}}}^{\nu_{\text{max}}} j_{\nu} d\nu d\omega dV = \sum_{r}^{N_{\text{ems. rays}}} \sum_{n}^{N_{\nu}} E_{r,p}^0 \label{eq:E_emission} \\
 E_{\text{abs.}} &=& \sum_{r}^{N_{\text{abs. rays}}} \sum_{n}^{N_{\nu}} \left ( - \Delta E_{r,p} \right ) \label{eq:E_absorption} 
\end{eqnarray}

\noindent Here $N_{\text{abs. rays}}$ is the total number of ray segments traversing the current cell, $N_{\text{ems. rays}}$ is the total number of rays emitted by this cell and the frequency domain has been divided into $N_{\nu}$ intervals between $\nu_{min}$ and $\nu_{max}$. 
$E_{r,n}^0$ is the initial power carried by photon packet $n$ with frequency interval $\Delta \nu_{n}$ from ray $r$ with solid angle $\Delta \omega_{r}$:

\begin{equation}
 E_{r,p}^0 = j_{\nu} \Delta \nu_{n} \Delta \omega_{r} V
\end{equation}

The radiative heat flux incident on wall elements $q_\text{rad}$ is calculated as the sum of the remaining energy $E_{r,n}$ from all incident rays $N_\text{inc. rays}$ divided by the wall element area $A$:

\begin{equation}
 q_\text{rad} = \frac{ E_\text{abs.} }{A} =  \frac{ \displaystyle \sum_{r}^{N_{\text{inc. rays}}} \displaystyle \sum_{p}^{N_{\nu}} E_{r,n} }{ A }
 \label{eq:my_divq}
\end{equation}

\noindent Note that all surfaces are currently treated as blackbodies, and therefore reflection is not considered.

\subsubsection{Absorption model}

The radiative power lost by photon packet $p$ while traversing from points $s_{i}$ to $s_{f}$ along a ray is calculated from Beers law:

\begin{equation}
 - \Delta E_{r,n} = - ( 1 - e^{-\kappa_{\nu_{n}}(s) \Delta s } ) E_{r,n}(s_{i})
 \label{eq:E_rn}
\end{equation}

\noindent where $\Delta s = s_{f} - s_{i}$.

\subsubsection{Determination of ray density distribution}
\label{sec:ray_density_distribution}

A number of approaches are implemented for determining the number of rays emitted from each volume element, $N_{\text{ems. rays}}$:

\begin{description}
 \item[No clustering] All elements emit the same number of photons
 \item[Area clustering] The number of emitted photons for each element is proportional to the total radiative power density times area (2D grids only)
 \item[Volume clustering] The number of emitted photons for each element is proportional to the total radiative power density times volume (2D and 3D grids)
\end{description}

\noindent The area and volume clustering approaches allow more rays to be emitted from elements emitting more radiation, resulting in each ray in the calculation handling approximately the same energy.
The area clustering approach is included to avoid the problem of very small cell volumes near the axis of symmetry for axisymmetric grids.
The number of rays emitted from a volume element $i$ via the area clustering approach is:

 \begin{equation}
 N_{i} = \text{int} \left [ N_{\text{max}} \frac{ e_{i} A_i }{ \text{max}_i ( e_{i} A_i ) } \right ] \label{eq:N_photons}
\end{equation}

\noindent where $N_{\text{max}}$ is the maximum permitted number of rays, $e_{i}$ is the total radiative power density from volume element $i$ and $A_i$ is the area of element $i$.
The total radiative power density for volume element $i$ is calculated as:

\begin{equation}
 e_i = \int_{\nu_{\text{min}}}^{\nu_{\text{max}}} j_{\nu,i} d \nu
\end{equation}

Similarly, the number of photons emitted from a volume element $i$ via the volume clustering approach is:

 \begin{equation}
 N_{i} = \text{int} \left [ N_{\text{max}} \frac{ e_{i} V_i }{ \text{max}_i ( e_{i} V_i ) } \right ] \label{eq:N_photons}
\end{equation}

\noindent where $V_i$ is the volume of volume element $i$.
Analogous expressions are implemented for determining the number of rays emitted from surface elements, where length and area weightings are used in place of area and volume.

\subsubsection{Radiation sub-grid}
\label{sec:radiation_subgrid}

The radiation sub-grid for planar and axisymmetric grids of quadrilateral cells described by Elbert and Cinella~\cite{elbert_cinnella} and Karl~\cite{karl2001} is implemented with slight modifications.
The radiation sub-grid coordinates of a point at distance $L$ along a ray with elevation and azimuth angles $\phi$ and $\theta$ originating from position $x_0$, $y_0$ are:

\begin{eqnarray}
 x' &=& x_0 + L \text{cos} \left ( \phi \right ) \text{cos} \left ( \theta \right ) \\
 y' &=& y_0 + L \text{sin} \left ( \phi \right ) \\
 z' &=& L \text{cos} \left ( \phi \right ) \text{sin} \left ( \theta \right )
\end{eqnarray}

\noindent For 2D calculations, the corresponding CFD grid coordinates are then calculated from the following transformation:

\begin{eqnarray}
 x &=& x' \\
 y &=& \sqrt{y'^2 + z'^2}
\end{eqnarray}

\noindent This transformation has the effect of reflecting rays intersecting the symmetry axis at $y=0$, as required (see Figure~\ref{fig:radiation-subgrid}).
For planar geometries the radiation sub-grid is formed in the $x$--$y$ plane as the CFD domain is symmetrical along the $z$ axis.
For ray $i$ of $N_\text{rays}$ the elevation and azimuth angles are calculated as:

\begin{eqnarray}
 \theta &=& \left \lbrace \begin{array}{ccc} 0 & \text{for} & 0 \leq \alpha < \pi/2  \\ 
 									    \pi & \text{for} & \pi/2 \leq \alpha < 3\pi/2 \\
									     0 & \text{for} & 3\pi/2 \leq \alpha < 2\pi \end{array} \right . \\
 \phi &=& \left \lbrace \begin{array}{ccc} \alpha & \text{for} & 0 \leq \alpha < \pi/2  \\ 
 									    \pi - \alpha & \text{for} & \pi/2 \leq \alpha < 3\pi/2 \\
									     \alpha - \pi/2 & \text{for} & 3\pi/2 \leq \alpha < 2\pi \end{array} \right .
\end{eqnarray}

\noindent where  $\alpha = i \frac{2\pi}{N_\text{rays}}$.
For axisymmetric geometries the radiation sub-grid must be formed in three-dimensions as the CFD domain is radially symmetrical about $y=0$.
Elbert and Cinella~\cite{elbert_cinnella} and Karl~\cite{karl2001} use the symmetric vertices of the icosahedron platonic solid to generate the axisymmetric elevation and azimuth angles.
This approach, however, limits the total number of rays to a set of fixed values as the 20 icosahedron faces must be subdivided equally to achieve approximate uniformity.
To allow for an arbitrary number of axisymmetric rays, the so-called `Golden Section Spiral' method~\cite{Spiral} of generating uniform points on a sphere has been implemented.
As illustrated in Figure~\ref{fig:gspirals}, this method arranges nodes on the surface of a unit sphere by considering spirals with successive longitudes chosen according to the golden ratio $\frac{\sqrt{5}-1}{2}$.

\begin{figure}[h]
\centering
\subfloat[$N_{rays} = 23$]{\scalebox{0.9}{\includegraphics{radiation/figures/golden_spiral-A.png}}} \hspace{0.2cm}
\subfloat[$N_{rays} = 133$]{\scalebox{0.9}{\includegraphics{radiation/figures/golden_spiral-B.png}}} \hspace{0.2cm}
\subfloat[$N_{rays} = 480$]{\scalebox{0.9}{\includegraphics{radiation/figures/golden_spiral-C.png}}}
\caption{Approximately uniform points on a sphere generated via the `Golden Section Spiral' method~\cite{Spiral}.}
\label{fig:gspirals}
\end{figure}

The unit sphere coordinates of ray $i$ of $N_\text{rays}$ are calculated as:

\begin{eqnarray}
 x^\ast &=& r \text{cos}(\alpha) \\
 y^\ast &=& w \\
 z^\ast &=& r \text{sin}(\alpha)
\end{eqnarray}

\noindent where:

\begin{eqnarray}
 w &=& i \frac{2}{N_\text{rays}} - 1 + N_\text{rays} \\
 r &=& \sqrt{ 1 - w^2 } \\
 \alpha &=& i \pi ( 3 - \sqrt{5} )
\end{eqnarray}


\noindent The elevation and azimuth angles are then calculated as:

\begin{eqnarray}
 \phi &=& \text{arcsin} \left ( y^\ast \right ) \\
 \theta &=& \left \lbrace \begin{array}{ccccc} \text{arctan} \left ( \frac{z^\ast}{x^\ast} \right ) & \text{ for } & x^\ast > 0 & \text{and} & z^\ast > 0 \\
 										  \pi / 2                                                                 & \text{ for } & x^\ast = 0 & \text{and} & z^\ast > 0 \\
  										  \pi - \text{arctan} \left ( \frac{z^\ast}{-x^\ast} \right ) & \text{ for } & x^\ast < 0 & \text{and} & z^\ast > 0 \\ 
										  \pi                                                                      & \text{ for } & x^\ast < 0 & \text{and} & z^\ast = 0 \\
										  \pi + \text{arctan} \left ( \frac{-z^\ast}{x^\ast} \right ) & \text{ for } & x^\ast < 0 & \text{and} & z^\ast < 0 \\ 
										  3 \pi / 2                                                              & \text{ for } & x^\ast = 0 & \text{and} & z^\ast < 0 \\ 
										  2\pi - \text{arctan} \left ( \frac{-z^\ast}{x^\ast} \right ) & \text{ for } & x^\ast > 0 & \text{and} & z^\ast < 0 \\
										  0                                                                        & \text{ for } & x^\ast > 0 & \text{and} & z^\ast = 0 \\ \end{array} \right .
\end{eqnarray}

\noindent The angular distribution of the rays for this method is not as uniform as that for the subdivided icosahedron approach, however the ray number flexibility is a distinct advantage.

\subsubsection{Ray-tracing}
\label{sec:ray_tracing}

The core of the ray-tracing method are cell searching algorithms that allows the radiation sub-grid to be mapped onto the CFD grid, Figure~\ref{fig:cell-searching}.
Given the coordinates of a point along a ray, these algorithms solve for what cell this point lies in.

\begin{figure}[h]
 \centering
 \includegraphics[width=0.5\linewidth]{radiation/figures/cell_searching.pdf}
 \caption{Mapping of the radiation sub-grid onto the CFD grid, Reference~\cite{karl2001}.}
 \label{fig:cell-searching}
\end{figure}

\paragraph{Two dimensional grids}

For a given trial cell with indices $i$ and $j$ the following cross-products are evaluated:

\begin{eqnarray}
  a_{12} \cdot \vec{n} &=& \vec{r}_{01} \times \vec{r}_{12} \\
  a_{23} \cdot \vec{n} &=& \vec{r}_{02} \times \vec{r}_{23} \\
  a_{34} \cdot \vec{n} &=& \vec{r}_{03} \times \vec{r}_{34} \\
  a_{41} \cdot \vec{n} &=& \vec{r}_{04} \times \vec{r}_{41}
\end{eqnarray}

\noindent where $\vec{n}$ is the unit vector normal to the x-y plane.
If all four cross-products are positive the point $p$ is inside the trial cell.
Otherwise, the indices of the next trial cell are obtained according to:

\begin{eqnarray}
 \text{if } \left ( a_{23} < 0 \text{ and } a_{41} > 0 \right ) \text{ then } i = i + 1 \\ 
 \text{if } \left ( a_{23} > 0 \text{ and } a_{41} < 0 \right ) \text{ then } i = i - 1 \\
 \text{if } \left ( a_{12} < 0 \text{ and } a_{34} > 0 \right ) \text{ then } j = j + 1 \\
 \text{if } \left ( a_{12} > 0 \text{ and } a_{34} < 0 \right ) \text{ then } j = j - 1 
\end{eqnarray}

\paragraph{Three dimensional grids}

For a given trial hexahedral cell with indices $i$, $j$ and $k$ the following product is evaluated for each face:

\begin{equation}
 a = \vec{v}_1 \cdot \left ( \vec{v}_2 \times \vec{v}_3 \right )
\end{equation}

\noindent where $\vec{v}_1$, $\vec{v}_2$ and $\vec{v}_3$ are any three consecutive vertices in the clockwise direction (observed from the exterior of the cell) of the four vertices forming the face.
The sign of $a$ determines which side of the face the point lies.
If all $a$ values are negative, the point is inside the cell.
Otherwise, the indices of the next trial cell are obtained according to:

\begin{eqnarray}
 \text{if } \left ( a_\text{east} > 0 \text{ and } a_\text{west} < 0 \right ) \text{ then } i = i + 1 \\ 
 \text{if } \left ( a_\text{east} < 0 \text{ and } a_\text{west} > 0 \right ) \text{ then } i = i - 1 \\ 
 \text{if } \left ( a_\text{north} > 0 \text{ and } a_\text{south} < 0 \right ) \text{ then } j = j + 1 \\ 
 \text{if } \left ( a_\text{north} < 0 \text{ and } a_\text{south} > 0 \right ) \text{ then } j = j - 1 \\ 
 \text{if } \left ( a_\text{top} > 0 \text{ and } a_\text{bottom} < 0 \right ) \text{ then } k = k + 1 \\ 
 \text{if } \left ( a_\text{top} < 0 \text{ and } a_\text{bottom} > 0 \right ) \text{ then } k = k - 1 
\end{eqnarray}

\noindent As the mapping of the radiation sub-grid is performed successively for each point along a ray, a good initial guess is always available and the above methods are highly efficient.

\subsubsection{Solution procedure}

The procedure for calculating the radiative divergence field and heat flux profiles in Eilmer3  with the modified Discrete Transfer model is as described in Figure~\ref{fig:eilmer-DT-solve-procedure}.
Note that the frequency range is able to be divided up into multiple blocks, allowing the total memory requirement for the calculation to be reduced.
This is necessary when the CFD and spectral grids are vey fine.

\begin{figure}[htbp]
\small
\begin{center}
\fbox{\parbox{13cm}{
 \begin{enumerate}
  \item Perform geometric ray-tracing to define the radiation sub-grid
  \item Calculate emission and absorption spectra for each cell and wall element
  \item Trace each photon packet through the grid
  \begin{enumerate}
   \item Subtract emitted energy from origin cell
   \item Add absorbed energy to each traversed cell
   \item Record exiting energy on wall elements 
  \end{enumerate}
  \item Evaluate $- \nabla \cdot \vec{q}_\text{rad}$ for each cell and $q_\text{rad}$ for each wall element from the results
  \item Repeat steps 2 - 4 for each frequency block
 \end{enumerate}
}}
\end{center}
\caption{Sequence of operations for calculating the radiative divergence field and heat flux profiles in \texttt{Eilmer3} with the modified Discrete Transfer model.}
\label{fig:eilmer-DT-solve-procedure}
\end{figure}

This procedure is parallelised via \verb OpenMP  where each processor has access to all data describing the computational domain.
The calculation is divided amongst the available processors on a cell-by-cell basis when computing spectra, and a ray-by-ray basis when tracing and integrating along lines-of-sight.

%\subsubsection{Verification and validation}

%It is necessary at this point to verify the implementation and validate the theory of the ray-tracing radiation transport model.
%For this purpose, the infinite-cylinder test case proposed by Karl~\cite{karl2001} is considered.
%This test case is performed at four different grid and ray resolutions and with 1, 2 and 4 CPU cores to test the convergence and parallelisibility of the method.

%\par

%\begin{figure}[b!]
%\centering
%\includegraphics[width=\linewidth]{radiation/figures/greyslab_1B.pdf}
%\caption{Single-block computational domains for the infinite-slab and infinite-cylinder test cases (not-to-scale).}
%\label{fig:GS_domains}
%\end{figure}

%The computational domain for the planar and axisymmetric test cases is shown in Figure~\ref{fig:GS_domains}.
%The present implementation of the ray-tracing algorithm only permits reflective boundary conditions along the $y = 0$ line; the east and west boundaries are therefore considered to be walls, rather than symmetry boundaries.
%As will be demonstrated, however, the aspect ratio of 10:1 is sufficient to permit the infinite-slab and infinite-cylinder approximations at the slab mid-section ($x=5$\,m).
%Grid resolutions of $8 \times 8$,  $16 \times 16$, $32 \times 32$ and $64 \times 64$ uniformly spaced cells are considered, with the number of rays set to 8, 16, 32 and 64 for the planar case and 32, 64, 128 and 256 rays for the axisymmetric case\footnote{More rays are required for axisymmetric geometries compared with planar geometries to achieve similar accuracy as the radiation sub-grid must be formed in three dimensions rather than just two.}.
%For the planar test case a temperature gradient is applied from $T_i = 0$~K at y = 0~m to $T_f = 10,000$~K at y = 1~m, while for the axisymmetric test case the temperature is constant at 10,000~K.
%A `grey-gas' is assumed with a constant absorption coefficient of $\kappa$ = 1.0~m$^{-1}$, making the total emissive power density:

%\begin{equation}
% J = \frac{\kappa \sigma T^{4}}{\pi}
% \label{eq:grey_emission}
%\end{equation}

%
%Although the radiation from a grey-gas can be described without considering spectral distributions, for this test case this is desirable so as to maintain similarity with the non-Planck spectra the model is designed to be applied to.
%The spectral emission and absorption coefficients are then calculated as:

%\begin{eqnarray}
%      j_{\nu} &=& \kappa B_{\nu}(T) \\
% \kappa_{\nu} &=& \kappa 
%\end{eqnarray}

%\noindent where $B_{\nu}(T)$ is the Planck function:

%\begin{equation}
% B_{\nu}(T) = \frac{2 h}{c^3} \frac{\nu^3}{e^\frac{h \nu }{k T} -1 }
%\end{equation}

%The spectral range considered is 10 to 3000\,nm, and is discretised with 500 intervals.
%Integrating the Planck function with this discretisation matches the Stephan-Maxwell equation to within 0.1\%.
%The exact solution for the planar test case is obtained from the tangent-slab approximation presented in Equations~\ref{eq:divq_TS} to~\ref{eq:qwall}, and the exact solution for the axisymmetric test case is obtained from the analytical expressions presented by Sakai \textit{et al.}~\cite{SSM1998}.

%\paragraph{Planar infinite-slab test case}

%Before analysing the radiative diverge fields, it is useful to consider the heat flux profiles to verify the tangent-slab approximation.
%Figure~\ref{fig:planar_heat_flux} presents the heat-flux profiles along the top edge of the planar infinite-slab from the $8 \times 8$ grid with 8 rays, the  $16 \times 16$ grid with 16 rays and the $32 \times 32$ grid with 32 rays.
%The heat flux profiles for the ray-tracing calculations are approximately uniform for $3 \leq x \leq 7$, and the heat flux at x=5\,m converges to the tangent-slab approximation with increasing resolution.
%The test case is therefore well described by the tangent-slab approximation for the mid-point cross section at $x=5$\,m.

%\begin{figure}[t]
% \centering
% \includegraphics[width=0.85\linewidth]{radiation/figures/planar/heat_flux.pdf}
% \caption{Comparison of heat-flux profiles along the top edge of the planar infinite-slab.}
% \label{fig:planar_heat_flux}
%\end{figure}

%\par

%Figure~\ref{fig:planar_divq_profiles} presents the radiative divergence results for the planar infinite-slab test case; comparisons of radiative divergence are shown in the left column, and the resulting errors compared with the tangent-slab solution are shown in the right column.
%The profiles are taken from the mid-point cross section where $x=5$\,m.
%In all plots the optical thickness $\tau_y$ is used as the spatial coordinate, defined as:

%\begin{equation}
% \tau_y = \int_0^y \kappa_y dy \label{eq:OT}
%\end{equation}

%A complete summary of the planar infinite-slab results is presented in Table~\ref{tab:planar_results}.
%The quoted error in $\nabla \cdot \vec{q}_\text{rad}$ is the average of the absolute percentage difference along the $x=5$\,m profile referenced to the tangent-slab solution, and the quoted error in $q_\text{rad}$ is the percentage difference from the tangent-slab solution at $x=5$\,m and $y=1$\,m.
%All simulations were run using the serial version of the code on a single core of a Linux workstation with two Intel Dual Core Xeon Pro 5130 CPU's (4MB cache, 2.00GHz, 1333MHz FSB).

%\begin{figure}[p!]
%\centering
%\subfloat[$8 \times 8$ grid]{\scalebox{0.6}{\includegraphics{radiation/figures/planar/Q_rad_profiles_8x8-grid.pdf}} \scalebox{0.6}{\includegraphics{radiation/figures/planar/Q_rad_errors_8x8-grid.pdf}}} \\
%\subfloat[$16 \times 16$ grid]{\scalebox{0.6}{\includegraphics{radiation/figures/planar/Q_rad_profiles_16x16-grid.pdf}} \scalebox{0.6}{\includegraphics{radiation/figures/planar/Q_rad_errors_16x16-grid.pdf}}} \\
%\subfloat[$32 \times 32$ grid]{\scalebox{0.6}{\includegraphics{radiation/figures/planar/Q_rad_profiles_32x32-grid.pdf}} \scalebox{0.6}{\includegraphics{radiation/figures/planar/Q_rad_errors_32x32-grid.pdf}}} \\
%\caption{Comparison of radiative divergence solution and error profiles at $x=5$\,m for the planar infinite-slab radiation transport test case. The spatial coordinate is the optical thickness as defined in Equation~\ref{eq:OT}.}
%\label{fig:planar_divq_profiles}
%\end{figure}
% 
%\begin{table}[h]
% \centering
% \small
% \caption{Tabulated results for the planar infinite-slab radiation transport test case.}
% \label{tab:planar_results}
%  \begin{threeparttable}
% \begin{tabular*}{0.95\textwidth}%
%     {@{\extracolsep{\fill}}cccccc}
%  \hline \hline \textbf{Grid Resolution}                                 & \textbf{Ray resolution}      & \textbf{Memory}  &  \textbf{Wall time} & \textbf{Av. $\mathbf{| \nabla \cdot \vec{q}_\text{rad}|}$} & \textbf{$\mathbf{q_\text{rad}}$}  \\
%$\mathbf{(N_{\text{cells,x}}\times N_{\text{cells,y}})}$     & $\mathbf{N_\text{rays}}$ & \textbf{(MB)}        & \textbf{(s)}                      & \textbf{Error (\%)}                                                                & \textbf{Error (\%)}                \\
%  \hline  $8 \times 8$                                                               & 8                                          &  -                            &  0.46                               & 25.63                                                                                    &  -28.07  \\
%                                                                                                   & 16                                       &   -                           &   0.48                               & 5.17                                                                                      &  -4.51  \\
%                                                                                                   & 32                                       &  -                            &   0.79                               & 5.24                                                                                      &   1.19  \\
%              $16 \times 16$                                                          & 8                                         &  70                        &    1.46                              & 27.90                                                                                    &   -28.75 \\ 
%                                                                                                   & 16                                      &   76                       &    2.37                              & 5.35                                                                                      &   -5.95 \\
%                                                                                                   & 32                                      &   89                       &   4.36                               & 1.86                                                                                      &    0.67 \\
%              $32 \times 32$                                                          & 8                                         & 129                       &   9.95                               & 29.34                                                                                   &    -29.58 \\
%                                                                                                   & 16                                      & 168                       &   16.66                             & 6.70                                                                                     &   -7.18 \\
%                                                                                                   & 32                                      &  254                      &   31.69                             & 0.87                                                                                     &   -0.23 \\
%         
% \hline
% \end{tabular*}
% \end{threeparttable}
%\end{table}

%In general, the results demonstrate the ray-tracing radiation transport model converges towards the exact solution with increasing grid and ray resolution for planar geometries.
%An exception is the $8 \times 8$ cell grid with 32 rays, where the average error in $\nabla \cdot \vec{q}_\text{rad}$ increases from 5.17\% for the 16 ray case to 5.24\%.
%This anomaly can be attributed to the coarse grid resolution providing an inadequate description of the temperature gradient, resulting in convergence towards a solution with significant error.
%For the $16 \times 16$ and $32 \times 32$ grid cases, the error in both $\nabla \cdot \vec{q}_\text{rad}$ and $q_\text{rad}$ reduces in magnitude approximately in proportion to the number of rays squared.
%From Figure~\ref{fig:planar_divq_profiles}, the ray-traced solutions are observed to be most accurate close to the symmetry boundary at $\tau_y=0$ and least accurate at the outer boundary where $\tau_y=1$.
%This is due to the linear temperature gradient that has a maximum at $\tau_y=1$; as $\tau_y$ increases, the total emission increases due to the higher temperature, and thus the absolute error is able to increase.
%Overall the ray-tracing model performs very well for the planar test case.

%\paragraph{Axisymmetric infinite-cylinder test case}

%Figure~\ref{fig:axi_heat_flux} presents the heat-flux profiles along the outer surface ($y=1$\,m) of the cylinder from the $8 \times 8$ grid with 32 rays, the  $16 \times 16$ grid with 64 rays, the $32 \times 32$ grid with 128 rays and the $64 \times 64$ grid with 256 rays.
%The heat flux profiles are approximately uniform for $3 \leq x \leq 7$\,m, and the heat flux at x=5\,m is seen to converge towards the infinite-cylinder approximation with increasing resolution.
%The test case is therefore well described by the infinite-cylinder approximation for the mid-point cross section at $x=5$\,m.

%\begin{figure}[h!]
% \centering
% \subfloat[$ 0 \leq x \leq 10$\,m profile]{\includegraphics[width=0.85\linewidth]{radiation/figures/axi/heat_flux.pdf}} \\
% \subfloat[$ 3 \leq x \leq 7$\,m profile]{\includegraphics[width=0.85\linewidth]{radiation/figures/axi/heat_flux_detail.pdf}}
% \caption{Comparison of heat-flux profiles for the axisymmetric infinite-cylinder test case along the outer surface ($y=1$\,m).}
% \label{fig:axi_heat_flux}
%\end{figure}

%\par

%\begin{figure}[p]
%\centering
%\subfloat[$8 \times 8$ grid]{\scalebox{0.6}{\includegraphics{radiation/figures/axi/Q_rad_profiles_8x8-grid.pdf}} \scalebox{0.6}{\includegraphics{radiation/figures/axi/Q_rad_errors_8x8-grid.pdf}}} \\
%\subfloat[$16 \times 16$ grid]{\scalebox{0.6}{\includegraphics{radiation/figures/axi/Q_rad_profiles_16x16-grid.pdf}} \scalebox{0.6}{\includegraphics{radiation/figures/axi/Q_rad_errors_16x16-grid.pdf}}} \\
%\subfloat[$32 \times 32$ grid]{\scalebox{0.6}{\includegraphics{radiation/figures/axi/Q_rad_profiles_32x32-grid.pdf}} \scalebox{0.6}{\includegraphics{radiation/figures/axi/Q_rad_errors_32x32-grid.pdf}}} \\
%\caption{Comparison of radiative divergence solution and error profiles at $x=5$\,m for the axisymmetric infinite-cylinder radiation transport test case. The spatial coordinate is the optical thickness as defined in Equation~\ref{eq:OT}.}
%\label{fig:axi_divq_profiles}
%\end{figure}

%\begin{figure}[t]
%\ContinuedFloat
%\centering
%\subfloat[$64 \times 64$ grid]{\scalebox{0.6}{\includegraphics{radiation/figures/axi/Q_rad_profiles_64x64-grid.pdf}} \scalebox{0.6}{\includegraphics{radiation/figures/axi/Q_rad_errors_64x64-grid.pdf}}} \\
%\caption{(\textit{Continued}) Comparison of radiative divergence solution and error profiles at $x=5$\,m for the axisymmetric infinite-cylinder radiation transport test case. The spatial coordinate is the optical thickness as defined in Equation~\ref{eq:OT}.}
%\label{fig:axi_divq_profiles}
%\end{figure}

%Figure~\ref{fig:axi_divq_profiles} presents radiative divergence and error profiles for the infinite-cylinder test case; comparisons of radiative divergence are shown in the left column, and the resulting absolute errors compared with the infinite-cylinder solution are shown in the right column.
%The profiles are taken from the mid-point cross section where $x=5$\,m.
%A quantitative summary of the infinite-cylinder results are presented in Table~\ref{tab:axi_results}.
%The quoted error in $\nabla \cdot \vec{q}_\text{rad}$ is the average of the absolute percentage difference along the $x=5$\,m profile referenced to the infinite-cylinder solution, and the quoted error in $q_\text{rad}$ is the percentage difference from the infinite-cylinder solution at $x=5$\,m and $\tau_y=1$\,m.
%All simulations were run using the serial version of the code on a single core of a Linux workstation with two Intel Dual Core Xeon Pro 5130 CPU's.

%\begin{table}[h]
% \centering
% \small
% \caption{Tabulated results for the axisymmetric infinite-cylinder radiation transport test case.}
% \label{tab:axi_results}
%  \begin{threeparttable}
% \begin{tabular*}{0.95\textwidth}%
%     {@{\extracolsep{\fill}}cccccc}
%  \hline \hline \textbf{Grid Resolution}                                & \textbf{Ray resolution}      & \textbf{Memory}  &  \textbf{Wall time} & \textbf{Av. $\mathbf{\nabla \cdot \vec{q}_\text{rad}}$} & \textbf{$\mathbf{q_\text{rad}}$} \\
%$\mathbf{(N_{\text{cells,x}}\times N_{\text{cells,y}})}$    & $\mathbf{N_\text{rays}}$  & \textbf{(MB)}        & \textbf{(s)}              & \textbf{Error (\%)}                                                      & \textbf{Error (\%)}                \\
%  \hline  $8 \times 8$                                                              & 32                                         &   27                        & 0.53                        & 9.08                                                                             & 4.37 \\
%                                                                                                  & 64                                         &    29                       &  1.02                       & 6.42                                                                             & 4.22 \\
%                                                                                                  & 128                                      &    32                        & 1.97                        & 4.58                                                                             & 4.40\\\
%                                                                                                  & 256                                      &    43                        &  3.86                       & 4.88                                                                             & 4.55 \\
%              $16 \times 16$                                                         & 32                                         &   43                        & 3.79                        & 9.57                                                                             & 1.84 \\
%                                                                                                  & 64                                        &    60                        & 7.30                       & 4.59                                                                            & 2.07 \\
%                                                                                                  & 128                                      &   95                        & 14.63                      & 4.32                                                                             & 2.36 \\
%                                                                                                  & 256                                      &    161                    & 29.09                      & 2.49                                                                            & 2.20 \\
%              $32 \times 32$                                                         & 32                                         &  162                      & 29.39                      & 12.65                                                                          & 0.79 \\
%                                                                                                  & 64                                        &   278                      & 56.89                      & 4.74                                                                            & 0.86 \\
%                                                                                                  & 128                                      &   526                     & 113.64                    & 3.34                                                                            & 1.07 \\        
%                                                                                                  & 256                                      &  1003                   & 225.77                     & 1.40                                                                            & 9.58 \\
%              $64 \times 64$                                                         & 32                                         &  1006                   & 231.69                     & 15.39                                                                         & 0.17 \\
%                                                                                                  & 64                                        &   1876                   & 449.81                      & 6.38                                                                           & 0.21 \\
%                                                                                                  & 128                                      &   3676                   & 902.20                     & 3.17                                                                           & 0.42 \\        
%                                                                                                  & 256                                      &   7281                   & 1812.17                    & 1.24                                                                            & 0.34 \\
% \hline
% \end{tabular*}
% \end{threeparttable}
%\end{table}

%
%In general, the results demonstrate the ray-tracing model converges towards the exact solution with increasing grid and ray resolution for axisymmetric geometries.
%For all grid resolutions the average error in $\nabla \cdot \vec{q}_\text{rad}$ decreases with increasing ray resolution, and the smallest average error in $\nabla \cdot \vec{q}_\text{rad}$ is obtained from the simulation with the largest number of total rays ($32 \times 32$ cell grid with 128 rays per cell).
%While the error in $q_\text{rad}$ improves with increasing grid resolution, it is approximately constant for each grid resolution and only weakly dependent on ray resolution.
%The total memory usage and wall time is approximately linearly proportional to the total number of rays ( $N_{\text{cells,x}}\times N_{\text{cells,y}} \times N_\text{rays}$ ) as is to be expected.
%From the results in Figure~\ref{fig:axi_divq_profiles}, however, it is evident the ray-tracing model performs poorly close to the symmetry axis at $y=0$.
%For the $32 \times 32$ cell grid with 128 rays, for example, the error in $\nabla \cdot \vec{q}_\text{rad}$ at $\tau_y = 0.016$ is approximately 27.3\% whilst the error in $\nabla \cdot \vec{q}_\text{rad}$ at $\tau_y = 0.984$ is approximately  0.1\%.
%This behaviour is due to the vanishingly small volume per radian for cells close to the symmetry axis; rays emitted from elsewhere in the grid are unlikely to intersect such a small target volume.
%In an attempt to provide more rays in this region, the ray density was clustered towards the symmetry axis.
%With the same number of total rays, however, this strategy did not provide a consistent improvement in accuracy.
%Therefore it is suggested the ray density $N_\text{rays}$ be increased uniformly for all cells until the solution at the symmetry axis is acceptable.

%\par

%Table~\ref{tab:slab_resource_usage} compares the computational resource usage for the $32 \times 32$ cell grid and 128 ray case run with the \verb OpenMP  version of the code using 1, 2 and 4 CPU cores of the same Linux workstation used for the resolution studies. %  (2 x Intel Xeon Pro 5130 CPU's at 2.00GHz, 4MB cache, and 1333MHz FSB).
%The results indicate the ray-tracing model performs reasonably well under parallelisation for simple grids.
%The speed-ups for the 2 and 4 block cases are 1.70 and 3.19 respectively, giving parallelised code fractions of 70 and 73\% according to Gustafson's law~\cite{gustafson88}.
%Furthermore, the increase in the memory usage for the 2 and 4 core cases is minimal, being only 7.6 and 12.5\% respectively.
%Unfortunately there is significant efficiency penalty paid by using the \verb OpenMP  version of the code on a single core compared to using the serial version of the code.
%The serial version of the code completes the calculation in 113.64\,s, whilst the \verb OpenMP  version of the code completes the calculation is 160.75\,s -- a 42\% efficiency drop.
%This is due to the overhead associated with the \verb OpenMP  threading and is unavoidable.
%% Nevertheless, the speedup of the code under multiple processors is respectable.

%\begin{table}[b!]
% \centering
% \small
% \caption{Comparison of resource usage for the $32 \times 32$ cell grid and 128 ray infinite-cylinder radiation transport test case with 1, 2 and 4 blocks.}
% \label{tab:slab_resource_usage}
%  \begin{threeparttable}
% \begin{tabular*}{0.95\textwidth}%
%     {@{\extracolsep{\fill}}cccccc}
%  \hline \hline \textbf{Number of CPU cores}, $\mathbf{N_\text{core}}$                          &  \textbf{Memory (MB)}  &  \textbf{Wall time (s)} &   \textbf{Speed up} \\
% \hline                                                    1                                                                                   & 526                                  &   160.75                       &  - \\
%                                                               2                                                                                   & 566                                  &   94.39                         & 1.70 \\
%                                                               4                                                                                   & 592                                  &   50.47                         & 3.19 \\
%  \hline
% \end{tabular*}
% \end{threeparttable}
%\end{table}

%\newpage


\subsection{Photon Monte-Carlo model}
\label{sec:photon_monte_carlo}

A photon Monte-Carlo model based on the model described by Wang and Modest~\cite{WM2007} has been implemented in \texttt{Eilmer3}.
Both line-by-line and gray gas spectral models can be considered.
This model is similar to the modified Discrete Transfer model described in the previous section in that a ray-tracing approach is used.
The Monte-Carlo approach, however, treats the emission and absorption of photons in a statistical manner and has the advantage of being easily adapted to flowfields with strongly varying gas properties.

\subsubsection{Governing equations}

The total radiative divergence for a finite-volume cell is calculated as the difference between the total emissive power $E_{\text{ems.}}$ and absorptive power $E_{\text{abs.}}$ divided by the cell volume $V$:

\begin{equation}
 - \nabla \cdot \vec{q}_\text{rad} = \frac{ - \left ( E_{\text{ems.}} - E_{\text{abs.}} \right ) }{V}
 \label{eq:my_divq}
\end{equation}

\noindent where:

\begin{eqnarray}
 E_{\text{ems.}} &=& \int_{V} \int_{0}^{{4\pi}} \int_{\nu_{\text{min}}}^{\nu_{\text{max}}} j_{\nu} d\nu d\omega dV = \sum_{p}^{N_{\text{ems. photons}}} E_{p}^0 \label{eq:E_emission} \\
 E_{\text{abs.}} &=& \sum_{p}^{N_{\text{abs. photons}}} \left ( - \Delta E_{p} \right ) \label{eq:E_absorption} 
\end{eqnarray}

\noindent Here $N_{\text{abs. photons}}$ is the total number of photon packets with energy being absorbed by the current cell and $N_{\text{ems. photons}}$ is the total number of photons emitted by this cell.
$E_{p}^0$ is the initial power carried by photon packet $p$:

\begin{equation}
 E_{p}^0 = \frac{ 4 \pi V  }{ N_{\text{ems. photons} } } \int_{\nu_{\text{min}}}^{\nu_{\text{max}}} j_{\nu} d \nu
\end{equation}

The radiative heat flux incident on wall elements $q_\text{rad}$ is calculated as the sum of the remaining energy $E_{p}$ from all incident photons $N_\text{inc. photons}$ divided by the wall element area $A$:

\begin{equation}
 q_\text{rad} = \frac{ E_\text{abs.} }{A} =  \frac{ \displaystyle \sum_{p}^{N_{\text{inc. photons}}} \displaystyle E_{p} }{ A }
 \label{eq:my_divq}
\end{equation}

\subsubsection{Non-gray spectral models}

For non-gray spectral models (e.g. line-by-line representation), each photon is assigned a statistically determined frequency $\nu_p$ that is determined by satisfying the following relation:

\begin{equation}
R_{\nu,p} = \frac{\int_{\nu_{\text{min}}}^{\nu_p} j_{\nu} d \nu}{\int_{\nu_{\text{min}}}^{\nu_{\text{max}}} j_{\nu} d \nu}
\end{equation}

\noindent where $R_{\nu,p}$ is a uniformly distributed random number in the range [0, 1].

\subsubsection{Absorption models}

Energy absorption is treated according to one of two models:  (1) the standard model~\cite{Mod03}, or (2) the partitioned energy model~\cite{MP77}.
The standard model treats absorption statistically, where all the energy of the photon packet is absorbed in a single cell when the following optical thickness ($\tau_p$) criteria is met for photon $p$:

\begin{equation}
 \tau_p \geq - \text{ln} \left ( 1 - R_{\tau,p} \right ) 
\end{equation}

\noindent where $R_{\tau,p}$ is a uniformly distributed random number in the range [0, 1].
The standard model, however, is not efficient for gases close to the optically thin or optically thick limits.
The partitioned energy model attempts to circumvent this inefficiency by treating absorption deterministically, where the radiative power lost by photon packet $p$ while traversing from points $s_{i}$ to $s_{f}$ along a ray is calculated from Beers law:

\begin{equation}
 - \Delta E_{p} = - ( 1 - e^{-\kappa_p(s) \Delta s } ) E_{p}(s_{i})
 \label{eq:E_p}
\end{equation}

\noindent where $\Delta s = s_{f} - s_{i}$.

\subsubsection{Photon density distribution}

The photon Monte-Carlo approach described by Wang and Modest~\cite{WM2007} determines the the emission point of photons randomly according to the emissive energy distribution in the domain.
This results in more photons being emitted from the more strongly emitting regions of the computational domain.
Due to the complexity of constructing random number relations for the emission locations, the non-statistical approaches described in Section~\ref{sec:ray_density_distribution} are considered in the present implementation to achieve a similar effect.

\subsubsection{Ray-tracing}

Photons are emitted in random directions from the centre of each volume and surface element with direction angles calculated via:

\begin{eqnarray}
 \phi &=& 2 \pi R_{\phi} \\
 \Theta &=& \text{arccos} \left ( \sqrt{ 1 - R_\Theta } \right )
\end{eqnarray}

\noindent where $R_{\phi}$ and $R_{\Theta}$ are uniformally distributed random numbers in the range [0,1].
The ray-tracing and cell-finding techniques described in Section~\ref{sec:ray_tracing} are then applied to trace the path of the photons throughout the computational domain.

\subsubsection{Solution procedure}

The procedure for calculating the radiative divergence fields and heat-flux profiles in Eilmer3  via the photon Monte-Carlo model is described in Figure~\ref{fig:eilmer-PMC-solve-procedure}.

\begin{figure}[htbp]
\small
\begin{center}
\fbox{\parbox{13cm}{
 \begin{enumerate}
  \item Calculate emission and absorption spectra for each cell and wall element
  \item Calculate photon density distributions
  \item Calculate direction angles and frequencies for each photon
  \item Trace each photon packet through the grid
  \begin{enumerate}
   \item Subtract emitted energy from origin cell
   \item Add absorbed energy to each traversed cell
   \item Record exiting energy on wall elements 
  \end{enumerate}
  \item Evaluate $- \nabla \cdot \vec{q}_\text{rad}$ for each cell and $q_\text{rad}$ for each wall element from the results
 \end{enumerate}
}}
\end{center}
\caption{Sequence of operations for calculating the radiative divergence fields and heat-flux profiles in Eilmer3 via the photon Monte-Carlo model.}
\label{fig:eilmer-PMC-solve-procedure}
\end{figure}

This procedure is parallelised via \verb OpenMP  where each processor has access to all data describing the computational domain.
The photon Monte-Carlo calculation is divided amongst the available processors on a cell-by-cell basis when computing spectra, and a ray-by-ray basis when transporting photons.
