% thermochem.tex

\section{Thermochemistry}
\label{thermochem-sec}
The gas-dynamic equations of the previous section are closed by a set of relations
between the various thermodynamic properties of the gas mixture.
This section describes a gas library which provides various models of gas behaviour for use
in the simulation codes.

There are a number of gas models available as part of the gas library. They are desribed here:
\begin{description}
 \item[ideal gas mix] \hspace{1cm} \\
     The ideal gas mix is used to model one or more components, all of which
     have ideal (perfectly elastic collisions and calorically perfect) behaviour.
 \item[equilibrium gas mix] \hspace{1cm} \\
     The equilibrium gas mix models a fixed-composition gas mix which is assumed to be in thermal and
     chemical equilibrium at the local thermodynamic conditions.
 \item[thermally perfect gas mix] \hspace{1cm} \\
     The thermally perfect gas mix models a gas with one or more components, all of which
     have perfect (collisional) behaviour but each have all internal energy modes excited to
     an equilibrium described by a single temperature
 \item[multi-temperature gas mix] \hspace{1cm} \\
     This model represents a mixture of gases with perfect collisional behaviour but the
     the internal energy modes are ascribed different Boltzmann temperatures in order
     to model the effect of thermal nonequilibrium.
 \item[Noble-Abel gas mix] \hspace{1cm} \\
     The Nobel-Abel gas mix model is used in interior ballistics work where high pressures mean
     that real gas effects cannot be neglected.  It models the effect
     of the finite volume occupied by gas particles when computing the equation of state.
     Collisions between gas particles are still considered to be perfectly elastic.
 \item[van der Waals gas mix] \hspace{1cm} \\
     The van der Waals gas mix models real gas effects at high pressures/densities.  The finite volume
     occupied by the gas particles and the non-ideal particle collisions due to van der Waals' forces
     are considered in the model.
\end{description}

In simulating impulse facilities, we typically only use the equilibrium gas and thermally perfect gas models.
Therefore, these two models are discussed in the remainder of this section.
The thermally perfect gas mix is used when we also wish to simulate the effect of finite-rate chemistry
between the gases.
The multi-temperature gas mix has also been used recently to model the X2 facilities~\cite{potter_etal_08}, however, we consider
the validation of this model a work in progress.

\subsection{An equilibrium gas mixture}
At higher temperatures, the gases in impulse facilities will undergo chemical reactions.
For example in air, by 2\,000\,K oxygen molecules will begin to dissociate, and
by 4\,000\,K nitrogen molecules will begin to dissociate.
When the chemical reactions are rapid compared to the flow transit times, we use a model
of the gas which assumes chemical equilibrium.
Along with the assumption of chemical equilibrium, thermal equilibrium is also assumed;
the internal energy modes of the gas rapidly equilibriate with the translational temperature
when compared to the characteristic flow time.
This assumption is most appropriate in the high pressure, high temperature gas that is driven by
the piston against an unruptured diaphgram: the high pressures and temperatures give rise
to rapid changes in chemical composition.

\medskip
The equilibrium gas mix is implemented as a look-up table where the thermodynamic properties
are interpolated (or extrapolated) from a table with indexing based on density and internal
energy.
The look-up table is built for a fixed gas composition over a range of densities
and energies prior to the gas dynamics simulation and is read into memory
at the start of the simulation.
A tool provided in the gas library builds the look-up table by running the CEA2 program~\cite{mcbride_gordon_96}
numerous times.

%
\subsection{A mixture of thermally perfect gases}
As mentioned above, the gas model for a mixture of thermally perfect gases is often used in conjunction
with a finite-rate chemistry simulation.
The thermodynamic relations for the gas mixture are presented here.
The implementation of finite-rate chemical effects is discussed in Section~\ref{sec:chemistry}.

\subsubsection{A single thermally perfect gas}
%
The assumed behaviour of a thermally perfect gas is that
all internal energy modes are in equilibrium at a single temperature.
For atoms this means that the Boltzmann distributions for translational
and electronic energy are governed by one temperature value.
Similarly for molecules, the Boltzmann distributions for translational,
rotational, vibrational and electronic energy are described by
a single temperature value.

\medskip
To model a thermally perfect gas requires a knowledge
of how the gas stores energy as a function of temperature.
It is convenient to have available the specific
heat at constant pressure as a function of temperature, $C_p(T)$.
From this, specific enthalpy of the gas can be computed as
\begin{equation}
h = \int_{T_{ref}}^T C_p(T) dT + h(T_{ref})
\end{equation}
and entropy is given as
\begin{equation}
s = \int_{T_{ref}}^T \frac{C_P(T)}{T} dT + s(T_{ref}).
\end{equation}

\medskip
The transport properties, viscosity and thermal conductivity, can be calculated
as a function of temperature for a single component of the gas mix.
The transport properties for a single component can be combined by an appropriate
mixing rule to give a mixture viscosity and thermal conductivity.

\medskip
In the implementation, a thermally perfect gas is characterised
by five curve fits all of which are functions of temperature:
\begin{enumerate}
   \item specific heat at constant pressure, $C_p(T)$,
   \item enthalpy, $h(T)$,
   \item entropy, $s(T)$,
   \item viscosity, $\mu(T)$, and
   \item thermal conductivity, $k(T)$.
\end{enumerate}
The form of these curve fits follows that used by
McBride and Gordon~\cite{mcbride_gordon_96}.
The curve fits for thermodynamic properties
in non-dimensional form are as follows:
\begin{eqnarray}
 \frac{C_p(T)}{R} & = & a_0 T^{-2} + a_1 T^{-1} + a_2 + a_3 T + a_4 T^2 + a_5 T^3 + a_6 T^4 \\
 \frac{H(T)}{RT} & = & -a_0 T^{-2} + a_1 T^{-1} \log{T} + a_2 + a_3 \frac{T}{2} +
                        a_4 \frac{T^2}{3} + a_5 \frac{T^3}{4} + a_6 \frac{T^4}{5} + \frac{a_7}{T} \\
 \frac{S(T)}{R} & = & -a_0 \frac{T^{-2}}{2} - a_1 T^{-1} + a_2 \log{T} + a_3 T + a_4 \frac{T^2}{2}  +
                        a_5 \frac{T^{3}}{3} + a_6 \frac{T^4}{4} + a_8
\end{eqnarray}
The coefficients for these curve fits are available for a large number of gaseous species in
the CEA program~\cite{mcbride_gordon_96} (and associated database files).
Each of these curve fits are only valid over a limited temperature range.
For example, the thermodynamic curve fits for molecular nitrogen (N$_2$) are comprised of three
segments: 200.0--1000.0\,K, 1000.0--6000.0\,K and 6000.0--20000.0\,K.
Beyond this range the values are extrapolated in this work.
The extrapolations are based on a crude assumption of constant $C_p$ outside of the
range.
Thus the extrapolations are as follows:
\begin{eqnarray*}
  \frac{C_p(T < T_{low})}{R} & = & \frac{C_p(T_{low})}{R}\\
  \frac{C_p(T > T_{high})}{R} & = & \frac{C_p(T_{high} )}{R} \\
  \frac{H(T < T_{low})}{RT} & = & \frac{1}{T} \left\lbrace H(T_{low})T_{low} - C_p(T_{low})( T_{low} - T ) \right\rbrace  \\
  \frac{H(T > T_{high})}{RT} & = & \frac{1}{T} \left\lbrace H(T_{high})T_{high} + C_p(T_{high})( T - T_{high}) \right\rbrace \\
  \frac{S(T < T_{low})}{R} & = & S(T_{low}) - C_p(T_{low}) \log{\left( \frac{T_{low}}{T}\right) }  \\
  \frac{S(T > T_{high})}{R} & = & S(T_{high}) + C_p(T_{high}) \log{\left( \frac{T}{T_{high}}\right) }
\end{eqnarray*}

\medskip
The curve fits for viscosity and thermal conductivity are also in the same form as
that used by the CEA program~\cite{mcbride_gordon_96}.
The curves are as follows.
\begin{equation*}
\log{\mu(T)}  =  a_0 \log{T} + \frac{a_1}{T} + \frac{a_2}{T^2} + a_3
\end{equation*}
\begin{equation*}
\log{k(T)} = b_0 \log{T} + \frac{b_1}{T} + \frac{b_2}{T^2} + b_3
\end{equation*}

\subsubsection{Mixing rules for a collection of thermally perfect gases}
%
The thermodynamic state for a mixture of thermally perfect gases is uniquely
defined by two state variables and the mixture composition.
The internal energy is computed as a mass fraction weighted sum of
individual internal energies,
\begin{equation}
e = \sum_{i=1}^{N} f_i e_i = \sum_{i=1}^{N} f_i \left( h_i - R_i T \right).
\end{equation}
Pressure is computed from Dalton's law of partial pressures,
\begin{equation}
p = \sum_{i=1}^{N} \rho_i R_i T.
\end{equation}
The specific gas constant for the mixture is defined as
\begin{equation}
R = \sum_{i=1}^{N} f_i R_i.
\end{equation}
The calculation of $C_p$ is based on a mass fraction weighted sum of component
specific heats,
\begin{equation}
C_p = \sum_{i=1}^{N} f_i {C_p}_i.
\end{equation}
The specific heat at constant volume is then computed as
\begin{equation}
C_v = C_p - R.
\end{equation}
The ratio of specific heats, $\gamma$, is given by its definition,
\begin{equation}
\gamma = \frac{C_p}{C_v}.
\end{equation}
The frozen sound speed for the mixture, $a$, is calculated as
\begin{equation}
a = \sqrt{\gamma R T}.
\end{equation}

\medskip
During a compressible flow simulation, the values of $\rho$ and $e$ are most
readily available from the conserved quantities that are solved for during
each time increment.
This leads to the specific problem of solving for the thermodynamic
state of the gas mixture given $\rho$, $e$, and the mixture composition, $\overrightarrow{f}$.
However, the formulae previously presented are all explicit in temperature.
We solve for temperature using the Newton iteration technique for zero solving,
\begin{equation}
T_{n+1} = T_n - \frac{f_0(T_n)}{f_0'(T_n)},
\end{equation}
where the zero function, $f_0(T)$, is based on the given internal energy, $e$, and
a guess for internal energy based on temperature,
\begin{equation}
f_0(T) = e_{guess} - e = \sum_{i=1}^{N} f_i \left( h_i - R_i T_{guess} \right) - e.
\end{equation}
Using the fact that ${C_v}_i = \frac{de_i}{dt}$, we can conveniently find the derivative
function for the Newton technique by computing the mixture $C_v$,
\begin{equation}
\frac{df_0(T)}{dT} = \sum_{i=1}^{N} f_i \frac{de_i}{dT} = \sum_{i=1}^{N} f_i {C_v}_i = C_v.
\end{equation}
The Newton iteration is set to converge when the accuracy of the temperature value is
within $\pm 1.0\times10^{-6}$\,K.
Personal experience has shown that this kind of error tolerance is required when temperature
is used in a finite-rate chemistry calculation to compute rates of composition change.

\medskip
The calculation of mixture transport properties is not as straight forward
as the thermodynamic properties.
A mixing rule is required to compute the mixture viscosity and thermal
conductivity.
Wilke's mixing rule~\cite{wilke_50} has been implemented in the work
presented here.
Specifically, the mixing rules used by Gordon and McBride~\cite{gordon_mcbride_94} in the CEA program are used
for calculating mixture transport properties in this work; these rules are a variant of Wilke's original formulation~\cite{wilke_50}.
\begin{equation}
   \mu_{mix} = \sum_{i=1}^N \frac{x_i \mu_i}{x_i + \sum_{\genfrac{}{}{0pt}{}{j=1}{j\neq i}}^N x_j \phi_{ij}}
\end{equation}
and
\begin{equation}
   k_{mix} = \sum_{i=1}^N \frac{x_i k_i}{x_i + \sum_{\genfrac{}{}{0pt}{}{j=1}{j\neq i}}^N x_j \psi_{ij}}
\end{equation}
where $x_i$ is the mole fraction of species $i$.

\medskip
The interaction potentials, $\phi_{ij}$ and $\psi_{ij}$, can be calculated a number of ways.
Again, the formulae suggested by Gordon and McBride~\cite{gordon_mcbride_94} have been used,
\begin{equation}
   \phi_{ij} = \frac{1}{4} \left[ 1 + \left(\frac{\mu_i}{\mu_j}\right)^{1/2} 
               \left(\frac{M_j}{M_i}\right)^{1/4} \right]^2\left(\frac{2M_j}{M_i + M_j}\right)^{1/2}
\end{equation}
and
\begin{equation}
\psi_{ij} = \phi_{ij} \left[ 1 + \frac{2.41(M_i - M_j)(M_i - 0.142M_j)}{(M_i + M_j)^2}\right]
\end{equation}
where $M_i$ and $M_j$ refer to the molecular weights of species $i$ and $j$ respectively.

\medskip
Once the mixture viscosity and thermal conductivity have been computed, it is possible to
compute the mixture Prandtl number from its definition
\begin{equation}
Pr = \frac{\mu C_p}{k}.
\end{equation}

\subsection{Finite-rate chemistry implementation}
\label{sec:chemistry}

\subsubsection{Rates of species change due to chemical reaction}
\label{subsec:chem_rates}
By assuming a collection of simple reversible reactions, the chemically reacting system
can be represented as,
\begin{equation}
   \sum_{i=1}^N \alpha_i X_i \rightleftharpoons \sum_{i=1}^N \beta_i X_i,
\end{equation}
where $\alpha_i$ and $\beta_i$ represent the stoichiometric coefficients for the reactants and
products respectively.
The case of an irreversible reaction is represented by setting the backward rate to zero.
For a given reaction $j$, the rate of concentration change of species $i$ is given as,
\begin{equation}
  \left( \frac{d[X_i]}{dt} \right)_j = \nu_i \left\{ k_f \prod_i [X_i]^{\alpha_i} 
                       - k_b \prod_i [X_i]^{\beta_i} \right\},
\end{equation}
where $\nu_i = \beta_i - \alpha_i$.
By summation over all reactions, $N_r$, the total rate of concentration change is,
\begin{equation}
\label{eqn:chem_ode_system}
  \frac{d[X_i]}{dt} = \sum_{j=1}^{N_r} \left( \frac{d[X_i]}{dt} \right)_j.
\end{equation}
For certain integration schemes it is convenient to have the production and loss
rates available as separate quantities.
In this case,
\begin{equation}
\label{eq:split}
  \frac{d[X_i]}{dt} = q_i - L_i = \sum_{j=1}^{N_r} \dot{\omega}_{app_{i,j}} - \sum_{j=1}^{N_r} \dot{\omega}_{va_{i,j}} 
\end{equation}
The calculation of $\dot{\omega}_{app_{i,j}}$ and $\dot{\omega}_{va_{i,j}}$ depends on the
value of $\nu_i$ in reach reaction $j$ as shown in Table~\ref{tab:omega}.

\begin{table}[ht]
\caption[Chemical production and loss terms]{The form of the chemical production and loss terms based on the value of $\nu_i$}
\label{tab:omega}
\begin{center}
\begin{tabular}{ccc}
\hline \hline
                           & $\nu_i > 0$                          & $\nu_i < 0$  \\ \hline
 $\dot{\omega}_{app_{i}}$  & $\nu_i k_f \prod_i [X_i]^{\alpha_i}$ & $-\nu_i k_b \prod_i [X_i]^{\beta_i}$ \\
 $\dot{\omega}_{va_{i}}$   & $-\nu_i k_b \prod_i [X_i]^{\beta_i}$ & $\nu_i k_f \prod_i [X_i]^{\alpha_i}$ \\ \hline \hline
\end{tabular}
\end{center}
\end{table}

\medskip
The calculation of the reaction rate coefficients, $k_f$ and $k_b$, and the solution methods for the ordinary differential
equation system of species concentration changes are discussed in the subsequent sections.

\subsubsection{Reaction rate coefficients}
%
The reaction rate coefficients for a reaction can be determined by experiment (often
shock tube studies are used) or from theory.
In a great number of cases, estimates of the reaction rate from theory can vary by
orders of magnitude from experimentally determined values.
For this reason, fits to experimental values are most commonly used.

For the single-temperature gas model discussed in this chapter,
the forward reaction rate coefficients are calculated using
the generalised Arrhenius form,
\begin{equation}
\label{eqn:ga}
k_f = A T^{n} \exp{\left( \frac{-E_a}{kT}\right)}
\end{equation}
where $k$ is the Boltzmann constant and $A$, $n$ and $E_a$ are constants
of the model.

\medskip
The backward rate coefficient can also be calculated using a modified
Arrhenius form,
\begin{equation}
k_b = A T^{n} \exp{\left(\frac{-E_a}{kT}\right)}
\end{equation}
or it can be calculated by first calculating the equilibrium constant
for the reaction,
\begin{equation}
k_b = \frac{k_f}{K_{c}}.
\end{equation}
%
If the backward rate coefficient is calculated from the equilibrium constant,
then a method of calculation of the equilibrium constant is required.
The equilibrium constant for a specific reaction can be calculated from curve fits or, as is done in this work,
using the principles of thermodynamics.
The equilibrium constant based on concentration is related to the equilibrium constant
based on pressure by,
\begin{equation}
K_c = K_p \left( \frac{p_{atm}}{\mathcal{R}T} \right)^{\nu}
\end{equation}
where $p_{atm}$ is atmospheric pressure in Pascals, $\mathcal{R}$ is the universal
gas constant, $\nu = \sum_i^{N_S} \nu_i$ and
\begin{equation}
K_p = \exp{\left( \frac{-\Delta G}{\mathcal{R}T} \right) }.
\end{equation}
The derivation of the formula for $K_p$, the equilibrium constant based on partial pressures,
can be found in any introductory text on classical thermodynamics which covers chemical equilibrium.
The differential Gibbs function for the reaction, $\Delta G$, is calculated using
\begin{equation}
\Delta G = \sum_i^{N_s} \nu_i G_i
\end{equation}
where each $G_i$ is computed from the definition of Gibbs free energy,
\begin{equation}
G_i(T) = H_i(T) - T \times S_i(T)
\end{equation}
and $G_i$ is in units of J/mol.
$H_i$ and $S_i$ can be computed in the appropriate units
by using the CEA polynomials and multiplying by $\mathcal{R}T$ and
$\mathcal{R}$ respectively.

\medskip
Some caution should be exercised in the selection and use of reaction rates for a specific flow problem.
In many cases, a set of reaction rates may only be ``tuned'' for a specific problem domain.
This problem of ``tuned'' sets of reaction rates and an explanation for why it arises
is described by Oran and Boris (p. 38 of Ref.~\cite{oran_boris_01}):
\begin{quote}
 A problem that often arises in chemical reactions is that there are fundamental inconsistencies
 in a measured reaction rate.
 For example, there may be experimental measurements of both the forward and reverse rate constants,
 $k_f$ and $k_r$.
 Nonetheless, when either is combined with the equilibrium coefficient for that reaction, the other is not
 produced.
 This appears to represent a violation of equilibrium thermodynamics.
 The explanation is usually that $k_f$ and $k_r$ have been measured at rather different temperatures or
 pressures, and so there are inconsistencies when they are extrapolated outside the regime of validity of
 the experiments.
\end{quote}


\subsubsection{Solving the chemical kinetic ordinary differential equation}
%
The system represented in Equation~\ref{eqn:chem_ode_system} is a system of ordinary
differential equations (ODEs) which can be solved by
an appropriate method.
For certain chemical systems, the governing ODEs form a stiff system
due to rates of change varying by orders of magnitude for certain
species.
For these systems, special methods for stiff ODEs are required.
In this work, four methods for the numerical solution of the ODE system
have been implemented.
\begin{enumerate}
  \item Euler method
  \item modified Euler method
  \item alpha-QSS method, and
  \item Runge-Kutta-Fehlberg method
\end{enumerate}

\medskip
The Euler method and modified Euler method are standard techniques for solving
ODEs and the details can be found in any text dealing with numerical methods and
numerical analysis.
The fourth-order Runge-Kutta method uses a fifth-order error estimate as a means for controlling
the timestep used for integration as proposed by Fehlberg~\cite{fehlberg_69}.
This is particularly efficient for non-stiff systems.

\paragraph{alpha-QSS method}
%
The alpha-QSS (quasi-steady-state) method was proposed in Mott's thesis~\cite{mott_99}.
It is an ODE solver aimed specifically at the problem of stiffness in chemical
systems.
This ODE solver makes use of the forward and backwards rates of concentration change
as calculated by Equation~\ref{eq:split}.
This is a predictor-corrector type scheme in which the corrector is iterated
upon until a desired convergence is achieved.
The predictor and corrector are,
\begin{eqnarray}
\left[X_i\right]^1 & = & \left[X_i\right]^0 + \frac{ \Delta t q_i^0}{1 + \alpha_i^0 \Delta t L_i^0} \\
\left[X_i\right]^{n+1} & = & \left[X_i\right]^0 + \frac{ \Delta t \left(\bar{q}_i - \left[X_i\right]^0 \bar{L}_i \right)}{1 + \bar{\alpha}_i \Delta t \bar{L}_i }.
\end{eqnarray}
In the above equations,
\begin{equation}
\bar{L}_i = \frac{1}{2} \left( L_i^0 + L_i^n \right)
\end{equation}
and
\begin{equation}
\bar{q}_i = \bar{\alpha}_i q_i^n + \left( 1 - \bar{\alpha}_i\right) Q_i^0.
\end{equation}
The key to the scheme is calculating $\alpha$ correctly.
This $\alpha$ parameter controls how the update works on a given species integration.
Note that $\alpha$ is defined as
\begin{equation}
\alpha(L \Delta t) \equiv \frac{ 1 - \left( 1 - e^{-L \Delta t} \right) / (L \Delta t) }{1 - e^{-L \Delta t}}.
\end{equation}
Using Pade's approximation,
\begin{equation}
e^x \approx \frac{360 + 120x + 12x^2}{360 - 240x + 72x^2 - 12x^3 + x^4}
\end{equation}
it is possible to write a form of the expression for $\alpha$ which is more amenable to computation
as the expensive exponential function evaluation is avoided.
The approximation for $\alpha$ becomes,
\begin{equation}
\alpha(L \Delta t) \approx \frac{180 r^3 + 60 r^2 + 11 r + 1}{360 r^3 + 60 r^2 + 12 r + 1}
\end{equation}
where $r \equiv 1/(L \Delta t)$.

\subsubsection{Coupling chemistry effects to the flow solver}
Some details about the coupling of the chemistry effects to the gas dynamics
simulation are provided here.
In an unsteady, time-accurate flow simulation, the
allowable timestep is constrained by the Courant-Friedrichs-Lewy
(CFL) criterion.
In a viscous compressible flow, the CFL criterion allows one to
select an appropriate timestep and limit the propagation
of flow information to distances less than one cell-width.
The speed at which flow information propagates is a function
of inviscid wave speeds and viscous effects.

\medskip
When the effects of finite-rate chemistry are `split' from the flow
simulation, the chemical update is solved in a separate step in
which the flow is held frozen. (In fact, in true timestep-splitting, \emph{all}
other contributing physics is frozen during the chemistry update.)
Thus the chemistry problem is to find the updated species composition
at the end of the flow timestep.

\medskip
It may be, and is quite likely, that the flow timestep is not
an appropriate timestep to solve the chemical kinetic ODE problem.
When the timestep for the chemistry problem is smaller than the
flow timestep, the chemistry problem is subcycled a number of times
until the total elapsed time equals that of the flow timestep.
It is common to have simulations where
the chemistry timestep is 100--1000 times smaller than the flow timestep,
that is, 100-1000 subcycles are required to solve the chemistry problem.
When the timstep for the chemistry problem is larger than the flow
timestep, it is simply set to the value of the flow timestep.

\medskip
During the simulation process, the chemistry timestep is tracked for each
finite-volume cell in the simulation.
Although the flow `moves on' in subsequent timesteps, if the change of flow conditions is not large,
then the previous chemistry timestep will be a good estimate to begin the
new chemistry problem in the subsequent timestep.
An exceptional case is when a shock passes through the cell: the change of flow
conditions does become large.
In this instance, the old chemistry step is disregarded and a new step is selected.
The selection procedure for a new step is discussed in the next paragraph.
When using either the Runge-Kutta-Fehlberg or the alpha-QSS methods, an estimate
of the new chemistry timestep is provided as part of the ODE update routine.

\medskip
So, during a simulation, the old chemistry step at one iteration is used to
begin the new chemistry problem in the next iteration.
What is needed is a means to select the chemistry step on the initial iteration,
or whenever the old suggestion is not reasonable (as in the case of a shock passing
through the cell).
In this work, the initial step for the chemistry problem is selected
based on the suggestion by Young and Boris~\cite{young_boris_77},
\begin{equation}
 dt_{\mbox{chem}} = \epsilon_1 \min \left( \frac{[X_i](0)}{[\dot{X}_i](0)} \right)
\end{equation}
where $\epsilon_1$ is taken as $1.0 \times 10^{-3}$ in this work,
and the expression is evaulated at the initial values for the chemistry
subproblem.
Young and Boris~\cite{young_boris_77} suggest that $\epsilon_1$
be scaled from the convergence criteria.
We have found that the fixed value is adequate for the problems
of interest to our research group.

