\documentclass[a4paper,11pt,twoside,onecolumn]{memoir}

% The style of this LaTeX document unashamedly borrows
% inspiration from the book Programming in Lua by
% Roberto Ierusalimschy. My only defence is that
% 'imitation is the sincerest form of flattery.'
%
% Rowan J. Gollan, 03-Sep-2014 

% page layout
\semiisopage[12]
\checkandfixthelayout

% font selection
\usepackage{palatino}

% formatting of section headings
\usepackage[bf,sf,compact]{titlesec}
\titleformat{\chapter}[display]
{\bfseries\sffamily\LARGE}
{\filleft\bfseries\sffamily\Huge\thechapter}
{1ex}
{\vspace{1ex}%
\filleft}

\titleformat{\section}
{\bfseries\sffamily\large}
{\thesection}
{1ex}
{\vspace{0.5ex}}

\titlespacing{\section}{0pt}{2.0ex}{0.5ex}

% To have listings as a float
\usepackage{listings}
\lstset{basicstyle=\footnotesize\ttfamily}
\usepackage{float}
\floatstyle{boxed}
\newfloat{Listing}{h}{lol}

% Some convenient macros
\usepackage{xspace}
\newcommand{\dlang}{\texttt{dlang}\xspace}
\newcommand{\type}[1]{\texttt{#1}}

\begin{document}

\frontmatter

\begin{titlingpage}
\aliaspagestyle{titlingpage}{empty}
\raggedleft
\vspace*{\baselineskip}
{\Large\sffamily Rowan J. Gollan and Peter A. Jacobs}\\[0.2\textheight]
{\bfseries\sffamily The Developer's Guide to the}\\[\baselineskip]
{\Huge\sffamily \texttt{dlang} Gas Library}\par
\vspace*{0.1\textheight}
{\Large\sffamily September 2014}
\vfill
\centering
{\large School of Mechanical \& Mining Engineering\\The University of Queensland}
\end{titlingpage}

\chapter{Preface}

\tableofcontents

\mainmatter

\chapter{Introduction}

The \dlang gas module provides services and data structures that model
the behaviour of a continuum gas.
This module is used primarily as a service module in application codes that require
modelling of complex gases.

This document describes the source code in the gas module.
We will describe what is in the module (the layout of code),
why things are the way they are (the code design), 
and give hints on how to add new models and features.
The module itself was designed in a top-down manner and
it seems logical to organise this descriptive document
in that manner also.
The remainder of this Introduction describes the two
centrepiece structures with which the calling code interacts.
In other words, this first chapter serves as a mini API documentation,
if you will.
The two centrepiece structures are the \type{GasState}  
and the \type{GasModel}.
The remainder of the document is for developers who want
to understand the internals of the module.

\section{Overview of source code layout}

\section{GasState data structure}
\label{sec:GasState}

The \type{GasState}\index{GasState} data structure, as its name implies,
holds data members related to describing the state of the gas.
The data members included \type{GasState} are all set as \type{public}
members so that they may be freely manipulated by the owner of a 
\type{GasState} object. This is a deliberate choice; the \type{GasState}
is meant to function as a simple container for data related to the
gas.
The particular members and their units are shown in the source code
snippet included in Listing~\ref{lst:GasState} which has been extracted
from the source file \type{gasmodel.d}.

\begin{Listing}
\lstinputlisting{listings/gasstate.txt}
\caption{Excerpt from \type{gasmodel.d} showing member data in \type{GasState} struct.}
\label{lst:GasState}
\end{Listing}

If you are wondering why the energy, temperature and thermal conductivity are
vector fields it is because the \type{GasState} is designed to suit a 
general multi-temperature hypersonic flow solver.
Note that we omit certain common thermodynamic values from the \type{GasState} struct such as enthalpy and entropy.
One could also make an argument for including Gibbs free energy, and the list
could go on.
Our choice to limit the number of members in the data struct is motivated
by memory concerns.
The \type{GasState} struct is used many times in out CFD application code:
it is an embedded member of our FlowState object.
Thus, each additional member variable in the \type{GasState} incurs a direct
memory cost proportional to the number of grid points in a CFD mesh.
Our loose criterion for including certain members is based on those values
that are used most frequently by our CFD code.
For all other values, like entropy and enthalpy, we choose to compute them
as needed.
Bear this reasoning in mind if you think the \type{GasState} could benefit
from another data member in the structure.


\section{GasModel class}

\chapter{Thermodynamics and Equations of State}

\chapter{Diffusion Coefficients}

\chapter{Chemical Kinetics}


\end{document}
