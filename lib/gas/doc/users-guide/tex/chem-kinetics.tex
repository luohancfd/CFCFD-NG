\chapter{Chemical Kinetics}
\label{chap:chem-kinetics}

Chemical nonequilibrium flows, also called flows with finite-rate chemical effects,
are an important class of flows in the study of gas dynamics.
On a kinetic level, chemical nonequilibrium exists in the gas when there have not been 
sufficient particle collisions to reach chemical equilibrium
at a given pressure and temperature.
In terms of modelling fluid dynamics, the influence of chemical nonequilibrium
effects is important when the time for completion of the chemical reactions
is of the same order of magnitude as some characteristic flow time scale.
Consider
$\tau_f$ as a characteristic transit time for the flow and $\tau_c$ as a characteristic chemical relaxation time; this leads to a convenient classification of types of flows:
\begin{description}
  \item[frozen flow] the time taken for chemical relaxation is much longer
                     than the flow transit time of interest
                     \[ \tau_c \gg \tau_f \]
  \item[nonequilibrium flow] the time taken for chemical relaxation is comparable
                             to the time for flow transit (and influential on that flow)
                             \[ \tau_c \approx \tau_f \]
  \item[equilibrium flow] the time taken for chemical relaxation time is much quicker
                          than the flow transit time
                          \[ \tau_c \ll \tau_f \]
\end{description}

In gas dynamics, the influence of chemical nonequilibrium processes
is important in a number of classes of flow such as:
\begin{itemize}
\item blunt body flows during planetary-entry;
\item combustion processes;
\item detonation waves; and
\item nozzle flows.
\end{itemize}
The same basic theoretical treatment
can be applied to each of theses types of flows.
In this section, the model for computing 
the gas phase chemical reactions in nonequilibrium flows is described.


\section{Rates of species change}
\label{subsec:chem_rates}
By assuming a set of simple reversible reactions,
the chemically reacting system can be represented as,
\begin{equation}
   \sum_{i=1}^N \alpha_i X_i \rightleftharpoons \sum_{i=1}^N \beta_i X_i,
\end{equation}
where $\alpha_i$ and $\beta_i$ represent the stoichiometric coefficients for the reactants and
products respectively, and $X_i$ is the concentration of species $i$.
The case of an irreversible reaction is represented by setting the backward rate to zero.
For a given reaction $j$, the rate of concentration change of species $i$ is given as,
\begin{equation}
  \left( \frac{d[X_i]}{dt} \right)_j = \nu_i \left\{ k_f \prod_i [X_i]^{\alpha_i} 
                       - k_b \prod_i [X_i]^{\beta_i} \right\},
\end{equation}
where $\nu_i = \beta_i - \alpha_i$.
By summation over all reactions, $N_r$, the total rate of concentration change is,
\begin{equation}
\label{eqn:chem_ode_system}
  \frac{d[X_i]}{dt} = \sum_{j=1}^{N_r} \left( \frac{d[X_i]}{dt} \right)_j.
\end{equation}
For certain integration schemes it is convenient to have the production and loss
rates available as separate quantities.
In this case,
\begin{equation}
\label{eq:split}
  \frac{d[X_i]}{dt} = q_i - L_i = \sum_{j=1}^{N_r} \dot{\omega}_{app_{i,j}} - \sum_{j=1}^{N_r} \dot{\omega}_{va_{i,j}} 
\end{equation}
The calculation of $\dot{\omega}_{app_{i,j}}$ and $\dot{\omega}_{va_{i,j}}$ depends on the
value of $\nu_i$ in reach reaction $j$ as shown in Table~\ref{tab:omega}.

\begin{table}[h]
\caption[Chemical production and loss terms]{The form of the chemical production and loss terms based on the value of $\nu_i$}
\label{tab:omega}
\begin{center}
\begin{tabular}{ccc}
\hline \hline
                           & $\nu_i > 0$                          & $\nu_i < 0$  \\ \hline
 $\dot{\omega}_{app_{i}}$  & $\nu_i k_f \prod_i [X_i]^{\alpha_i}$ & $-\nu_i k_b \prod_i [X_i]^{\beta_i}$ \\
 $\dot{\omega}_{va_{i}}$   & $-\nu_i k_b \prod_i [X_i]^{\beta_i}$ & $\nu_i k_f \prod_i [X_i]^{\alpha_i}$ \\ \hline \hline
\end{tabular}
\end{center}
\end{table}

The calculation of the reaction rate coefficients, $k_f$ and $k_b$, is described in
Section~\ref{sec:reac_coeffs} and the solution methods for the ordinary differential
equation system of species concentration changes is presented in Section~\ref{sec:ODE}.

\section{Reaction rate coefficients}
\label{sec:reac_coeffs}

The reaction rate coefficients for a reaction can be determined by experiment (often
shock tube studies are used) or from theory.
In a great number of cases, estimates of the reaction rate from theory can vary by
orders of magnitude from experimentally determined values.
For this reason, fits to experimental values are most commonly used.
The methods for computing reaction rate coefficients available in the
gas library are described below.

\paragraph{A note on the units of reaction rate coefficients}
Explain the use of CGS versus MKS.


\subsection{Generalised Arrhenius rate coefficients}
\paragraph{Description}
The Arrhenius rate form is used to approximate reaction
rates derived from either experimental data
or theoretical calculations.

\paragraph{Equations}

The generalised Arrhenius expression for a rate coefficient, either forwards or backwards, is
\begin{equation}
\label{eqn:ga}
k = A T^{n} \exp{\left( \frac{-E_a}{k_bT}\right)}
\end{equation}
where $k_b$ is the Boltzmann constant and $A$, $n$ and $E_a$ are constants
of the model.
This may be written equivalently in terms of an activation
temperature, $T_a = E_a/k_b$, as
\begin{equation}
k = A T^{n} \exp{\left( \frac{-T_a}{T}\right)} \text{ . }
\end{equation}


The original Arrhenius formula does not include the pre-exponentiation
temperature cofactor, $T^n$.
In the gas library, the simpler Arrhenius formula is not
a separate modelling option.
Instead, one sets $n = 0$, to get
\begin{equation}
k = A \exp{\left( \frac{-E_a}{k_bT}\right)} \text{ . }
\end{equation}

\paragraph{Input parameters}

\begin{table}[h!]
\caption{Parameters for the generalised Arrhenius rate formula}
\label{tab:ga-formula}
\begin{tabular}{llp{10cm}}
\toprule
Parameter & Units & Description \\ \midrule
\multicolumn{3}{l}{\textit{User input}} \\
$A$       & ${}^*$ & pre-exponential coefficient in c.g.s units.
                    ${}^*$ The units vary depending on the particular
                    elementary reaction. For example, for bimolecular reactions,
                    the units of $A$ are cm$^3$/(mole.s). \\
$n$       &  --    & the power of the pre-exponentiation temperature cofactor \\
$T_a$     & K      & activation temperature \\
\bottomrule
\end{tabular}
\end{table}
When the reaction rate coefficient is given in terms
of activation energy, $E_a$, the equivalent activation
temperature is calculate as
\[ T_a = \frac{E_a}{k_b} \text{ . } \]


\subsection{Using the equilibrium constant to compute rate coefficients}
\paragraph{Description}
The Law of Mass Action can be used to relate the forwards
and backwards rate coefficient from the equilibrium constant.
This is useful when the reaction rate coefficient
is only given for one direction of the reaction.

\paragraph{Equations}
Given either a forwards or backwards rate coefficient, the
other coefficient may be obtained from the expression
\begin{equation}
K_c = \frac{k_f}{k_b}
\end{equation}
where $K_c$ is the equilibrium constant calculated based
on concentrations.

The equilibrium constant is sometimes computed from curve fits
or can be computed from thermodynamics principles.
In the gas library implementation, the equilibrium constant
is computed from first principles.
The equilibrium constant based on concentration is related to the equilibrium constant
based on pressure by,
\begin{equation}
K_c = K_p \left( \frac{p_{atm}}{R_uT} \right)^{\nu}
\end{equation}
where $p_{atm}$ is atmospheric pressure in Pascals, $R_u$ is the universal
gas constant, and $\nu = \sum_i^{N_S} \nu_i$.
The equilibrium constant based on pressure is
\begin{equation}
K_p = \exp{\left( \frac{-\Delta G^*(T)}{R_uT} \right) }.
\end{equation}
The derivation of the formula for $K_p$,
the equilibrium constant based on partial pressures,
can be found in any introductory text on classical thermodynamics
which covers chemical equilibrium.
The standard state differential Gibbs function for the reaction,
$\Delta G^*(T)$, is defined as
\begin{equation}
\Delta G^*(T) = \sum_i^{N_s} \nu_i \bar{g}_i^*(T)
\end{equation}
where each $\bar{g}_i^*$ is the molal Gibbs free energy 
at a pressure of one atmosphere,
\begin{equation}
\bar{g}_i^*(T) = \bar{h}_i(T) - T \bar{s}_i(T) \text{ . }
\end{equation}

\paragraph{Input parameters}
No input parameters are required.
The necessary inputs are part of the thermodynamics
model input.

\subsection{Pressure-dependent rate coefficients}
\paragraph{Description}
Reaction rate coefficients are strongly temperature dependent,
however the rates for certain reactions are additionally
dependent on the third body concentration.
These are pressure-dependent rate coefficients.
An example of a pressure dependent reaction is the
decomposition of the methane molecule, written
here as a unimolecular reaction:
%\[ \text{CH}_4 \leftrightharpoonsfill{2.5em} \text{CH}_3 + \text{H} \]
At low pressures, the rate of methane decomposition behaves as a bimolecular
reaction (the second modelcule being an arbitrary third-body, not shown
in the reaction equation).
At high pressures, the reaction proceeds at a rate as if the reaction
were a unimolecular one, as shown.
Rather than alter the reaction mechanism itself as the
pressure varies between low limits and high limits,
the transitioning behaviour between bimolecular to unimolecular reaction can be
accounted for in the calculation of the reaction rate coefficient.
Thus, a pressure-dependent reaction rate coefficient is used for these
types of reactions.

Pressure-dependent coefficients transition
between two limiting Arrhenius coefficients depending on the third body value.
This transition is due to the fact their reactions exhibit
both unimolecular and bimolecular 
(also known as first and second order) characteristics.
The rate coefficient tends to a constant value, $k_\infty$,
as the molar density goes to infinity.
At high temperatures, essentially all unimolecular
reactions exhibit this behaviour~\cite{gilbert_etal_1983}. 

Pressure-dependent reactions are typically denoted in
the literature using parentheses around the third body, as in
%\[
% \text{A} \left( + \text{M} \right) \autorightleftharpoons{$k_f$}{$k_r$} \text{B} + \text{C} \left( + \text{M} \right)~\text{.}
%\]
This signifies that the low pressure reaction involves two participants,
whereas the high pressure reaction involves only one.
Care must be taken when evaluating the units of the rate coefficient
for this reaction.
Pressure-dependent rate coefficients commonly take one of two forms: Lindemann-Hinshelwood or Troe.

\subsubsection{Lindemann-Hinshelwood form of pressure-dependent rate coefficients}
\paragraph{Equations}
The Lindemann-Hinshelwood form of the forward rate coefficient is given by
\begin{equation}
k_f=\frac{k_\infty k_0[\text{M}]}{k_\infty + k_0[\text{M}]}=
 \begin{cases}
  k_0[\text{M}] &\lim_{[\text{M}]\to 0}\\
  k_\infty &\lim_{[\text{M}]\to \infty}
 \end{cases}\quad\text{,}
\end{equation}
where the rate coefficients $k_0$ and $k_\infty$ are provided
by the kinetic mechanism in Arrhenius form. 

\paragraph{Input parameters}
\begin{table}[h!]
\caption{Parameters for the Lindemann-Hinshelwood form of pressure-dependent rate coefficients}
\label{tab:lh-pd}
\begin{tabular}{lp{10cm}}
\toprule
Parameter & Description \\ \midrule
\multicolumn{2}{l}{\textit{User input}} \\
$k_0$       & Arrhenius rate constants (see Table~\ref{tab:ga-formula}) at the
              low pressure limit \\
$k_{\infty}$ & Arrhenius rate constants (see Table~\ref{tab:ga-formula}) at the
              high pressure limit \\
\bottomrule
\end{tabular}
\end{table}

\subsubsection{Troe form of pressure-dependent rate coefficients}

\paragraph{Equations}
The Troe form is based on the Lindemann-Hinshelwood form but includes strong and weak collision broadening effects in one broadening factor, $F$ \cite{gilbert_etal_1983}.
\begin{equation}
 k_f=\frac{k_\infty k_0[\text{M}]}{k_\infty + k_0[\text{M}]} F
\end{equation}
$F$ may be calculated using
\begin{equation}
\log F=
% \begin{cases}
%  0 &\text{Lindemann Form}\\
  \left[ 1+ \left( \frac{\log P_r + c}{n-d\left(\log P_r + c\right)}\right)^2 \right]^{-1} \log F_{cent}% &\text{Troe Form}
% \end{cases}
\quad \text{,}
\end{equation}
where
\begin{equation}
\label{eqn:troe-params}
 \begin{aligned}
  P_r &= \frac{k_0[\text{M}]}{k_\infty} \\
  c &= -0.4-0.67\log F_{cent} \\
  n &= 0.75-1.27\log F_{cent} \\
  d &= 0.14 \quad \text{and}\\
  F_{cent} &= \left( 1-a \right) \exp \left( -\frac{T}{T^{***}} \right) + a \exp \left( -\frac{T}{T^*} \right) + \exp \left( -\frac{T^{**}}{T} \right)~~\text{.}
 \end{aligned}
\end{equation}

\paragraph{Input parameters}
\begin{table}[h!]
\centering
\caption{Parameters for the Troe form of pressure-dependent rate coefficients}
\label{tab:troe-pd}
\begin{tabular}{llp{10cm}}
\toprule
Parameter & Units & Description \\ \midrule
\multicolumn{3}{l}{\textit{User input}} \\
$k_0$       & - & Arrhenius rate constants (see Table~\ref{tab:ga-formula}) at the
              low pressure limit \\
$k_{\infty}$ & - & Arrhenius rate constants (see Table~\ref{tab:ga-formula}) at the
              high pressure limit \\
$a$         & - & parameter of the model as shown in Equation~\ref{eqn:troe-params} \\
$T1$       & K & value for $T^*$ in Equation~\ref{eqn:troe-params} \\
$T2$     & K & value for $T^{**}$ in Equation~\ref{eqn:troe-params} \\
$T3$    & K & value for $T^{***}$ in Equation~\ref{eqn:troe-params} \\
\bottomrule
\end{tabular}
\end{table}

\paragraph{Example of pressure-dependent rate}
An example calculation of a pressure-dependent rate
is shown here for methane decomposition.
%\[ \text{H}+\text{CH}_3 (+\text{M}) \rightleftharpoonsfill{2.5em}
% \text{CH}_4 (+\text{M}) \]
The rate parameters, given in Table~\ref{tab:meth-decomp},
are taken from Grimech 3.0.
This is reaction 51 in that set.

\begin{table}[h]
\centering
\caption{Parameters for the pressure-dependent rates for the decomposition of
methane.}
\label{tab:meth-decomp}
\begin{tabular}{ll}
\toprule
Parameters & Values \\ \midrule
$k_0$ & $A$ = 1.390$\times$10$^{16}$, $n$ = -0.534, $E_a$ = 536.0 \\
$k_{\infty}$ & $A$ = 2.620$\times$10$^{33}$, $n$ = -4.76, $E_a$ = 2440 \\
Troe coefficients & $a$ = 0.7830, $T^{***}$ = 74.0, $T^{*}$ = 2941.0, $T^{**}$ = 6964.0 \\ \bottomrule
\end{tabular}
\end{table}

At a fixed temperature, the reaction rate is plotted for a range of
pressures in Figure~\ref{fig:meth-decomp}.
Both the Lindemann-Hinshelwood and Troe forms of the reaction rate
coefficient evaluation are plotted to show the effect of the Troe coefficients.
Also shown are the low pressure and high pressure limits for the 
rate coefficient.
Note how the pressure-dependent forms vary between these two limits
over the range of pressure.

\begin{figure}[ht]
 \centering
% \scalebox{1.0}{%GNUPLOT: LaTeX picture with Postscript
\begin{picture}(0,0)%
\includegraphics{../figs/pd_rates}%
\end{picture}%
\begingroup
\setlength{\unitlength}{0.0200bp}%
\begin{picture}(18000,10800)(0,0)%
\put(3300,1650){\makebox(0,0)[r]{\strut{}$10^{5}$}}%
\put(3300,3525){\makebox(0,0)[r]{\strut{}$10^{6}$}}%
\put(3300,5400){\makebox(0,0)[r]{\strut{}$10^{7}$}}%
\put(3300,7275){\makebox(0,0)[r]{\strut{}$10^{8}$}}%
\put(3300,9150){\makebox(0,0)[r]{\strut{}$10^{9}$}}%
\put(3575,1100){\makebox(0,0){\strut{}$10^{3}$}}%
\put(5842,1100){\makebox(0,0){\strut{}$10^{4}$}}%
\put(8108,1100){\makebox(0,0){\strut{}$10^{5}$}}%
\put(10375,1100){\makebox(0,0){\strut{}$10^{6}$}}%
\put(12642,1100){\makebox(0,0){\strut{}$10^{7}$}}%
\put(14908,1100){\makebox(0,0){\strut{}$10^{8}$}}%
\put(17175,1100){\makebox(0,0){\strut{}$10^{9}$}}%
\put(5660,9700){\makebox(0,0){\strut{}$10^{0}$}}%
\put(7927,9700){\makebox(0,0){\strut{}$10^{1}$}}%
\put(10193,9700){\makebox(0,0){\strut{}$10^{2}$}}%
\put(12460,9700){\makebox(0,0){\strut{}$10^{3}$}}%
\put(14727,9700){\makebox(0,0){\strut{}$10^{4}$}}%
\put(16993,9700){\makebox(0,0){\strut{}$10^{5}$}}%
\put(550,5400){\rotatebox{90}{\makebox(0,0){\strut{}$k$}}}%
\put(10375,275){\makebox(0,0){\strut{}$p$, Pa}}%
\put(10375,10524){\makebox(0,0){\strut{}$\bar{\rho}$, mol.m\pow{-3}}}%
\put(3983,4800){\makebox(0,0)[l]{\strut{}$k_0\bar{\rho}$}}%
\put(14455,8400){\makebox(0,0)[l]{\strut{}$k_{\infty}$}}%
\put(14950,3875){\makebox(0,0)[r]{\strut{}Lindemann}}%
\put(14950,3325){\makebox(0,0)[r]{\strut{}Troe}}%
\put(14950,2775){\makebox(0,0)[r]{\strut{}$k_0\bar{\rho}$}}%
\put(14950,2225){\makebox(0,0)[r]{\strut{}$k_{\infty}$}}%
\end{picture}%
\endgroup
\endinput
}
 \caption{Pressure-dependent rate coefficient and the Arrhenius asymptotes (Reaction 51, GRI-Mech3.0).}
\label{fig:meth-decomp}
\end{figure}



