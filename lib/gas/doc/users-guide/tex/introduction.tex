\chapter{Introduction}
The gas library was first developed to provide a service
to a collection of CFD codes.
The library is intended to provide a level of abstraction
such that the calling CFD code need not be concerned
with the details of how the gas behaviour is modelled
and/or computed.
Examples of code in our collection which call on this
gas library are \texttt{L1d3}\cite{jacobs_L1d}, 
\texttt{Eilmer3}\cite{jacobs_etal_2010}, and
\texttt{Poshax3}.
The first section considers the conservation equations
describing the dynamics of compressible fluid.
From this discussion, we establish what services the
gas library should provide to the calling CFD code.

The second section extends that discussion to include
the conservation equations for a mixture of reacting
gases at high-temperature.
These equations are typically employed to model the 
behaviour of flow in the hypersonic regime.
These extensions to the basic Navier-Stokes compressible
flow equations require additional services from the
gas library.
Along with additional gas properties, the library was
developed to compute rates for chemical species change
and the exchange of energy amongst various modes.

Although originally developed for CFD applications,
with a little extra work the gas library is made
available as a stand-alone utility.
The final section introduces the other facilities
provided by the gas library which are not of direct
use by the calling CFD codes.

\section{Modelling gas behaviour for compressible flow CFD}
\label{sec:cfeqns}
The conservation equations of mass, momentum (Navier-Stokes)
and energy for a single-component compressible flow in
three dimensions are expressed in integral form as
\begin{equation}
 \frac{\partial}{\partial t} \int_{V} U dV = - \oint_{S} \left ( \overline{F}_{i} - \overline{F}_{v} \right ) \cdot \hat{n}~dA + \int_{V} Q dV \text{ , }
 \label{eq:conservation}
\end{equation}
where $S$ is the bounding surface and $\hat{n}$ is the outward-facing unit normal of the control surface.

The vector of conserved quantities is expressed in full as
\begin{equation}
 U = \left [ \begin{array}{c} \rho \\ 
                              \rho u \\ 
                              \rho v \\ 
                              \rho w \\ 
                              \rho E \\ 
             \end{array} \right ] \text{ . }
 \label{eq:U_vector_3D}
\end{equation}
Here, the conserved quantities are respectively density, $x$-momentum per volume, $y$-momentum per volume, total energy per volume.

The inviscid component of the fluxes is
\begin{equation}
 \overline{F}_{i} = \left [ \begin{array}{c}
                               \rho u \\
                               \rho u^{2} + p \\
                               \rho vu \\
                               \rho wu \\
                               \rho Eu + pu \\
                            \end{array} \right ] \hat{i} 
                  + \left [ \begin{array}{c} 
                               \rho v \\
                               \rho uv \\
                               \rho v^{2} + p \\
                               \rho wv \\
                               \rho Ev + pv  \\
                            \end{array} \right ] \hat{j} 
                  + \left [ \begin{array}{c} 
                               \rho w \\
                               \rho uw \\
                               \rho vw \\
                               \rho w^{2} + p \\
                               \rho Ew + pw \\
                            \end{array} \right ] \hat{k} 
                 \text{ . }
 \label{eq:F_i_3D}
\end{equation}
The viscous component is
\begin{eqnarray}
 \overline{F}_{v} & = & \left [ 
                            \begin{array}{c} 
                                0 \\
                                \tau_{xx} \\
                                \tau_{yx} \\
                                \tau_{zx} \\
                                \tau_{xx} u + \tau_{yx} v + \tau_{zx} w + q_{x} \\
                            \end{array}
                          \right ] \hat{i} 
                        + \left [ 
                             \begin{array}{c}
                                 0 \\
                                 \tau_{xy} \\
                                 \tau_{yy} \\
                                 \tau_{zy} \\
                                 \tau_{xy} u + \tau_{yy} v + \tau_{zy} w + q_{y} \\
                             \end{array}
                          \right ] \hat{j}  + \nonumber \\
                  &   &   \left [
                             \begin{array}{c}
                                 0 \\
                                 \tau_{xz} \\
                                 \tau_{yz} \\
                                 \tau_{zz} \\
                                 \tau_{xz} u + \tau_{yz} v + \tau_{zz} w + q_{z} \\
                             \end{array}
                          \right ] \hat{k} \text{ , }
 \label{eq:F_v_3D}
\end{eqnarray}
and the viscous stresses are
\begin{eqnarray}
 \tau_{xx} &=& 2 \mu \frac{\partial u }{\partial x} 
               + \lambda \left ( \frac{\partial u}{\partial x} 
                                 + \frac{\partial v}{\partial y} 
                                 + \frac{\partial w}{\partial z} \right ) \text{ , } \nonumber \\
 \tau_{yy} &=& 2 \mu \frac{\partial v }{\partial y} 
               + \lambda \left ( \frac{\partial u}{\partial x} 
                                 + \frac{\partial v}{\partial y} 
                                 + \frac{\partial w}{\partial z} \right ) \text{ , } \nonumber \\
 \tau_{zz} &=& 2 \mu \frac{\partial w }{\partial z} 
               + \lambda \left ( \frac{\partial u}{\partial x} 
                                 + \frac{\partial v}{\partial y} 
                                 + \frac{\partial w}{\partial z} \right ) \text{ , } \nonumber \\
 \tau_{xy} = \tau_{yx} &=& \mu \left ( \frac{\partial u}{dy} 
                                     + \frac{\partial v}{dx} \right ) \text{ , } \nonumber \\
 \tau_{xz} = \tau_{zx} &=& \mu \left ( \frac{\partial u}{dz} 
                                     + \frac{\partial w}{dx} \right ) \text{ , } \nonumber \\
 \tau_{yz} = \tau_{zy} &=& \mu \left ( \frac{\partial v}{dz} 
                                     + \frac{\partial w}{dy} \right ) \text{ , }
 \label{eq:taus_3D}
\end{eqnarray}
where the secondary viscosity coefficient $\lambda$ is expressed 
in terms of the primary coefficient $\mu$ via Stokes hypothesis, $\lambda = - \frac{2}{3} \mu$. 
The viscous heat fluxes are
\begin{eqnarray}
 q_{x} &=& k \frac{\partial T}{\partial x} \text{ , } \nonumber \\
 q_{y} &=& k \frac{\partial T}{\partial y} \text{ , } \nonumber \\
 q_{z} &=& k \frac{\partial T}{\partial z} \text{ . } \label{eq:qs}
\end{eqnarray}

The vector of source terms, $Q$, is zero when modelling a non-reacting single component gas.
The total energy, $E$, is comprised of internal energy of the gas and its kinetic
energy:
\begin{equation}
 E = e + \frac{1}{2}(u^2 + v^2 + w^2)\text{ . }
\end{equation}

Examining the above equations, there are nine unknown primary variables and only 
five conservation equations.
The nine unknowns are:
\[ \rho,\quad  u,\quad  v,\quad w,\quad p,\quad e,\quad \mu,\quad k,\quad \text{and}\quad T. \]
The gas library provides the last four equations which link the primary
variables.
Specifically, the gas library links thermodynamic properties
by using equations of state and evaluates the transport properties
using constitutive relations.
Typically, the thermodynamics of a gas are characterised by its
p-v-T behaviour, that is,
\begin{equation}
   p = p(\rho, T)
\end{equation}
and by its thermal behaviour
\begin{equation}
   e = e(\rho, T) \text{.}
\end{equation}
The constitutive relations are used to compute the dynamic viscosity
\begin{equation}
 \mu = \mu(T) \text{ , }
\end{equation}
and the thermal conductivity
\begin{equation}
 k = k(T) \text{ . }
\end{equation}
Additionally, the gas library computes the sound speed, $a$.
This quantity is used directly by various flux calculators in
the CFD code.


\section{Extension to high-temperature, reacting gases: hypersonic flows}
At hypersonic flow speeds, certain physical phenomena can no longer
be neglected as was the case when modelling a supersonic compressible
flow.
For example, at higher temperatures, the gas mixture will chemically
react.
There are also certain conditions where the gas will exhibit an 
appreciable degree of thermal nonequilibrium.
At the conditions where these high-temperature effects are important, 
additional conservation equations are used when modelling the flow.
Specifically, conservation of mass equations for each species and conservation
of energy equations for each energy mode are included.
Shown below are the extended set of conservation equations
for a nonequilibrium gas.
The ellipses ($\dots$) are used to denote where the entries in the
vector are unchanged from the equations presented in Section~\ref{sec:cfeqns}.
The vector of conserved quantities extended for a high-temperature
mixture of gases is
\begin{equation}
 U = \left [ \begin{array}{c} 
                 \vdots \\
                 \rho e_{m} \\
                 \rho f_{s}
              \end{array} \right ]
 \label{eq:U_ht}
\end{equation}
where the additional conserved quantities are energy for mode $m$ and
mass density of species $s$.
The inviscid component of the fluxes for a high-temperature mixture
of gases is
\begin{equation}
 \overline{F}_{i} = \left [ \begin{array}{c}
                               \vdots \\
                               \rho e_{m} u \\
                               \rho f_{s} u
                            \end{array} \right ] \hat{i} 
                  + \left [ \begin{array}{c} 
                               \vdots \\
                               \rho e_{m} v \\
                               \rho f_{s} v
                            \end{array} \right ] \hat{j} 
                  + \left [ \begin{array}{c} 
                               \vdots \\
                               \rho e_{m} w \\
                               \rho f_{s} w 
                            \end{array} \right ] \hat{k} \text{ . }
\label{eq:F_i_ht}
\end{equation}
The viscous component is
\begin{eqnarray}
 \overline{F}_{v} & = & \left [ 
                            \begin{array}{c} 
                                \vdots \\
                                q_{x,m} \\
                                J_{x,s}
                            \end{array}
                          \right ] \hat{i} 
                        + \left [ 
                             \begin{array}{c}
                                 \vdots \\
                                 q_{y,m} \\
                                 J_{y,s}
                             \end{array}
                          \right ] \hat{j}  + \nonumber \\
                  &   &   \left [
                             \begin{array}{c}
                                 \vdots \\
                                 q_{z,m} \\
                                 J_{z,s}
                             \end{array}
                          \right ] \hat{k} \text{ , }
 \label{eq:F_v_ht}
\end{eqnarray}
and the viscous heat fluxes are
\begin{eqnarray}
 q_{x,m} &=& k_m \frac{\partial T_m}{\partial x} + J_{x,m} h_m\text{ , } \nonumber \\
 q_{y,m} &=& k_m \frac{\partial T_m}{\partial y} + J_{y,m} h_m\text{ , } \nonumber \\
 q_{z,m} &=& k_m \frac{\partial T_m}{\partial z} + J_{z,m} h_m\text{ . } \label{eq:qs_ht}
\end{eqnarray}
Assuming Fick's law, the diffusive fluxes are
\begin{eqnarray}
J_{x,s} = -\rho D_{s,\text{mix}} \frac{\partial f}{\partial x}\text{ , }\nonumber \\
J_{y,s} = -\rho D_{s,\text{mix}} \frac{\partial f}{\partial y}\text{ , }\nonumber \\
J_{z,s} = -\rho D_{s,\text{mix}} \frac{\partial f}{\partial z}\text{ . }\label{eq:Js_ht}
\end{eqnarray}
The mixture diffusion coefficient for each species, $D_{s,\text{mix}}$, is computed from the binary diffusion coefficients, $D_{A,B}$, for each pair of species in the mix.
Note that the expressions for the diffusive fluxes given in Equation~\ref{eq:Js_ht} can become more complicated if a different model for diffusive fluxes is used.
The point is that in all cases, the binary diffusion coefficients require calculation.
Finally, the source vector now includes terms for the energy exchange mechanisms,
such as vibrational relaxation, and species mass change due to chemical reactions.
\begin{equation}
Q  = \left [ \begin{array}{c} 
             \vdots \\
             \Omega_{m} \\
             \dot{\omega_s}
             \end{array} \right ]
\end{equation}

Again, the gas library assists with the closure of this equation set.
The thermodynamics models are used to compute the energies, $e_m$,
and enthalpies, $h_m$, in the various modes which are considered.
The models for transport properties provide calculations of
the thermal conductivities in each mode, $k_m$, and the binary
diffusion coefficients, $D_{A,B}$.
The gas library also provides modelling of high-temperature kinetic
processes.
These models are used to compute the various components which contribute
to the source terms for energy exchange and rate of species mass change.

\section{Additional gas properties}
Over time, additional functionality not necessarily required by the calling
CFD codes has been added to make this a general purpose library for
gas phase thermodynamics.
Along with the thermodynamic properties already mentioned, the gas library can be used to
calculate a number of other state variables including entropy and the Gibbs function.
The library calculates a number of useful state derivatives.
These are introduced in Chapter~\ref{chap:thermo}.

\section{Structure of the report}

This introduction presented the motivation for this gas library as a service library
for compressible flow CFD codes.
The remainder of Part~I deals with the underlying theory of the models provided
by the gas library.
Chapter~\ref{chap:thermo} presents the theory for the thermodynamics modelling.
The calculation of transport properties is discussed in Chapter~\ref{chap:trans-prop}.
Chapter~\ref{chap:kinetics} concerns itself with the modelling of kinetic processes
in high-temperature gases, namely, chemical reactions and nonequilibrium energy exchange.
The intent of this Theory section is twofold: (1) to give the user an overview of the available modelling options in the gas library, and (2) to serve a reference manual for the models implemented in the library.

The second part of the report serves as a User's Guide.
Those readers who are not concerned with the details of the underlying theory
might be better served by skipping ahead to Part~II.
The User's Guide explains a number of things, including:
\begin{itemize}
\item how to install the library (Chapter~\ref{chap:install});
\item how to setup input files (Chapter~\ref{chap:input});
\item how to call the library from a C++ program (Chapter~\ref{chap:cfd}); and
\item how to use the library as a stand-alone Python module (Chapter~\ref{chap:python}).
\end{itemize}
This guide section concludes with a number of examples in Chapter~\ref{chap:examples}
which demonstrate the use of the library to build small programs.

The appendices contain some useful sections for the advanced user.
Appendix~\ref{app:code} contains the code for every figure displayed
in this report.
Thus, this serves as an extensive set of additional examples.
For gases whose properties are computationally expensive to evaluate,
a look-up table implementation is available.
Appendix~\ref{app:lut} documents the format of look-up tables used
by this code so that users can construct their own look-up table.
The complete application programming interface (API) is documented in
Appendix~\ref{app:api}.












