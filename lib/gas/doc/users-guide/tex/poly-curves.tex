\chapter{Polynomials and curve fits for property evaluations}
\label{chap:poly}

\section{CEA polynomials for thermodynamic properties}
\label{sec:cea-thermo}
The CEA polynomials can be used to evaluate the specific heat
at constant pressure ($C_p$), enthalpy ($h$), and
the temperature-dependent component of entropy ($s^o$).
For a given species, the polynomials are split over various temperature
ranges of validity.
Typically, the coefficients for the polynomials are valid over temperature ranges
from 200.0 -- 1000.0\,K and and 1000.0 -- 6000.0\,K.
For some species, there is a high-temperature range from 6000.0 -- 20,000.0\,K.
The CEA thermodynamics database gives are nine coefficients ($a_0 \ldots a_8$) for each polynomial piece.
Additionally, an enthalpy of formation is listed for each species.
This may be used as the reference enthalpy value in Equation~\ref{eq:tpg_h}.
The polynomials in non-dimensional form are:
\begin{eqnarray}
 \frac{C_p(T)}{R} & = & a_0 T^{-2} + a_1 T^{-1} + a_2 + a_3 T + a_4 T^2 + a_5 T^3 + a_6 T^4 \\
 \frac{h(T)}{RT} & = & -a_0 T^{-2} + a_1 T^{-1} \log{T} + a_2 + a_3 \frac{T}{2} +
                        a_4 \frac{T^2}{3} + a_5 \frac{T^3}{4} + a_6 \frac{T^4}{5} + \frac{a_7}{T} \\
 \frac{s^o(T)}{R} & = & -a_0 \frac{T^{-2}}{2} - a_1 T^{-1} + a_2 \log{T} + a_3 T + a_4 \frac{T^2}{2}  +
                        a_5 \frac{T^{3}}{3} + a_6 \frac{T^4}{4} + a_8
\end{eqnarray}

\section{CEA polynomials for transport properties}
\label{app:cea-visc}
The constants $a_0 \dots a_3$ in the equations below are
available in the CEA databse for transport properties.
When using these expressions, the resulting units for viscosity
are in $\mu$Poise, and the units for thermal conductivity
are in $\mu$W/(cm.K).
For use in the library, the constants should be input exactly
as given in the CEA database.
Internally, the code converts the units for viscosity and thermal
conductivity to give values in S.I.\ units.

\begin{equation}
\log{\mu(T)}  =  a_0 \log{T} + \frac{a_1}{T} + \frac{a_2}{T^2} + a_3
\end{equation}
\begin{equation}
\log{k(T)} = b_0 \log{T} + \frac{b_1}{T} + \frac{b_2}{T^2} + b_3
\end{equation}

