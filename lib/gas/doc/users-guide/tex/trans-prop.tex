\chapter{Transport Properties}
\label{chap:trans-prop}


\section{Viscosity of single components}

\subsection{Sutherland's law for viscosity}

\paragraph{Description}
Sutherland's law for viscosity is an approximation dervied from kinetic theory.
The approximation is based on an idealised intermolecular force
potential.
Sutherland's law is applicable in the low-density limit for gases where
the pressure is much lower than the critical pressure.

\paragraph{Equation}

\begin{equation}
\mu = \mu_{\text{ref}} \left(\frac{T}{T_{\text{ref}}}\right)^{3/2} \frac{T_{\text{ref}} + S}{T + S}
\end{equation}

\paragraph{Parameters}
The parameters of the model are described in Table~\ref{tab:Suth-visc}.
The values of these constants for particular gases can be found in texts
on fluid dynamics (see, for example, White~\cite{white_2006}), transport phenomena and the kinetic theory of gases.

\begin{table}[h]
\caption{Parameters for Sutherland's viscosity law}
\label{tab:Suth-visc}
\begin{tabular}{llp{10cm}}
\toprule
Parameter        & Units     & Description \\ \midrule
$\mu_{\text{ref}}$ & N.s/m$^2$ & reference viscosity \\
$T_{\text{ref}}$   & K         & reference temperature  \\
$S$              & K         & Sutherland's constant; an effective temperature \\
\bottomrule
\end{tabular}
\end{table}

\subsection{Viscosity from curve fits}
Along with thermodynamic curve fits, the CEA program also uses curve fits
for ther transport properties, viscosity and thermal conductivity.
These curve fits are implemented in the library.
The form of these curves and the input parameters are
described in Appendix~\ref{app:cea-visc}.

\section{Thermal conductivity of single components}



\section{Mixing rules}

\section{Diffusion coefficients}


