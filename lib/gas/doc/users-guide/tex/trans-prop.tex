\chapter{Transport Properties}
\label{chap:trans-prop}


\section{Viscosity of single components}

\subsection{Sutherland's law for viscosity}

\paragraph{Description}
Sutherland's law for viscosity is an approximation derived from kinetic theory.
The approximation is based on an idealised intermolecular force
potential.
Sutherland's law is applicable in the low-density limit for gases where
the pressure is much lower than the critical pressure.

\paragraph{Equation}

\begin{equation}
\mu = \mu_{\text{ref}} \left(\frac{T}{T_{\text{ref}}}\right)^{3/2} \frac{T_{\text{ref}} + S}{T + S}
\end{equation}

\paragraph{Parameters}
The parameters of the model are described in Table~\ref{tab:Suth-visc}.
The values of these constants for particular gases can be found in texts
on fluid dynamics (see, for example, White~\cite{white_2006}), transport phenomena and the kinetic theory of gases.

\begin{table}[h]
\caption{Parameters for Sutherland's viscosity law}
\label{tab:Suth-visc}
\begin{tabular}{llp{10cm}}
\toprule
Parameter        & Units     & Description \\ \midrule
$\mu_{\text{ref}}$ & N.s/m$^2$ & reference viscosity \\
$T_{\text{ref}}$   & K         & reference temperature  \\
$S$              & K         & Sutherland's constant; an effective temperature \\
\bottomrule
\end{tabular}
\end{table}

\subsection{Viscosity from curve fits}
Along with thermodynamic curve fits, the CEA program also uses curve fits
for ther transport properties, viscosity and thermal conductivity.
These curve fits are implemented in the library.
The form of these curves and the input parameters are
described in Appendix~\ref{app:cea-visc}.

\section{Thermal conductivity of single components}

\subsection{Sutherland's law for thermal conductivity}

\paragraph{Description}
Analogously to Sutherland's law for viscosity, a similar expression can be
written for the thermal conductivity of a gas.

\paragraph{Equation}

\begin{equation}
k = k_{\text{ref}} \left(\frac{T}{T_{\text{ref}}}\right)^{3/2} \frac{T_{\text{ref}} + S}{T + S}
\end{equation}

\paragraph{Parameters}
The parameters of the model are described in Table~\ref{tab:Suth-k}.
The values of these constants for some common gases are given in the text by White~\cite{white_2006}.

\begin{table}[h]
\caption{Parameters for Sutherland's thermal conductivity law}
\label{tab:Suth-k}
\begin{tabular}{llp{10cm}}
\toprule
Parameter        & Units     & Description \\ \midrule
$k_{\text{ref}}$   & W/(m.K)   & reference thermal conductivity \\
$T_{\text{ref}}$   & K         & reference temperature  \\
$S$              & K         & Sutherland's constant; an effective temperature \\
\bottomrule
\end{tabular}
\end{table}

\subsection{Thermal conductivity from curve fits}
Just as with viscosity, there is curve fit data for the thermal conductivity
of various species available in the CEA program database.
The library implements these curve fits, the form of which is
given in Appendix~\ref{app:cea-visc}

\section{Mixing rules}
The transport properties for a gas mixture may be computed from the individual
component transport properties and the use of an appropriate mixing rule.
In this section, the various mixing rules available in the library
are presented.

\subsection{Wilke's mixing rule}

\paragraph{Description}

At low densities, the semi-empirical mixing rule of Wilke~\cite{wilke_1950}
serves well.
The library implements a modified version of Wilke's rule as used
by Gordon and McBride~\cite{gordon_mcbride_1996} in the CEA program.

\paragraph{Equation}

\begin{equation}
   \mu_{mix} = \sum_{i=1}^N \frac{x_i \mu_i}{x_i + \sum_{\genfrac{}{}{0pt}{}{j=1}{j\neq i}}^N x_j \phi_{ij}}
\end{equation}
and
\begin{equation}
   k_{mix} = \sum_{i=1}^N \frac{x_i k_i}{x_i + \sum_{\genfrac{}{}{0pt}{}{j=1}{j\neq i}}^N x_j \psi_{ij}}
\end{equation}
where $x_i$ is the mole fraction of species $i$.

The interaction potentials, $\phi_{ij}$ and $\psi_{ij}$, can be calculated a number of ways.
Again, the formulae suggested by Gordon and McBride~\cite{gordon_mcbride_94} have been used,
\begin{equation}
   \phi_{ij} = \frac{1}{4} \left[ 1 + \left(\frac{\mu_i}{\mu_j}\right)^{1/2} 
               \left(\frac{M_j}{M_i}\right)^{1/4} \right]^2\left(\frac{2M_j}{M_i + M_j}\right)^{1/2}
\end{equation}
and
\begin{equation}
\psi_{ij} = \phi_{ij} \left[ 1 + \frac{2.41(M_i - M_j)(M_i - 0.142M_j)}{(M_i + M_j)^2}\right]
\end{equation}
where $M_i$ and $M_j$ refer to the molecular weights of species $i$ and $j$ respectively.

\subsection{The mixing rule of Gupta and Yos}

DAN to provide equations.

\section{Diffusion coefficients}
The rigorous way to compute multi-component mass diffusion involves the computation of binary diffusion coefficients for all interacting species.
This section discusses the various models available to compute the binary
difussion coefficient for diffusion of species A into B.

\subsection{Hard sphere model}

To document.

\subsection{Model based on Lennard-Jones potential}

To implement.

\subsection{Gupta-Yos model}





