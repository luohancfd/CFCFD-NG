\chapter*{Preface}
\addcontentsline{toc}{chapter}{Preface}

The development of this gas library first began life in 2002
as C code written as part of the first author's doctoral work.
At the time, the goal was to implement a model of a mixture
of thermally perfect gases which could be hooked up to
our in-house compressible flow solver, \texttt{mb\_cns}.
This gas model implementation was precursor work to
developing modelling capability for chemically-reacting
and radiating hypersonic flows.

Over time the gas library code was expanded and generalised
to model a wider range of gases.
The second incarnation of this library was rewritten as
C++ code in 2006.
This was partly an exercise in experimenting with C++
as an implementation language.
A lot of the gas modelling lends itself to `variation-on-a-theme'
type situations, so the object-oriented facilities seemed
worth experimenting with.
At that time it was used by the main workhorse codes in our
CFD collection: \texttt{L1d2}, for quasi-1D flows;
\texttt{mbcns2}, for two-dimensional flows; and \texttt{elmer2}
for 3D flows.

By and large, the experiment with C++ was a success.
The development of the present version of the library
(the one documented here) began in July, 2008.
The authors had gained a fair amount of C++ experience by that
time --- that is, they had stumbled over some of its hurdles, and learnt
from those mistakes --- and were ready to implement a cleaner design
which offered greater flexibility.
This coincided with the gas models being used in new areas of applications
such interior ballistics modelling and turbine flows.
These applications required that the modelling capability
be extended to real gases, that is, gases where high-density (or high-pressure) effects are important.

\section*{Audience}
The intended audience for this report is for users of the gas library, 
be they direct users or indirect.
Indirect users are those who use the compressible flow CFD collection
and would like to learn more about the underlying gas model.
Direct users are those who want to hook the library into their
own application or use the library directly via the supplied Python
module.
Use of the Python module allows users to build their own stand-alone
programs for specific tasks.
For those who are interested in further development of the library,
there is an accompanying \textit{Developer's Guide}.

The first part of this report presents the modelling options offered
by the gas library and documents the underlying theory.
It is intended that users of the calling CFD code can access this
section as a Theory Manual.
The theory section also provides detailed references to the appropriate
literature.
Thus, when users of the library come to publish the results of their
calculations, all of the references for the modelling theory are
documented here.
The theory section is also useful for all users
as it links the required user inputs (constants, coefficients, etc.)
to the various model equations.
This section is also intended to be somewhat of a reference manual
such that the reader can open to a specific model description
and obtain the desired information.
Because of this, there is some unavoidable repetition in the
text as various concepts and models are explained.

The second part discusses how to install and use the gas library.
For the most part, the indirect users need not be concerned with this
section.
For day-to-day use of the CFD codes, all the information that the
indirect user needs is included in the appropriate guides for the
CFD codes.
Only if the user requires some special gas model not included in
the standard selection would he/she need to consult this part
of the report.

The second part of the report is intended for the direct users of the gas library.
It shows how to install and use the library as part of a calling application.
There is also a section that discusses the user input in detail and how
to use the stand-alone Python module.
A section of examples goes on to show how small specialised programs
may be built using the Python module.
Additionally, the appendix of this report contains the code that produced
every figure in this report.
This collection serves as an extensive set of examples.

\section*{Acknowledgements}
.. list some other people who have contributed either by influence or by use ...








