\chapter{Creating Input Files}
\label{chap:input}

The input data to create a new spectral radiation model is provided in a Lua file.
This file contains all the parameters and spectroscopic data to perform a spectral radiation calculation.
To simplify the creation of this file for the user, the Lua file can be created via the \texttt{script\_rad2.py} Python tool that interfaces with a library of spectroscopic data.
Instructions on how to use this tool can be obtained from the command line via:

\begin{lstlisting}[basicstyle=\ttfamily\small]
$ script_rad2.py --help
Usage: script_rad2.py -i rad_desc.py|--input-script=rad_desc.py
                      -L LUA_output.lua|--LUA-file=LUA_output.lua

Options:
  -h, --help            show this help message and exit
  -i INFILE, --input-script=INFILE
                        input Python script for radiation description
  -L LUAFILE, --LUA-file=LUAFILE
                        output configuration file for 'librad2' C++ module in
                        LUA format
\end{lstlisting}

\noindent where \texttt{rad-model.py} is the name of the user created Python script describing the spectral model and \texttt{rad-model.lua} is the desired name of the resulting Lua file.
A sample user-created Python script for creating a basic radiation model for 11 species, 2 temperature air is shown below:

\noindent \topbar
\lstinputlisting[language=python,caption={Example spectral model input file, \texttt{air-radiators.py}}]{../../../testing/air-radiators.py}
\bottombar

In `\texttt{1.}', the desired spectral model is selected:

\begin{lstlisting}[basicstyle=\ttfamily\small]
# 1. Select the spectral model
gdata.spectral_model = "photaura"
\end{lstlisting}

Here the in-house \texttt{photaura} model has been selected; other available models are \texttt{spradian} and \texttt{parade}.
See Appendix~\ref{} for an example of how to construct an input file for these.
The spectral grid is then defined in '\texttt{2.}':

\begin{lstlisting}[basicstyle=\ttfamily\small]
# 2. Define the spectral grid
gdata.lambda_min = 50.0
gdata.lambda_max = 1000.0
gdata.spectral_points = 95000
\end{lstlisting}

\noindent Here the range $50 \leq \lambda \leq 1000$~nm has been requested, discretised by 95000 equidistant points in frequency space.
Presently only spectral domains with consant frequency intervals are permitted.
Finally, in '\texttt{3.}' the desired radiating species are requested and defined:

\begin{lstlisting}[basicstyle=\ttfamily\small]
# 3. Request and define the radiating species
species   = [ "N2", "N2_plus", "NO", "NO_plus", "O2", "O2_plus",
               "N",  "N_plus",  "O",  "O_plus", "e_minus" ]
radiators = [ "N2", "N2_plus", "NO", "O2", "N", "N_plus",
              "O", "O_plus", "e_minus" ]
for rad_name in radiators:
    rad = gdata.request_radiator(rad_name)
    rad.default_data()
    rad.isp = species.index(rad_name)
    rad.iTe = 1
    if rad.type == "diatomic_radiator":
        rad.iTv = 1
\end{lstlisting}

A radiator is requested from the library with the \texttt{gdata.request\_radiator()} function call.
If the radiator is not present in the library, \texttt{script\_rad2.py} will fail here with an error message indicating what species was not avaiable.  
The \texttt{rad.default\_data()} function requests the nominal electronic level and transition probability set from the library -- for most calculations this will be suitable.
Other data that is set here is the species index in the \texttt{rad.isp} field, the electronic temperature index in the \texttt{rad.iTe} field and the vibrational temperature index in the \texttt{rad.iTv} field.  