\chapter{Introduction}

The general differential form of the radiative transfer equation in a participating medium can be written as~\cite{Mod03}:

 \begin{equation}
  \underbrace{ \frac{1}{c} \frac{ \partial I_\nu }{ \partial t } }_\text{temporal var.} + \underbrace{\frac{ \partial I_\nu }{ \partial s }}_\text {spatial var.} = \underbrace{j_\nu}_\text{emission} - \underbrace{\kappa_\nu I_\nu}_\text{absorption} - \underbrace{ \sigma_{s,\nu} I_\nu + \frac{\sigma_{s,\nu}}{4 \pi} \int I_\nu ( \hat{s}_i ) \phi ( \hat{s}_i, \hat{s} ) d \Omega_i }_\text{scattering} \label{eq:RTE}
 \end{equation}
 
 \noindent This radiation module allows the spectral emission and absorption coefficients, $j_\nu$ and $\kappa_\nu$, to be calculated for high temperatures gases.
 The radiation transport models within the Eilmer3 and Poshax3 programs make use of this radiation module when solving for the radiative divergence and radiative heat fluxes within a given computational domain.
 A number of tools also are provided within the radiation module to solve the radiative transfer equation along a line-of-sight for a non-scattering medium.
 The spectral models and line-of-sight tools can be used via the provided Python programs, or in a user-created Python script by loading the \texttt{radpy} module.

\section{Structure of the report}

Part 1 presents the theoretical formulation of the \texttt{photaura} spectral model for high temperature gases.
In \textsection~\ref{sec:spec_rad} the calculation method for the spectral coefficients is described, while in \textsection~\ref{sec:CR} the collisional-radiative model is described.
Part 2 presents a user guide for the radiation module software.  
