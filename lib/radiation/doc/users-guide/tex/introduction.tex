\chapter{Introduction}

The radiation module has three principle applications:

\begin{enumerate}
 \item Determination of the radiance heat flux on a solid surface (e.g. the TPS of a re-entry vehicle)
 \item Determination of the radiative divergence field for a CFD flowfield (e.g. \texttt{eilmer3})
 \item Determination of radiance spectra along optical lines-of-sight for comparison with spectroscopy measurements
\end{enumerate}

The determination of the radiative divergence field is required in a CFD calculation to account for the effect of energy losses or gains due to radiative processes.
This phenomena is often referred to as radiation-flowfield coupling.  

\section{Surface heating due to radiation}



\section{Radiation-flowfield coupling}

The three-dimensional Navier--Stokes equations can be expressed in integral form as
\begin{equation}
 \frac{\partial}{\partial t} \int_{V} U dV = - \oint_{S} \left ( \overline{F}_{i} - \overline{F}_{v} \right ) \cdot \hat{n}~dA + \int_{V} Q dV \text{ , }
 \label{eq:conservation}
\end{equation}
where $S$ is the bounding surface and $\hat{n}$ is the outward-facing unit normal of the control surface.
The vector of conserved quantities for a multi-component and multi-temperature gas are:
\begin{equation}
 U = \left [ \begin{array}{c} \rho \\ 
                              \rho u \\ 
                              \rho v \\ 
                              \rho w \\ 
                              \rho E \\ 
                              \rho e_m \\
                              \rho_s \\
             \end{array} \right ] \text{ . }
 \label{eq:U_vector_3D}
\end{equation}
Here, the conserved quantities are respectively density, $x$-momentum per volume,
$y$-momentum per volume, $z$-momentum per volume, total energy per volume, energy per volume for nonequilibrium thermal mode $m$ and density for species $s$.
The accompanying vector of source terms for a radiating and thermochemically relaxing gas are:

\begin{equation}
 Q = \left [  \begin{array}{c} 0 \\ 
                  0 \\ 
                  0 \\ 
                  0 \\ 
                   -\nabla \cdot \vec{q}_\text{rad} \\ 
                   -\nabla \cdot \vec{q}_{\text{rad},m} + \Omega_m \\
                   \dot{\omega_{s}}  \\
             \end{array} \right ] \text{ . }
 \label{eq:U_vector_3D}
\end{equation}

\noindent where $\nabla \cdot \vec{q}_\text{rad}$ is the total divergence of the radiative heat flux vector, $\nabla \cdot \vec{q}_{\text{rad},m}$ is the component of the radiative divergence due to thermal mode $m$, $\Omega_m$ is the rate of thermal energy exchange for thermal mode $m$ due to particle collisions and $\dot{\omega_{s}}$ is the chemical species mass production rate.
%When bulk species are considered, the radiation modelling only affects the radiative divergence terms; when electronic level specific species are considered however, the species mass production rates are also influenced by the radiation modelling.

\section{Structure of the report}












