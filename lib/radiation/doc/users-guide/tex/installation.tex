\chapter{Installation}
\label{chap:install}

The core of the radiation module is written in C++, and there are a number of tools written in Python.
The input data is provided to the C++ binaries via Lua files.
Assuming the Compressible-Flow CFD repository is located in \verb ~/cfcfd3-hg , the radiation module can be compiled and installed via:

\begin{lstlisting}[basicstyle=\ttfamily\small]
$ cd ~/cfcfd3-hg/lib/radiation/build
$ make install
\end{lstlisting}

A number of test programs can also be compiled and installed via:

\begin{lstlisting}[basicstyle=\ttfamily\small]
$ cd ~/cfcfd3-hg/lib/radiation/build
$ make test
\end{lstlisting}

If the user has the SPRADIAN07 code available and wishes to use it as the spectral model, this needs to be requested at compile time:

\begin{lstlisting}[basicstyle=\ttfamily\small]
$ cd ~/cfcfd3-hg/lib/radiation/build
$ make WITH_SPRADIAN=1 install
\end{lstlisting}

The SPRADIAN07 source code and input-data files need to be located in \\
\verb ~/cfcfd3-hg/extern/spradian07 :

\begin{lstlisting}[basicstyle=\ttfamily\small]
$ ls ~/cfcfd3-hg/extern/spradian07/
atom.dat     diatom.dat   radipac6.f90 triatom.dat
\end{lstlisting}